\iffalse
\title{Assignment}
\author{K.AKSHAY TEJA}
\section{mcq-single}
\fi 
%Question 16
\item If the constant term in the expansion of $\brak{ 3x^3 - 2x^2 + \frac{5}{x^5} }^{10}$ is $2^k \cdot l$, where $l$ is an odd integer, then the value of $k$ is equal to: \hfill \brak{June 2022}
\begin{multicols}{4}    
\begin{enumerate}
    \item 6
    \item 7
    \item 8
    \item 9
\end{enumerate}
\end{multicols}

% Question 17
\item $\int_{0}^{5} \cos \brak{\pi \brak{ x - \sbrak{ \frac{x}{2}} } } dx,$ where $\sbrak{ t} $ denotes the greatest integer less than or equal to $t$, is equal to :\hfill \brak{June 2022}

\begin{multicols}{4}    
\begin{enumerate}
    \item $-3$
    \item $-2$
    \item $2$
    \item $0$
\end{enumerate}
\end{multicols}

% Question 18
\item Let PQ be a focal chord of the parabola $y^2 = 4x$ such that it subtends an angle of $\frac{\pi}{2}$ at the point $\brak{3, 0}$. Let the line segment PQ be also a focal chord of the ellipse $E: \frac{x^2}{a^2} + \frac{y^2}{b^2} = 1, a^2 > b^2$. If $e$ is the eccentricity of the ellipse, then the value of $\frac{1}{e^2}$ is equal to:\hfill \brak{June 2022}

\begin{multicols}{4}
\begin{enumerate}
    \item $1 + \sqrt{2}$
    \item $3 + 2\sqrt{2}$
    \item $1 + 2\sqrt{3}$
    \item $4 + 5\sqrt{3}$
\end{enumerate}
\end{multicols}

% Question 19
\item Let the tangent to the circle $C_1: x^2+y^2=2$ at the point $M\brak{-1,1}$ intersects the circle $C_2: \brak{x-3}^2+\brak{y-2}^2=5$ at two distinct points A and B. If the tangents to $C_2$ at the points A and B intersect at N, then the area of the triangle ANB is equal to:\hfill \brak{June 2022}

\begin{multicols}{4}
\begin{enumerate}
\item $\frac{1}{2}$
\item $\frac{2}{3}$
\item  $\frac{1}{6}$
\item $\frac{5}{3}$
\end{enumerate}
\end{multicols}

% Question 20
\item Let the mean and the variance of 5 observations $x_1,x_2,x_3,x_4,x_5$ be $\frac{24}{5}$ and $\frac{194}{25}$ respectively. If the mean and variance of the first 4 observation are $\frac{7}{2}$ and $a$ respectively, then $\brak{4a + x_5}$ is equal to \hfill \brak{June 2022}

\begin{multicols}{4}
\begin{enumerate}
    \item 13
    \item 15
    \item 17
    \item 18
\end{enumerate}
\end{multicols}

% Question 21
\item Let $S = \{z \in \mathbb{C}: \abs{z-2} \leq 1, z\brak{1+i}+\overline{z}\brak{1-i} \leq 2\}.$ Let $\abs{z-4i}$ attains minimum and maximum values, respectively, at $z_1 \in S$ and $z_2 \in S.$ If $5\brak{\abs{z_1}^2+\abs{z_2}^2} = \alpha + \beta\sqrt{5}$, where $\alpha$ and $\beta$ are integers, then the value of $\alpha + \beta$ is equal to \hfill \brak{June 2022}


% Question 22
\item Let $y = y\brak{x}$ be the solution of the differential equation \\$\frac{dy}{dx} + \frac{\sqrt{2}y}{2\cos^4 x - \cos 2x} = xe^{\tan^{-1}\brak{\sqrt{2}\cot 2x}}, \quad 0 < x < \frac{\pi}{2}$ with $y\brak{\frac{\pi}{4}} = \frac{\pi^2}{32}.$ If $y\brak{\frac{\pi}{3}} = \frac{\pi^2}{18}e^{-\tan^{-1}\brak{\alpha}}$, then the value of $3\alpha^2$ is equal to\hfill \brak{June 2022}



% Question 23
\item Let $d$ be the distance between the foot of perpendiculars of the points P$\brak{1, 2, -1}$ and Q$\brak{2, -1, 3}$ on the plane $-x+y+z=1.$ Then $d^2$ is equal to \hfill \brak{June 2022}


% Question 24
\item The number of elements in the set $S = \left\{\theta \in \sbrak{-4\pi, 4\pi} : 3\cos^2 2\theta + 6\cos 2\theta - 10\cos^2 2\theta + 5 = 0 \right\}$ is 
\hfill \brak{June 2022}

% Question 25
\item The number of solutions of the equation $2\theta - \cos^2\theta + \sqrt{2} = 0$ in $R$ is equal to
\hfill \brak{June 2022}


% Question 26
\item 50$ \tan\brak{3 \tan^{-1}\brak{\frac{1}{2}} + 2 \cos^{-1}\brak{\frac{1}{\sqrt{5}}}} + 4\sqrt{2} \tan\brak{\frac{1}{2} \tan^{-1}\brak{2\sqrt{2}}}$\hfill \brak{June 2022}


% Question 27
\item Let $c, k \in R$. If $f\brak{x} = \brak{c+1}x^2 + \brak{1 - c^2}x + 2k$ and $f\brak{x+y} = f\brak{x} + f\brak{y} - xy, \ \forall x, y \in R$, then the value of $\abs{ 2 \sbrak{ f\brak{1} + f\brak{2} + f\brak{3} + \dots + f\brak{20} } }$ is equal to
\hfill \brak{June 2022}

% Question 28
\item Let $H: \frac{x^2}{a^2} - \frac{y^2}{b^2} = 1, a > 0, b > 0$, be a hyperbola such that the sum of lengths of the transverse and the conjugate axes is $4\brak{2\sqrt{2} + \sqrt{14}}$. If the eccentricity of $H$ is $\frac{\sqrt{11}}{2}$, then the value of $a^2 + b^2$ is equal to
\hfill \brak{June 2022}

% Question 29
\item Let $P_1: \overrightarrow{r} \cdot \brak{2\hat{i} + \hat{j} - 3\hat{k}} = 4$ be a plane. Let $P_2$ be another plane which passes through the points $\brak{2, 3, 2}$, $\brak{2, -2, 3}$, and $\brak{1, -4, 2}$. If the direction ratios of the line of intersection of $P_1$ and $P_2$ are $16, \alpha, \beta$, then the value of $\alpha + \beta$ is equal to
\hfill \brak{June 2022}

% Question 30
\item Let $b_1, b_2, b_3, b_4$ be a 4-element permutation with $b_i \in \{1, 2, 3, \ldots, 100\}$ for $1 \leq i \leq 4$ and $b_i \neq b_j$ for $i \neq j$, such that either $b_1, b_2, b_3$ are consecutive integers or $b_2, b_3, b_4$ are consecutive integers. Then the number of such permutations $b_1, b_2, b_3, b_4$ is equal to
\hfill \brak{June 2022}

%\end{enumerate}
%\end{document}
