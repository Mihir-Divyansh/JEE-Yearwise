\iffalse
\title{Assignment 3}
\author{AI24BTECH11018}
\section{integer}
\fi

\item $f: \mathbb{R} \to \mathbb{R}$ satisfy $f\brak{x+y}=2^xf\brak{y}+4^yf\brak{x}$,$\forall x,  y \in \mathbb{R}$.If $f\brak{2}=3$, then $14 \cdot \frac{f^{\prime}\brak{4}}{f^{\prime}\brak{2}}$ is equal to 
\hfill{\brak{\text{June 2022}}}
    \item Let $p$ and $q$ be two real numbers such tht $p+q=3$
and $p^4+q^4=369$. Then $\brak{\frac{1}{p}+\frac{1}{q}}^{-2}$ is equal to 
\hfill{\brak{\text{June 2022}}}
\item if $z^2+z+1=0$, $z \in \mathbb{C}$, then $\abs{\sum_{n=1}^{15}\brak{Z^n+{\brak{-1}^n\frac{1}{Z^n}}}^2}$ is equal to
\hfill{\brak{\text{June 2022}}}
\item Let $\mathbf{X} = \begin{pmatrix}
0 & 1 & 0 \\
0 & 0 & 1 \\
0 & 0 & 0
\end{pmatrix}$ ,$Y=\alpha I+\beta X+\gamma X^2$ and $Z={\alpha^2}I-\alpha \beta X+\brak{{\beta^2}-\alpha \gamma}X^2,\alpha ,\beta ,\gamma \in \mathbb{R}$. if $\mathbf{Y^-1} = \begin{pmatrix}
\frac{1}{5} & \frac{-2}{5} & \frac{1}{5} \\
0 & \frac{1}{5} & \frac{-2}{5} \\
0 & 0 & \frac{1}{5}
\end{pmatrix}$ then $\brak{\alpha -\beta +\gamma}^2$ is equal to
\hfill{\brak{\text{June 2022}}}
\item The total number of $3-digit$ numbers, whose greatest coomon divisor with $36$ is $2$, is
\hfill{\brak{\text{June 2022}}}
\item $\brak{\mathrm{40C_0}}+\brak{\mathrm{41C_1}}+\brak{\mathrm{42C_2}}+\cdots +\brak{\mathrm{60C_{20}}}=\frac{m}{n}\mathrm{60C_{20}}$
\hfill{\brak{\text{june 2022}}}
\item if $a_1$\brak{\textgreater 0},$a_2,a_3,a_4,a_5$ are in a $G\cdot P\cdot ,a_2+a_4=2a_3+1$ and $3a_2+a_3=2a_4$, then $a_2+a_4+2a_5$ is equal to
\hfill{\brak{\text{June 2022}}}
\item The integral $\frac{24}{\pi}\int_{0}^{\sqrt{2}}\frac{\brak{2-x^2}}{\brak{2+x^2}\brak{\sqrt{4+x^4}}}$ is equal to 
\hfill{\brak{\text{June 2022}}}
\item Let a line $L_1$ be tangent to the hyperbola $\frac{x^2}{16}-\frac{y^2}{4}=1$ and let $L_2$ be the line passing through the orgin and perpendicular to $L_1$. If the locus of the point of intersection of $L_1$ and $L_2$ is $\brak{x^2+y^2}^2=\alpha x^2+\beta y^2$, then $\alpha + \beta $ is equal to 
\hfill{\brak{\text{June 2022}}}
\item If the probability that a randomly chosen $6-digit$ number formed by using digits $1$ and $8$ only is a multiple of $21$ is $p$, then $96 p$ is equal to 
\hfill{\brak{\text{June 2022}}}
%\end{enumerate}

%\end{document}
