\iffalse
\title{Assignment 3}
\author{AI24BTECH11018}
\section{mcq-single}
\fi

%\begin{enumerate}
    \item[1.] Let $\overrightarrow{a}=\hat{i}+\hat{j}+2\hat{k}$, $\overrightarrow{b}=2\hat{i}-3\Hat{j}+\hat{k}$ and $\overrightarrow{c}=hat{i}-\hat{j}+\hat{k}$ be three given vectors. Let $\overrightarrow{v}$ be a vector in the plane of $\overrightarrow{a}$ and $\overrightarrow{b}$ whose projection on $\overrightarrow{c}$ is $\frac{2}{\sqrt{3}}$. If $\overrightarrow{v}\cdot\hat{j}=7$, then$\overrightarrow{v}\cdot\brak{\hat{i}+\hat{k}}$ is equal to :
    \begin{enumerate}
        \item 6
        \item 7
        \item 8
        \item 9
    \end{enumerate}
    \hfill{\brak{\text{June 2022}}}
    \item[2.] The mean and standard deviation of 50 observations are 15 and 2 respectively. It was found that one incorrect observation was taken such that the sum of correct and incorrect observation is 70.If the correct mean is 16, then the correct variance is equal to :
    \begin{enumerate}
        \item 10
        \item 36
        \item 43
        \item 60
    \end{enumerate}
    \hfill{\brak{\text{June 2022}}}
    \item[3.] $16\sin{\brak{20^\circ}}\sin{\brak{40^\circ}}\sin{\brak{80^\circ}}$ is equal to :
    \begin{enumerate}
        \item $\sqrt{3}$
        \item $2\sqrt{3}$
        \item $3$
        \item $4\sqrt{3}$
    \end{enumerate}
    \hfill{\brak{\text{June 2022}}}
    \item[4.] If the inverse trignometric functions take principal values, then $\cos^{-1}\brak{\frac{3}{10}\cos\brak{\tan^{-1}\brak{\frac{4}{3}}}+\frac{2}{5}\sin\brak{\tan^{-1}\brak{\frac{4}{3}}}}$ is equal to :
    \begin{enumerate}
        \item 0
        \item $\frac{\pi}{4}$
        \item $\frac{\pi}{3}$
        \item $\frac{\pi}{6}$
    \end{enumerate}
    \hfill{\brak{\text{June 2022}}}
    \item[5.] Let $r \in {p,q,\neg p,\neg q}$ be such that the logical statement $r\lor \brak{\neg p}\implies\brak{p\land q}\lor r$ is a tautology.Then $r$ is equal to :
    \begin{enumerate}
        \item p
        \item q
        \item $\neg p$
        \item $\neg q$
    \end{enumerate}
    \hfill{\brak{\text{June 2022}}}
 %\end{document}
