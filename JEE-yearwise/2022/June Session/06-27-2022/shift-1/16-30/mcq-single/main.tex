\iffalse
\title{Assignment-2}
\author{EE24BTECH11049}
\section{mcq-single}
\fi

%\begin{document}
%\begin{enumerate}
    
    %1st Question 
    \item 
    Five numbers $x_1, x_2, x_3, x_4, x_5$ are randomly selected from the numbers $1, 2, 3,\dots, 18$ and are arranged in the increasing order $\brak{x_1 < x_2 < x_3 < x_4 < x_5}$. The probability that $x_2 = 7$ and $x_4 = 11$ is:

    \hfill{\sbrak{\text{Jun 2022}}}
    
    \begin{enumerate}
    \begin{multicols}{4}	
        \item $\frac{1}{136}$
        \item $\frac{1}{72}$
        \item $\frac{1}{68}$
        \item $\frac{1}{34}$
    \end{multicols}
    \end{enumerate}

    %2nd Question
    \item 
    Let $X$ be a random variable having binomial distribution $\vec{B}\brak{7, p}$. If $\vec{P}\brak{X = 3} = 5\vec{P}\brak{X = 4}$, then the sum of the mean and the variance of $X$ is: 

    \hfill{\sbrak{\text{Jun 2022}}}
    
    \begin{enumerate}
    \begin{multicols}{4}
        \item $\frac{105}{16}$
        \item $\frac{7}{16}$
        \item $\frac{77}{36}$
        \item $\frac{49}{16}$
    \end{multicols}
    \end{enumerate}

    %3rd Question
    \item 
    The value of 
    \begin{align*}
        \cos{\brak{\frac{2\pi}{7}}} + \cos{\brak{\frac{4\pi}{7}}} + \cos{\brak{\frac{6\pi}{7}}} 
    \end{align*}
    is equal to;

    \hfill{\sbrak{\text{Jun 2022}}}
    
    \begin{enumerate}
    \begin{multicols}{4}
        \item $-1$
        \item $-\frac{1}{2}$
        \item $-\frac{1}{3}$
        \item $-\frac{1}{4}$
    \end{multicols}
    \end{enumerate}

    %4th Question
    \item 
    \begin{align*}
        \sin^{-1}{\brak{\sin{\frac{2\pi}{3}}}} + \cos^{-1}{\brak{\cos{\frac{7\pi}{6}}}} + \tan^{-1}{\brak{\tan{\frac{3\pi}{4}}}}
    \end{align*}
    is equal to;

    \hfill{\sbrak{\text{Jun 2022}}}
    
    \begin{enumerate}
    \begin{multicols}{4}
        \item $\frac{11\pi}{12}$
        \item $\frac{17\pi}{12}$
        \item $\frac{31\pi}{12}$
        \item $-\frac{3\pi}{4}$
    \end{multicols}
    \end{enumerate}

    %5th Question
    \item 
    The boolean expression $\brak{ \sim \brak{p \land q}} \lor q$ is equivalent to: 

    \hfill{\sbrak{\text{Jun 2022}}}
    
    \begin{enumerate}
    \begin{multicols}{2}
        \item $q \to \brak{p \land q}$
        \item $p \to q$
        \item $p \to \brak{p \to q}$
        \item $p \to \brak{p \lor q}$
    \end{multicols}
    \end{enumerate}

%\end{enumerate}
%\end{document}
