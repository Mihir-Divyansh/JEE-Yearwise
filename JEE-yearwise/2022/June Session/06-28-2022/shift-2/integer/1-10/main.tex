
\iffalse
    \title{2020}
    \author{EE24BTECH11015}
    \section{integer}
\fi

\item  Let the image of the point $\vec{P}\myvec{1\\ 2\\ 3}$ in the line $L: \frac{x - 6}{3} = \frac{y - 1}{2} = \frac{z - 2}{3}$ be $\vec{Q}$. Let $\vec{R} \myvec{\alpha \\ \beta \\ \gamma}$ be a point that divides internally the line segment PQ in the ratio 1 : 3. Then the value of $22\brak{\alpha + \beta + \gamma}$ is equal to :\hfill\brak{June 2022}
\\


\item Suppose a class has 7 students. The average marks of these students in the mathematics examination is 62, and their variance is 20. A student fails in the examination if he/she gets less than 50 marks, then in worst case, the number of students can fail is:\hfill\brak{June 2022}
\\


\item If one of the diameters of the circle $x^2 + y^2 - 2\sqrt{2}x - 6\sqrt{2}y + 14 = 0$ is a chord of the circle
${\brak{ x - 2\sqrt{2} }}^2 + {\brak{ y - 2\sqrt{2} }}^2 = r^2$ , then the value of $r^2$ is equal to:

\hfill\brak{June 2022}
\\

\item If $\lim_{x \to 1} \frac{\sin\brak{3x^2 - 4x + 1} - x^2 + 1}{2x^3 - 7x^2 + ax + b} = -2$, then the value of $\brak{a - b}$ is equal to:

\hfill\brak{June 2022}
\\

\item Let for $n = 1, 2,\dots, 50, \; S_n$ be the sum of the infinite geometric progression whose first term is $n^2$ and whose common ratio is $\frac{1}{\brak{n + 1}^2}$. Then the value of $\frac{1}{26} + \sum_{n=1}^{50} \brak{ S_n + \frac{2}{n + 1} - n - 1} $ is equal to:\hfill\brak{June 2022}
\\


\item If the system of linear equations $2x - 3y = \gamma + 5, \alpha x + 5y = \beta + 1, \text{ where } \alpha, \beta, \gamma \in R$ has infinitely many solutions, then the value of $\abs{9\alpha + 3\beta + 5\gamma}$ is equal to:

\hfill\brak{June 2022}
\\

\item  Let $A = \myvec{1+\iota && 1\\-\iota &&0} $ where $\iota=\sqrt{-1}.$ Then, the number of elements in the set $ \cbrak{n \in \{1, 2, \ldots, 100\} : A_n = A} $  is:\hfill\brak{June 2022}\\

\item Sum of squares of modulus of all the complex numbers $z$ satisfying ${z} = \iota z^2 + z^2 - z$ is equal to:\hfill\brak{June 2022}
\\

\item Let $S = \cbrak{1, 2, 3, 4}.$ Then the number of elements in the set $\cbrak{f : S \times S \Longrightarrow S : f \text{ is onto and }f\brak{a,b}=f\brak{b,a} \geq a \; \forall \brak{a,b} \in  S \times S}$ is:

\hfill\brak{June 2022}
\\

\item  The maximum number of compound propositions, out of $p \lor r \lor s, \; p \lor r \lor \sim s, \; p \lor \sim q \lor s, \; \sim p \lor \sim r \lor s, \; \sim p \lor \sim r \lor \sim s, \; \sim p \lor q \lor \sim s, \; q \lor r \lor \sim s, \; q \lor \sim r \lor \sim s, \; \sim p \lor \sim q \lor \sim s$ that can be made simultaneously true by an assignment of the truth values to $p, q, r \text{ and } s,$ is equal to:\hfill\brak{June 2022}


