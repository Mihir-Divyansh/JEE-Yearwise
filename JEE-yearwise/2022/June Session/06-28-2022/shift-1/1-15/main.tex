\iffalse
  \title{Assignment}
  \author{ee24btech11030}
  \section{mcq-single}
\fi

%   \begin{enumerate}
\item  If\\
    
    $$ \sum_{k=1}^{31} \binom{31}{k} \binom{31}{k-1} - \sum_{k=1}^{30} \binom{30}{k} \binom{30}{k-1} = \frac{\alpha \cdot (60!)}{(30!) \cdot (31!)} $$\\ where $\alpha$ $\in$ R, then the value of 16$\alpha$ is equal to \hfill{[JUN 2022]}
    \begin{multicols}{4}
    \begin{enumerate}
        \item 1411
        \item 1320
        \item 1615
        \item 1855
    \end{enumerate}
    \end{multicols}
    \bigskip
    \item Let a function f : N $\rightarrow$ N be defined by\\
    f(x) = $\left[\begin{array}{ll}2n& , n = 2,4,6,8,\cdots\\ n - 1 & , n = 3,7,11,15,\cdots\\\frac{n + 1}{2}  &, n = 1,5,9,13,\cdots \end{array}\right.$\\
    then, f is \hfill{[JUN 2022]}
    \begin{enumerate}
        \item One-one but not onto
        \item Onto but not one-one
        \item Neither one-one nor onto
        \item One-one and onto
    \end{enumerate} 
    \bigskip
    \item If the system of linear equations\\
    2x + 3y - z = -2\\
    x + y + z = 4\\
    x - y + $|\lambda|$z = 4$\lambda$ - 4\\
    where $\lambda$$\in$ R, has no solution, then \\\hfill{[JUN 2022]}
    \begin{multicols}{4}
    \begin{enumerate}
        \item $\lambda = 7$
        \item $\lambda = -7$
        \item $\lambda = 8$
        \item $\lambda^2 = 1$
    \end{enumerate} 
    \end{multicols}
    \bigskip
    \item Let A be a matrix of order 3 $\times$ 3 and det (A) = 2. Then det (det (A) adj (5 adj (A3))) is equal to  \hfill{[JUN 2022]}
    \begin{multicols}{2}
    \begin{enumerate}
        \item $512\times 10^6$
        \item $256\times 10^6$
        \item $1024\times 10^6$
        \item $256\times 10^{11}$
    \end{enumerate} 
    \end{multicols}
    \bigskip
    \item he total number of 5-digit numbers, formed by using the digits 1, 2, 3, 5, 6, 7 without repetition, which are multiple of 6, is \hfill{[JUN 2022]}
    \begin{multicols}{4}
    \begin{enumerate}
        \item 36
        \item 48
        \item 60
        \item 72
    \end{enumerate} 
    \end{multicols}
    \bigskip
    \item Let $A_1, A_2, A_3, \cdots $ be an increasing geometric progression of positive real numbers. If $A_1A_3A_5A_7 = \frac{1}{1296}$ and $A_2 + A_4 = \frac{7}{36}$ then, the value of $A_6 + A_8 + A_{10}$ is equal to\hfill{[JUN 2022]} 
    \begin{multicols}{4}
    \begin{enumerate}
        \item $33$
        \item $37$
        \item $43$
        \item $47$
    \end{enumerate} 
    \end{multicols}
    \bigskip
    \item Let [t] denote the greatest integer less than or equal to t. Then, the value of the integral $\int_{0}^{1}[-8x^{2} + 6x - 1] dx$ is equal to \hfill{[JUN 2022]}
    \begin{multicols}{4}
    \begin{enumerate}
        \item $-1$
        \item $\frac{-5}{4}$
        \item $\frac{\sqrt{17} - 13}{8}$
        \item $\frac{\sqrt{17} - 16}{8}$
    \end{enumerate} 
    \end{multicols}
    \bigskip
    \item Let f: $R \rightarrow R$ be defined as\\
    f(x) = $\left[\begin{array}{ll}0& ,  x<0 \\ ae^{x} - 1 & ,  0 \leq x < 1 \\b & , x = 1\\ 
    b - 1 &, 1 < x < 2\\-c &, x \geq 2 \end{array}\right.$\\Where a, b, c $\in$  R and [t] denotes greatest integer less than or equal to t. Then, which of the following statements is true? \hfill{[JUN 2022]}
    \begin{enumerate}
        \item There exists a, b, c $\in$  R  such that f is continuous on $\in$  R .
        \item If f is discontinuous at exactly one point, then a + b + c = 1
        \item If f is discontinuous at exactly one point, then a + b + c $\neq$ 1
        \item f is discontinuous at atleast two points, for any values of a, b and c
    \end{enumerate}
    \bigskip
    \item The area of the region\\
    $\left\{(x,y) : y^2 \leq 8x , y \geq \sqrt{2}x , x \geq 1 \right\}$
    is \hfill{[JUN 2022]}
    \begin{multicols}{4}
    \begin{enumerate}
        \item $\frac{13\sqrt{2}}{6}$
        \item $\frac{11\sqrt{2}}{6}$
        \item $\frac{5\sqrt{2}}{6}$
        \item $\frac{19\sqrt{2}}{6}$
    \end{enumerate} 
    \end{multicols}
    \bigskip
    \item Let the solution curve y = y(x) of the differential equation 
    $\left[\frac{x}{\sqrt{x^2 - y^2}} + e^{\frac{y}{x}}\right]x\frac{dy}{dx} = x + \left[\frac{x}{\sqrt{x^2 - y^2}} + e^{\frac{y}{x}}\right]y$
    pass through the points (1, 0) and $(2\alpha, \alpha)$, $\alpha > 0$. Then $\alpha$ is equal to \hfill{[JUN 2022]}
    \begin{multicols}{2}
    \begin{enumerate}
        \item $\frac{1}{2}exp(\frac{\pi}{6} + \sqrt{e} - 1)$\\
        \item $\frac{1}{2}exp(\frac{\pi}{3} + \sqrt{e} - 1)$
        \item $exp(\frac{\pi}{6} + \sqrt{e} + 1)$\\
        \item $2 exp(\frac{\pi}{3} + \sqrt{e} - 1)$
    \end{enumerate} 
    \end{multicols}
    \bigskip
    \item Let y = y(x) be the solution of the differential equation $x(1-x^2)\frac{dy}{dx} + 3x^2y - y - 4x^3 = 0$ , x$>$1 with y(2) = -2. Then y(3) is equal to \hfill{[JUN 2022]}
    \begin{multicols}{4}
    \begin{enumerate}
        \item -18
        \item -12
        \item -6
        \item -3
    \end{enumerate}
    \end{multicols}
    \bigskip
    \item The number of real solutions of $x^7 + 5x^3 + 3x + 1 = 0$ is equal to \hfill{[JUN 2022]}
    \begin{multicols}{4}
    \begin{enumerate}
        \item 0
        \item 1
        \item 3
        \item 5
    \end{enumerate} 
    \end{multicols}
    \bigskip
    \item Let the eccentricity of the hyperbola\\H : $\frac{x^2}{a^2} + \frac{y^2}{b^2} = 1\\$be $\frac{\sqrt{5}}{2}$ and length of its latus rectum be $6\sqrt{2}$, If $y = 2x + c$ is a tangent to the hyperbola H. then the value of $c^2$ is equal to \hfill{[JUN 2022]}
    \begin{multicols}{4}
    \begin{enumerate}
        \item 18
        \item 20
        \item 24
        \item 32
    \end{enumerate} 
    \end{multicols}
    \bigskip
    \item If the tangents drawn at the points $\vec{O}\myvec{0,0}$ and $\vec{P}\myvec{1 + \sqrt{5}, 2}$ on the circle $x^2 + y^2 - 2x - 4y = 0$ intersect at the point Q, then the area of the triangle OPQ is equal to \hfill{[JUN 2022]}
    \begin{multicols}{4}
    \begin{enumerate}
        \item $\frac{3 + \sqrt{5}}{2}$
        \item $\frac{4 + 2\sqrt{5}}{2}$
        \item $\frac{5 + 3\sqrt{5}}{2}$
        \item $\frac{7 + 3\sqrt{5}}{2}$
    \end{enumerate} 
    \end{multicols}
    \bigskip
    \item If two distinct points Q, R lie on the line of intersection of the planes $-x + 2y - z = 0$ and $3x - 5y + 2z = 0$ and PQ = PR = $\sqrt{18}$ where the point P is (1, -2, 3), then the area of the triangle PQR is equal to \hfill{[JUN 2022]}
    \begin{multicols}{4}
    \begin{enumerate}
        \item $\frac{2}{3}\sqrt{38}$
        \item $\frac{4}{3}\sqrt{38}$
        \item $\frac{8}{3}\sqrt{38}$
        \item $\sqrt{\frac{152}{3}}$
    \end{enumerate} 
    \end{multicols}
    \bigskip
%   \end{enumerate}
