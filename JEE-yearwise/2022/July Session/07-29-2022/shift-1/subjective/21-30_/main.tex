\iffalse
\title{2022}
\author{AI24BTECH11013}
\section{subjective}
\fi
\item Let $S = \theta \in \brak{0,2\pi} : 7 cos^2\theta - 3 sin^2 \theta - 2 cos^2\theta = 2$. Then the sum of roots of all the equations $x^2 - 2 \brak{tan^2\theta + cot^2\theta} x + 6 sin^2\theta = 0 $ $\theta \in S$, is :
\hfill{\brak{July 2022}}
\item Let the mean and the variance of 20 observations $x_1,x_2,....x_20$ be 15 and 9, respectively. For $\alpha \in R$, if the mean of $\brak{x_1 + \alpha}^2,\brak{x_2 + \alpha}^2,...brak{x_20 + \alpha}^2$ is 178, then the square of the maximum value of $\alpha$ is eqal to :
\hfill{\brak{July 2022}}
\item let a line with direction ratios a, -4a, -7 be perpendicular to the lines with direction ratios 3, -1, 2b and b, a, -2. If the point of intersection of the line $\frac{x + 1}{a^2 +b^2} = \frac{y - 2}{a^2 - b^2} = \frac{z}{1}$ and the plane $x - y + z = 0$ is $\brak{\alpha, \beta, \gamma}$, then $\alpha + \beta + \gamma$ is equal to 
\hfill{\brak{July 2022}}
\item Let $a_1, a_2, a_3,....$ be an A.P. If $\sum_{n=1}^{\infty} \frac{a_r}{2^r}=4$, then $4a_2$ is equal to 
\hfill{\brak{July 2022}}
\item Let the ratio of the fifth term from the beginning to the fifth term from the end in the binomial expansion of $\brak{\sqrt[4]{2}+\frac{1}{\sqrt[4]{3}}}^n$, in the increasing powers of $\frac{1}{\sqrt[4]{3}}$ be $\sqrt[4]{6} : 1$. If the sixth term from the beginning is $\frac{\alpha}{\sqrt[4]{3}}$, then $\alpha$ is equal to 
\hfill{\brak{July 2022}}
\item Let number of matrices of order 3 * 3, whose entries are either 0 or 1 and the sum of all the entries is a prime number, is 
\hfill{\brak{July 2022}}
\item Let p and p + 2 be prime numbers and let \\
$\Delta = 
\begin{vmatrix}
p! & \brak{p+1}! & \brak{p+2}! \\
\brak{p+1}! & \brak{p+2}! & \brak{p+3}! \\
\brak{p+2}! & \brak{p+3}! & \brak{p+4}!
\end{vmatrix}$

Then the sum of the maximum values of $\alpha$ and $\beta$, such that $p^\alpha$ and $\brak{p +2}^\beta$ divide $\Delta$, is 
\hfill{\brak{July 2022}}
\item If $\frac{1}{2*3*4} + \frac{1}{3*4*5} + \frac{1}{4*5*6} +....+ \frac{1}{100*101*102} = \frac{k}{101}$, then 34 k is equal to
\hfill{\brak{July 2022}}
\item Let $S = {4,6,9}$ and $T = {9,10,11,....1000}$. If $A = {a_1+a_2+...+a_k : K\in N, a_1,a_2,a_3,...,a_k \in S}$, then the sum of all the elements in the set $T - A$ is equal to 
\hfill{\brak{July 2022}}
\item Let the mirrir image of a circle $c_1 : x^2 +y^2 - 2x - 6y +\alpha = 0$ in line $y = x + 1$ be $c_2 : 5x^2 + 5y^2 +10gx + 10fy + 38 =  0.$ If r is the radius of the circle $c_2$, then $\alpha + 6r^2$ is equal to
\hfill{\brak{July 2022}}

