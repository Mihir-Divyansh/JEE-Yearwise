
\iffalse
  \title{2022}
  \author{Murra Rajesh Kumar Reddy}
  \section{integer}
\fi
    \item Let $A = \cbrak{1,2,3,4,5,6,7}$.Define $B = \cbrak{T \subset A : \text{eitther } 1 \notin T \text{or } 2 \in T}$ and $C = \cbrak{T \subset A : T \text{the sum of all the elements of} T \text{ is a prime number}}$.Then the number of elements in the set $B \cup c$ is .
\item Let $f\brak{x}$ be a quadractic polynomial with leading coefficient $1$ such that $f\brak{0} = p,p \neq 0$. and $f\brak{1}=\frac{1}{3}$.If the equation $f\brak{x}=0$ and $fo fo fo fo f\brak{x} = 0$ have a common real root, then $f\brak{-3}$ is eqaul to .
\item Let $A = \myvec{1&a&a \\ 0&1&b \\ 0&0&1} ,a,b \in R$.If for some $n\in N,A^n = \myvec{1&48&2160 \\ 0&1&96 \\ 0&0&1}$ \\
then $n+a+b$ is equal to .
\item The sum of the maximum and minimum values of the function $f\brak{x} = |5x-7| + \sbrak{x^2+2x}$ in the interval $\sbrak{\frac{5}{4},2}$, where $\sbrak{t}$ is the greatest integer $\leq t$, is .
\item Let $y=y\brak{x}$ be the solution of the differential equation \\
$\frac{dy}{dx} = \frac{4y^3+2yx^2}{3xy^2+x^3},y\brak{1}=1$. \\
If for some $n \in N, y\brak{2} \in \lsbrak{n-1},\rbrak{n}$, then n is equal to .
\item let $f$ be twice differentiable function on R.If $f'\brak{0} = 4$ and ${f\brak{x}+ \int_{0}^{x} \brak{x-1}f'\brak{t}dt =  \brak{e^{2x} + e^{-2x}}\cos{2x} + \frac{2}{a} x}$,then $\brak{2a+5}^5a^2$ is equal to .
\item Let $a_n=\int_{-1}^{n}\brak{1+\frac{x}{2}+\frac{x^2}{3}+\dots+\frac{x^{n-1}}{n}}dx$//
for every $n \in N$.Then the sum of all the elments of the set $\cbrak{n \in N: a_n \in \brak{2,30}}$ is .
\item If the circles $x^2+y^2+6x+8y+16=0$ and ${x^2+y^2+2\brak{3-\sqrt{x}}x+2\brak{4-\sqrt{6}}y = k+6\sqrt{3} +8\sqrt{6},k\ge 0}$,touch internally at the point 
$P\brak{\alpha,\beta}$, then $\brak{\alpha+\sqrt{3}}^2+\brak{\beta+\sqrt{6}}^2$ is equal to .
\item Let the area enclosed by the x-axis, and the tangent and normal drawn to the curve $4x^3-3xy^2+6x^2-5xy-8y^2+9x+14 = 0$ at the point $\brak{-2,3}$ be A. Then $8A$ is equal to .
\item Let $x=\sin\brak{2\tan^{-1}\alpha}$ and $y=\sin\brak{\frac{1}{2}\tan^{-1}\frac{4}{3}}$.If ${S=\cbrak{\alpha \in R : y^2=1-x}}$, then ${\sum_{\alpha \in S} 16\alpha^3}$ is equal to .

