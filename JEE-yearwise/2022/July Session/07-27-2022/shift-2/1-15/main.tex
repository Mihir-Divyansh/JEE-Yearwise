\iffalse
\title{July 2022}
\author{EE24BTECH11058}
\section{mcq-single}
\fi

%\begin{enumerate}
    \item The domain of the function 
    \begin{align*}
        f\brak{x} = \sin^{-1}\sbrak{2x^2 - 3}+ \log_2\brak{\log_{1/2}\brak{x^2 - 5x + 5}}
    \end{align*} where $\sbrak{t}$ is the greatest integer function,is : 
    \hfill(July 2022)
    
    \begin{enumerate}
        \item $\brak{-\sqrt{\frac{5}{2}},\frac{5-\sqrt{5}}{2}}$
        \item $\brak{\frac{5-\sqrt{5}}{2}, \frac{5+\sqrt{5}}{2}}$
        \item $\brak{1,\frac{5-\sqrt{5}}{2}}$ 
        \item $\brak{1,\frac{5+\sqrt{5}}{2}}$ \\
    \end{enumerate}
    

    \item Let S be the set of all $\brak{\alpha,\beta}$ ,$\pi <\alpha, \beta < 2\pi , $ for which the complex number $\frac{1-i\sin\alpha}{1-2i\cos\beta}$ is purely real. Let $Z_{\alpha\beta} =\sin2\alpha + i\cos2\beta,\brak{\alpha,\beta} \in S $. Then $\sum_{\brak{\alpha,\beta}\in S }\brak{iZ_{\alpha\beta} + \frac{1}{i\bar{Z}_{\alpha\beta}}} $ is equal to:
    \hfill(July 2022)
    
    
    \begin{enumerate}
        \item $3$
        \item $3i$
        \item $1$
        \item $2-i$
    \end{enumerate}
    

    \item  If $\alpha, \beta$ are the roots of the equation 
    \begin{align*}
        x^{2} - \brak{5 + 3^{\sqrt{\log_3{5}}} -  5^{\sqrt{\log_5{3}}}}  +  3\brak{3^{(\log_3{5})^\frac{1}{3}} - 5^{(\log_5{3})^\frac{2}{3}} - 1} = 0
    \end{align*} 
    then the equation,whose roots are $\alpha + \frac{1}{\beta} $ and $\beta + \frac{1}{\alpha},$ 
    \hfill(July 2022)
    
    \begin{enumerate}
        \item $3x^2 - 20x -12$
        \item $3x^2 -10x-4$ 
        \item $3x^2 - 10x +2$ 
        \item $3x^2 -20x + 16$   
    \end{enumerate}


     \item Let A = $\myvec{4 & -2 \\ \alpha & \beta}$ If $A^2 + \gamma A + 18I = 0,$ then det(A) is equal to 
     \hfill(July 2022)
     
     \begin{enumerate}
         \item $-18$ 
         \item $18$ 
         \item $-50$ 
         \item $50$     
     \end{enumerate}


     \item If for $p \neq q \neq 0,$ the function 
     \begin{align*}
         f\brak{x} =\frac{\sqrt[7]{p(729 + x)}-3}{\sqrt[3]{729 +qx} -9}  
     \end{align*} 
     is continuous at x=0, then :
     \hfill(July 2022)
     \begin{enumerate}
         \item $7pqf(0)- 1 =0$
         \item $63qf(0)- p^2 =0$
         \item $21qf(0) - p^2 =0$
         \item $7pqf(0)-9 = 0$
     \end{enumerate}


     \item Let
     \begin{align*}
        f\brak{x} =2 + |x| - |x-1| + |x+1| 
     \end{align*} 
     x $\in \textbf{R} $ Consider
     
         $(S1): f^\prime\brak{-{\frac{3}{2}}} + f^\prime\brak{-{\frac{1}{2}}} + f^\prime\brak{{\frac{1}{2}}} f^\prime\brak{{\frac{3}{2}}} = 2$\\
     $(S2) : \int_{-2}^2 f(x) \, dx = 12$ Then,
     \hfill(July 2022)
     \begin{enumerate}
         \item both (S1) and (S2) are correct 
         \item both (S1) and (S2) are wrong 
         \item only (S1) is correct 
        \item only (S2) is correct 
     \end{enumerate}


     \item Let the sum of an infinite G.P. , whose first term is a and the commom ratio is r, be $5$. Let the sum of its first five terms be $\frac{98}{25}.$ Then the sum of the first $21$ terms of an AP, whose first term is $10ar$, $n^{th}$ term is $a_n$ and the common difference is $10ar^{2}$ is equal to :
     \hfill(July 2022)
     \begin{enumerate}
         \item $21a_{11}$
         \item $22a_{11}$
         \item $15a_{16}$
         \item $14 a_{16}$
    \end{enumerate}


    \item The area of the region enclosed by $y \le 4x^2 , x^2 \le 9y$ and $y \le 4$ , is equal to :
    \hfill(July 2022)
    \begin{enumerate}
        \item $\frac{40}{3}$
        \item $\frac{56}{3}$
        \item $\frac{112}{3}$
        \item $\frac{80}{3}$     
    \end{enumerate}


    \item $\int_{0}^2 \brak{|2x^{2} - 3x|+ \sbrak{x -\frac{1}{2}}}\, dx$ where $\sbrak{t}$ is the greatest integer funnction, is equal to :
    \hfill(July 2022)
    \begin{enumerate}
        \item $\frac{7}{6}$ 
        \item $\frac{19}{12}$ 
        \item $\frac{31}{12}$ 
        \item $\frac{3}{2}$
    \end{enumerate}


    \item Consider a curve $y=y(x)$ in the first quadrant as shown in the figure.Let the area $A_1$ is twice the area $A_2$. Then the normal to the curve perpendicular to the line $2x-12y=15$ does $\textbf{NOT}$ pass through the point.
    \hfill(July 2022)
	    \\\begin{center}
		     

 \begin{tikzpicture}
    % Axes
    \draw[->] (-1,0) -- (12,0) node[right] {$x$};
    \draw[->] (0,-1) -- (0,5) node[above] {$y$};

    % Curve
    \draw[domain=4:12, samples=100] plot (\x, {sqrt(\x)}) ; 

    % Lines
    \draw[dashed] (0,2) -- (4,2);
    \draw[dashed] (0,3) -- (9,3);
    \draw[dashed] (4,0) -- (4,2);
    \draw[dashed] (9,0) -- (9,3);

    % Labels
    \node[below] at (4,0) {$4$};
    \node[below] at (9,0) {$x$};
    \node[left] at (0,2) {$2$};
    \node[left] at (0,3) {$y$};
    \node at (6.5,1) {$A1$};
    \node at (2.5,2.5) {$A2$};
    \node[ above right] at (4.5, {sqrt(5}) {$y = y(x)$};
    
    \end{tikzpicture}
	    \end{center}
    \begin{enumerate}
        \item $\brak{6,21}$
        \item $\brak{8,9}$
        \item $\brak{10,-4}$
        \item $\brak{12,-15}$
    \end{enumerate}


    \item The equation of the sides $AB, BC$ and $CA$ of a triangle $ABC$ are $2x+y=0 , x+py=39 $ and $x-y=3$ respectively and $\vec{P}\brak{2,3}$ is its circumcentre. Then which of the following is NOT true:
    \hfill(July 2022)
    \begin{enumerate}
        \item $(AC)^2 =9p$
        \item $(AC)^2 + p^2 = 136$
        \item $32 < area(\Delta ABC) <36$
        \item $34 < area(\Delta ABC) <38$
    \end{enumerate}
    
    
    \item A Circle $C_{1}$ passes through the origin $\vec{O}$ and has diameter $4$ on the positive $x-axis.$ The line $y=2x$ gives a chord $OA$ of a circle $C_{1}.$ Let $C_{2}$ be the circle with $OA$ as a diameter. If the tangent to $C_{2}$  at the point $\vec{A}$ meets the $x-axis$ at $\vec{P}$ and $y-axis$ at $\vec{Q}$, then $QA : AP$ is equal to :
    \hfill(July 2022)
    \begin{enumerate}
        \item $1:4$
        \item $1:5$
        \item $2:5$
        \item $1:3$
    \end{enumerate}


    \item If the length of the latus rectum of a parabola,whose focus is $\brak{a,a}$ and the tangent at its vertex is $x+y = a$, is $16$, then $|a|$ is equal to 
    \hfill(July 2022)
    \begin{enumerate}
        \item $2\sqrt{2}$
        \item $2\sqrt{3}$
        \item $4\sqrt{2}$
        \item $4$
    \end{enumerate}


    \item If the Length of the perpendicular drawn from the point $\vec{P} \myvec{a,4,2} , a > 0$ on the line $\frac{x+1}{2} = \frac{y-3}{3} = \frac{z-1}{-1}$ is $2\sqrt{6}$ units and $\vec{Q} \myvec{\alpha_{1},\alpha_{2},\alpha_{3}}$ is the image of the point $\vec{P}$ in this line, then $a + \sum_{i=1}^{3} \alpha_{i}$ is equal to :
    \hfill(July 2022)
    \begin{enumerate}
        \item $7$
        \item $8$
        \item $12$
        \item $14$
    \end{enumerate}

    \item If the line of intersection of the planes $ax+by=3$ and $ax+by+cz=0 , a>0$ makes an angle $30\degree$ with the plane $y-z+2=0$, then the direction cosines of the line are :
    \hfill(July 2022)
    \begin{enumerate}
        \item $\frac{1}{\sqrt{2}} , \frac{1}{\sqrt{2}}, 0$
        \item $\frac{1}{\sqrt{2}} , -\frac{1}{\sqrt{2}}, 0$
        \item $\frac{1}{\sqrt{5}} , -\frac{2}{\sqrt{5}}, 0$
        \item $\frac{1}{2} , -\frac{\sqrt{3}}{2}, 0$
    \end{enumerate}
  %\end{enumerate}


