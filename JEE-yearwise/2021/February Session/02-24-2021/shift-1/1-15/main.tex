\iffalse
  \title{Assignment}
  \author{ee24btech11030}
  \section{mcq-single}
\fi

%   \begin{enumerate}
\item The locus of the mid-point of the line segment joining the focus of the parabola $y^2 = 4ax$ to a moving point of the parabola, is another parabola whose directrix is: \\ \hfill{[FEB 2021]}
    \begin{multicols}{4}
    \begin{enumerate}
        \item x = a
        \item x = 0
        \item x = -a/2
        \item x = a/2
    \end{enumerate}
    \end{multicols}
    \item A scientific committee is to formed from 6 Indians and 8 foreigners, which includes at least 2 Indians and double the number of foreigners as Indians. Then the number of ways, the committee can be formed is: \\\hfill{[FEB 2021]}
    \begin{multicols}{4}
    \begin{enumerate}
        \item $560$
        \item $1050$
        \item $1625$
        \item $575$
    \end{enumerate}
    \end{multicols}
    \item The equation of the plane passing through the point (1,2,-3) and perpendicular to the planes $3x + y - 2z = 5$ and $2x - 5y - z = 7$, is: \\\hfill{[FEB 2021]}
    \begin{enumerate}
        \item $3x - 10y - 2z + 11 = 0$ 
        \item $6x - 5y - 2z - 2 = 0$
        \item $11x + y + 17z + 38 = 0$
        \item $6x - 5y + 2z + 10 = 0$
    \end{enumerate} 
    \item A man is walking on a straight line. The arithmetic mean of the reciprocals of the intercepts of this line on the coordinate axes is 1/4. Three stones A, B and C are placed at the points (1,1),(2,2), and (4,4) respectively. Then which of these stones is/are on the path of the man? \\\hfill{[FEB 2021]}
     \begin{multicols}{4}
    \begin{enumerate}
        \item B only 
        \item A only
        \item the three
        \item C only
    \end{enumerate} 
    \end{multicols}
    \item The statement among the following that is a tautology is: \\\hfill{[FEB 2021]}
    \begin{enumerate}
        \item $A \land \brak{A \lor B}$
        \item $B \rightarrow [A \land (A \rightarrow B)]$
        \item $A \lor \brak{A \land B}$
        \item $[A \land (A \rightarrow B) \rightarrow B]$
    \end{enumerate}
    \item Let f : R $\rightarrow$ R be defined as f(x) = $2x-1$ and g : R-\{1\} $\rightarrow$ R be defined as \\ g(x) = $\frac{(x-\frac{1}{2})}{(x-1)}$.\\Then the composition function f(g(x)) is: \\\hfill{[FEB 2021]}
    \begin{enumerate}
        \item Both one-one and onto
        \item onto but not one-one
        \item Neither one-one nor onto
        \item one-one but not onto
    \end{enumerate}
    \item If f : R $\rightarrow$ R is a function defined by f(x) = [x - 1] $\cos{(\frac{2x - 1}{2})\pi}$ , where [.] denotes the greatest integer function, then f is: \\\hfill{[FEB 2021]}
    \begin{enumerate}
        \item discontinuous only at x = 1
        \item discontinuous at all integral values of x except at x = 1
        \item continuous only at x = 1
        \item continuous for every real x
    \end{enumerate}
    \item The function f(x) = $\frac{\brak{4x^3 - 3x^2}}{6} - 2\sin{x} + \brak{2x -1}\cos{x}$ \\\hfill{[FEB 2021]}
    \begin{enumerate}
        \item increases in $[\frac{1}{2},\infty)$
        \item decreases $(-\infty,\frac{1}{2}]$
        \item increases in $(-\infty,\frac{1}{2}]$
        \item decreases  $[\frac{1}{2},\infty)$
    \end{enumerate}
    \item The distance of the point (1,1,9) from the point of intersection of the line $\frac{x - 3}{1} = \frac{y - 4}{2} = \frac{z - 5}{2}$ and the plane $x + y + z = 17$ is: \\\hfill{[FEB 2021]}
    \begin{enumerate}
        \item increases in $\sqrt{38}$
        \item decreases $19\sqrt{2}$
        \item increases in $2\sqrt{19}$
        \item decreases  $38$
    \end{enumerate}
    \item $\lim_{x \to 0} \frac{\int_{0}^{x^2} \sin(\sqrt{t}) \, dt}{x^3}$ is equal to: \\\hfill{[FEB 2021]}
    \begin{multicols}{4}
    \begin{enumerate}
        \item $\frac{2}{3}$
        \item 0
        \item $\frac{1}{15}$
        \item $\frac{3}{2}$
    \end{enumerate}
    \end{multicols}
    \item Two vertical poles are 150 m apart and the height of one is three times that of the other. If from the middle point of the line joining their feet, an observer finds the angles of elevation of their tops to be complementary, then the height of the shorter pole (in meters) is: \\\hfill{[FEB 2021]}
    \begin{multicols}{4}
    \begin{enumerate}
        \item 25
        \item $20\sqrt{3}$
        \item 30
        \item $25\sqrt{3}$
    \end{enumerate}
    \end{multicols}
    \item If the tangent to the curve $y = x^3$ at the point $\vec{P}\brak{t, t^3}$ meets the curve again at $\vec{Q}$, then the ordinate of the point which divides PQ internally in the ratio 1:2 is: \\\hfill{[FEB 2021]}
    \begin{multicols}{4}
    \begin{enumerate}
        \item $-2t^3$
        \item $-t^3$
        \item 0
        \item $2t^3$
    \end{enumerate}
    \end{multicols}
    \item The area (in sq.units) of the part of the circle $x^2 + y^2 = 36$, which is outside the parabola $y^2 = 9x$, is: \\\hfill{[FEB 2021]}
    \begin{multicols}{4}
    \begin{enumerate}
        \item $24\pi + 3\sqrt{3}$
        \item $12\pi + 3\sqrt{3}$
        \item $12\pi - 3\sqrt{3}$
        \item $24\pi - 3\sqrt{3}$
    \end{enumerate}
    \end{multicols}
    \item If $\int \frac{\cos x - \sin x}{\sqrt{8 - \sin 2x}} \, dx = a \frac{\sin^{-1}(\sin x + \cos x)}{b} + c$, where c is a constant of integration, then the ordered pair (a, b) is equal to: \\\hfill{[FEB 2021]}
    \begin{multicols}{4}
    \begin{enumerate}
        \item $(1,-3)$
        \item $(1,3)$
        \item $(-1,3)$
        \item $(3,1)$
    \end{enumerate}
    \end{multicols}
    \item The population P = P(t) at time 't' of a certain species follows the differential equation $\frac{dP}{dt} = 0.5P - 450$. If P(0) = 850, then the time at which population becomes zero is: \\\hfill{[FEB 2021]}
    \begin{multicols}{4}
    \begin{enumerate}
        \item $\frac{1}{2}\ln{18}$
        \item $2\ln{18}$
        \item $\ln{9}$
        \item $\ln{18}$
    \end{enumerate}
    \end{multicols}
%   \end{enumerate}

