\iffalse
    \title{2021}
    \author{EE24BTECH11001}
    \section{mcq-single}
\fi
\item 
		The number of seven-digit integers with the sum of the digits equal to 10 and formed by using the digits 1, 2 and 3 only is :
		\hfill{\brak{2021-Feb}}
	\begin{multicols}{4}
		\begin{enumerate}
			\item 77 
			\columnbreak
			\item 42
			\columnbreak
			\item 35
			\columnbreak
			\item 82
		\end{enumerate}
	\end{multicols}

	\item
		The maximum value of the term independent of $'t'$ in the expression of
		$\sbrak{\brak{tx^{\frac{1}{5}}+ \{ \frac{1 - x}{t}\}} ^{\frac{1}{10}}}^{10}$ where
		$x \in \brak{0, 1}$ is :
		\hfill{\brak{2021-Feb}}
		\begin{multicols}{4}
		\begin{enumerate}
			\item $\frac{10!}{\sqrt{3}\brak{5!}^{2}}$ \columnbreak
			\item $\frac{2 . 10!}{3\brak{5!}^2}$ \columnbreak
			\item $\frac{10!}{3\brak{5!}^2}$ \columnbreak
			\item $\frac{2 . 10!}{3\sqrt{3}\brak{5!}^2}$
		\end{enumerate}
	\end{multicols}


\item The value of 
	\begin{align}
		\sum_{n = 1}^{100} \int_{n - 1}^n e^{x - \sbrak{x}} \, dx
	\end{align} where $\sbrak{x}$ is the greatest integer $\le x$
		\hfill{\brak{2021-Feb}}
		\begin{enumerate}
			\begin{multicols}{2}
			\item $100\brak{e - 1}$ \columnbreak
			\item $100e$
			\end{multicols}
			\begin{multicols}{2}
			\item $100\brak{1 - e}$ \columnbreak
			\item $100\brak{1 + e}$
			\end{multicols}
		\end{enumerate}
		
	\item The rate of growth of bacteria in a culture is proportional to the number of bacteria present and the bacteria cout is $1000$ at initial time $t = 0$. The number of bacteria is increased by $20\%$ in 2 hours. If the population of bacteria is $2000$ after $\frac{k}{\log_e \brak{\frac{6}{5}}}$ hours, then $\brak{\frac{k}{\log_e 2}}^2 $is equal to :
		\hfill{\brak{2021-Feb}}
		\begin{enumerate}
			\begin{multicols}{4}
				\item 1 \columnbreak
				\item 2 \columnbreak
				\item 16 \columnbreak
				\item 8
			\end{multicols}
		\end{enumerate}

	\item If $\vec{a}$ and $\vec{b}$ are perpendicular, then 
	\begin{align*}
		\vec{a} \times \left(\vec{a} \times \left(\vec{a} \times \left(\vec{a} \times \vec{b}\right)\right)\right)
	\end{align*}
	is equal to 
		\hfill{\brak{2021-Feb}}
		\begin{enumerate}
			\item $\frac{1}{2}\norm{\vec{a}}^4 \vec{b}$ 
			\item $\vec{a} \times \vec{b}$ 
			\item $\norm{\vec{a}}^4 \vec{b}$  
			\item $\vec{0}$
		\end{enumerate}
	\item
		In an increasing geometric series, the sum of the second and the sixth terms is $\frac{25}{2}$
		and the product of the third and fifth term is 25. Then, the sum of $4^{th}, 6^{th}$ and $8^{th}$ terms
		is equal to :
		\hfill{\brak{2021-Feb}}
		\begin{multicols}{4}
		\begin{enumerate}
			\item 35 \columnbreak
			\item 30 \columnbreak
			\item 26 \columnbreak
			\item 32
		\end{enumerate}
	\end{multicols}
	\item Consider the three planes
		\begin{enumerate}
			\item[P1 :] $3x + 15y + 21z = 9,$
			\item[P2 :] $x - 3y - z = 5, $and 
			\item[P3 :] $2x + 10y + 14z = 5$
		\end{enumerate}
		Then, which of the following is true ?	
		\hfill{\brak{2021-Feb}}
		\begin{enumerate}
			\item P1 and P3 are parallel. 
			\item P2 and P3 are parallel. 
			\item P1 and P2 are parallel. 
			\item P1, P2 and P3 are parallel.
		\end{enumerate}
\item
	The sum of the infinite series
	\begin{align}
		1 + \frac{2}{3} + \frac{7}{3^2} + \frac{12}{3^3} + \frac{17}{3^4} + \frac{22}{3^5} + \dots
	\end{align} is equal to :
		\hfill{\brak{2021-Feb}}
	\begin{multicols}{4}
		\begin{enumerate}
			\item $\frac{9}{4}$ \columnbreak
			\item $\frac{15}{4}$ \columnbreak
			\item $\frac{13}{4}$ \columnbreak
			\item $\frac{11}{4}$
		\end{enumerate}
	\end{multicols}
\item The value of
		\begin{align}
			\mydet{
			\brak{a+1}\brak{a+2} & a+2 & 1 \\
			\brak{a+2}\brak{a+3} & a+3 & 1 \\
			\brak{a+3}\brak{a+4} & a+4 & 1
			}
		\end{align} is :
		\hfill{\brak{2021-Feb}}
	
		\begin{enumerate}
			\item -2 
			\item $\brak{a+1}\brak{a+2}\brak{a+3}$ 
			\item 0 
			\item $\brak{a+2}\brak{a+3}\brak{a+4}$
		\end{enumerate}
\item If
		\begin{align}
			\frac{\sin ^{-1}}{a} = \frac{\cos ^{-1}}{b} = \frac{\tan ^{-1}}{c};
			0 < x < 1
		\end{align}, then the value of $\cos \brak{\frac{\pi c}{a+b}}$ is :
		\hfill{\brak{2021-Feb}}
		
		\begin{enumerate}
			\item $\frac{1 - y^2}{2y}$ \\
			\item $\frac{1 - y^2}{1 + y^2}$\\
			\item $1 - y^2$ \\
			\item $\frac{1 - y^2}{y\sqrt{y}}$
		\end{enumerate}
\item Let $A$ be a symmetric matrix of order 2 with integer entries. If the sum of the diagonal
	elements of $A^2$ is 1, then the possible number of such matrices is:
		\hfill{\brak{2021-Feb}}
		
	\begin{multicols}{4}
		\begin{enumerate}
			\item 6 \columnbreak
			\item 1 \columnbreak
			\item 4 \columnbreak
			\item 12
		\end{enumerate}
	\end{multicols}
\item The intersection of the three lines x-y = 0, x + 2y = 3 and 2x + y = 6 is a :
		\hfill{\brak{2021-Feb}}
		\begin{enumerate}
			\item Equilateral triangle 
			\item Right angled triangle  
			\item Isosceles triangle 
			\item None of the above
		\end{enumerate}
		\begin{figure}
			\centering
			\begin{tikzpicture}[scale = 2]
    				\draw[black, thick, domain=1:2, samples=100] 
					plot ({\x}, {\x}); 
    				\draw[black, thick, domain=2:3, samples=100] 
					plot ({\x}, {6 - 2 *(\x)}); 
   				\draw[black, thick, domain=1:3, samples=100] 
					plot ({\x}, {1.5 - 0.5 *(\x)}); 
					
				\node [left] at (1,1.5) {$x-y = 0$};	
				\node [left] at (2,0) {$x+2y = 3$};	
				\node [left] at (4,1) {$2x+y = 6$};	
				\fill[black] (2, 2) circle (1pt) node[above] {\tiny $\brak{2, 2}$};
				\fill[black] (3, 0) circle (1pt) node[below right] {\tiny $\brak{3, 0}$};
				\fill[black] (1, 1) circle (1pt) node[below left] {\tiny $\brak{1, 1}$};		
			\end{tikzpicture}
		\end{figure}
	\item The maximum slope of the curve $y = \frac{1}{2}x^4 -5x^2 + 18x^2 -19x$ occurs at the point:
		\hfill{\brak{2021-Feb}}
	\begin{multicols}{4}
		\begin{enumerate}
			\item $\brak{2, 9}$ \columnbreak
			\item $\brak{2, 2}$ \columnbreak
			\item $\brak{3, \frac{21}{2}}$ \columnbreak
			\item $\brak{0, 0}$
		\end{enumerate}
	\end{multicols}
\item Let $f$ be any function defined on $\vec{R}$ and let it satisfy the condition: 
	\begin{align}
		\abs{f\brak{x} - f\brak{y}} \le \abs{\brak{x - y^2}}, \forall x, y \in \vec{R}
	\end{align} 
		If $f\brak{0} = 1$, then :
		\hfill{\brak{2021-Feb}}
		\begin{enumerate}
			\item $f\brak{x} < 0, \forall x \in \vec{R}$
			\item $f\brak{x}$ can take any vaule in $\vec{R}$
			\item $f\brak{x} = 0$
			\item $f\brak{x} > 0, \forall x \in \vec{R}$
		\end{enumerate}
\item The value of
		\begin{align*}
			\int_{\frac{-\pi}{2}} ^ {\frac{\pi}{2}} \frac{\cos ^2 x}{1 + 3^x} \, dx
		\end{align*} is :
		\hfill{\brak{2021-Feb}}
	\begin{multicols}{4}
		\begin{enumerate}
			\item $2\pi$ \columnbreak
			\item $4\pi$ \columnbreak
			\item $\frac{\pi}{2}$ \columnbreak
			\item $\frac{\pi}{4}$
		\end{enumerate}
	\end{multicols}
