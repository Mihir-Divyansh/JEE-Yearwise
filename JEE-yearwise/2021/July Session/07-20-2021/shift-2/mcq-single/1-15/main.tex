
  \iffalse
  \title{2021}
  \author{EE24BTECH11010}
  \section{mcq-single}
\fi
    \item For the natural numbers $m,n,$ if $(1-y)^m (1+y)^n = 1 + a_1y + a_2 y^2 + .... + a_{m+n} y^{m+n}$ and $a_1 = a_2 = 10 $, then the value of $(m+n)$ is equal to : \hfill[July 2021]
    \begin{enumerate}
    \begin{multicols}{2}
        \item 88
        \item 64
        \item 100
        \item 80
        \end{multicols}
    \end{enumerate}
    \item The value of $\tan\brak{2\tan^{-1}\brak{\frac{3}{5}}+ \sin^{-1}\brak{\frac{5}{13}}}$ is equal to : \hfill[July 2021]
    \begin{enumerate}
    \begin{multicols}{4}
        \item $\frac{-181}{69}$
        \item $\frac{220}{21}$
        \item $\frac{-291}{76}$
        \item $\frac{151}{63}$
    \end{multicols}
    \end{enumerate}
    \item  Let $r_1$ and $r_2$ be the radii of the largest and smallest circles, respectively, which pass through the point \brak{-4,1} and having their centres on the circumference of the circle $x^2 + y^2 + 2x + 4y - 4 = 0$. If $\frac{r_1}{r_2} = a + b\sqrt{2} $, then $a+b$ is equal to: \hfill[July 2021]
    \begin{enumerate}
    \begin{multicols}{2}
        \item 3
        \item 11
        \item 5
        \item 7  
    \end{multicols}
    \end{enumerate}
    \item Consider the following three statements: \\
    $(A)$ If 3 + 3 = 7 then 4 + 3 = 8. \\ 
    $(B)$ If 5 + 3 = 8 then earth is flat.\\
    $(C)$ If both $(A)$ and $(B)$ are true then 5 + 6 = 17.\\
    Then, which of the following statements is correct ? \hfill[July 2021]
    \begin{enumerate}
        \item (A) is false, but (B) and (C) are true 
        \item (A) and (C) are true while (B) is false
        \item (A) is true while (B) and (C) are false 
        \item (A) and (B) are false while (C) is true
    \end{enumerate}
    \item The lines $x = ay -1 = z 
- 2$ and $x = 3y -
 2 = bz -
 2, (ab \neq  
 0) $ are coplanar, if \\: 
   \hfill[July 2021]
 \begin{enumerate}
 \begin{multicols}{2}
     \item $b = 1 , a \in \mathbb{R} - \{ 0 \}$
     \item $a = 1 , b \in \mathbb{R} - \{ 0 \}$
     \item $a = 2, b = 2 $
     \item $ a =2 , b = 3$
 \end{multicols}
 \end{enumerate}
   \item  If $\sbrak{x}$ denotes the greatest integer less than or equal to $x$, then the value of the integral $\int_{-\frac{\pi}{2}}^{\frac{\pi}{2}} \sbrak{\sbrak{x} - \sin{x}}dx$ is equal to : \hfill[July 2021]
    \begin{enumerate}
    \begin{multicols}{2}
        \item $-\pi$
        \item $\pi$
        \item $0$
        \item 1
    \end{multicols}
    \end{enumerate}
    \item  If the real part of the complex number $\brak{1 - \cos \theta + 2i \sin\theta}^{-1}$
  is $\frac{1}{5}$ $ \theta \in \brak{0 , \pi}$, then the value of the integral $\int_{0}^{\theta} \sin xdx$ is equal to: \hfill[July 2021]
  \begin{enumerate}
 \begin{multicols}{2}
      \item 1
      \item 2
      \item  $-1$
      \item 0
 \end{multicols}
  \end{enumerate}
  \item Let $f : \mathbb{R} - \{ \frac{\alpha}{6} \} \to \mathbb{R}$ be defined by 
 $f(x) = \frac{5x + 3}{6x - \alpha}.$ Then the value of $\alpha$
 for which $\brak{f\circ f}\brak{x} = x$, for all $x = \mathbb{R} - \{ \frac{\alpha}{6} \}$
, is : \hfill[July 2021]
\begin{enumerate}
\begin{multicols}{2}
    \item No such $\alpha$ exists
    \item 5 
    \item 8
    \item 6
\end{multicols}
\end{enumerate}
    \item If $f : \mathbb{R} \to \mathbb{R}$ is given by $f(x) = x + 1$
, then the value of  \[\lim_{n\to\infty} \frac{1}{n} \sbrak{f(0) + f\brak{\frac{5}{n}} + f \brak{\frac{10}{n}} + ..... + f\brak{\frac{5(n - 1)}{n}}},\] \hfill[July 2021]
\begin{enumerate}
\begin{multicols}{4}
    \item $\frac{3}{2}$
    \item $\frac{5}{2}$
    \item $\frac{1}{2}$
    \item $\frac{7}{2}$
    \end{multicols}
\end{enumerate}
    \item Let A, B and C be three events such that the probability that exactly one of A and B occurs is (1, 
 -$k$), the probability that exactly one of B and C occurs is (1, -2$k$), the probability that exactly one of C and A occurs is (1, -k) and the probability of all A, B and C occur simultaneously is $k^2$, where $0 < k < 1$. Then the probability that at least one of A, B and C occur is : \hfill[July 2021]
 \begin{enumerate}
 \begin{multicols}{2}
     \item greater than $\frac{1}{8}$ but less than $\frac{1}{4}$
     \item greater than $\frac{1}{2}$
     \item greater than $\frac{1}{4}$ but less than $\frac{1}{2}$
     \item exactly equal to $\frac{1}{2}$
 \end{multicols}
 \end{enumerate}
    \item The sum of all the local minimum values of the twice differentiable function $f : \mathbb{R}\to
 \mathbb{R}$ defined by  $f(x) = x^3 - 3x^2 - \frac{3f^{\prime \prime}(2)}{2}x + f^{\prime\prime}(1)$ is : \hfill[July 2021]
 \begin{enumerate}
 \begin{multicols}{2}
     \item $-22$
     \item 5
     \item $-27$
     \item 0
 \end{multicols}
 \end{enumerate}
  \item Let in a right angled triangle, the smallest angle be $\theta$. If a triangle formed by taking the reciprocal of its sides is also a right angled, then $\sin \theta$ is equal to: \hfill[July 2021]
 \begin{enumerate}
  \begin{multicols}{4}
      \item $\frac{\sqrt{5}+1}{4}$
      \item $\frac{\sqrt{5}-1}{2}$
      \item $\frac{\sqrt{2}-1}{2}$
      \item $\frac{\sqrt{5}-1}{4}$
  \end{multicols}
 \end{enumerate}
  \item Let $y = y(x)$ satisfies the equation $\frac{dy}{dx} - \abs{A} = 0$ for all $x > 0$, where $A$ = $\myvec{y & \sin x & 1 \\ 0 & -1 & 1 \\ 2 & 0 & \frac{1}{x}}$. If $y(\pi) = \pi + 2$, then the value of $y\brak{\frac{\pi}{2}}$ is: \hfill[July 2021]
 \begin{enumerate}
 \begin{multicols}{4}
     \item $\frac{\pi}{2} + \frac{4}{\pi}$
     \item $\frac{\pi}{2} - \frac{1}{\pi} $
     \item $\frac{3\pi}{2} - \frac{1}{\pi}$
     \item $\frac{\pi}{2} - \frac{4}{\pi} $
 \end{multicols}
 \end{enumerate}
  \item Consider the line L given by the equation $\frac{x-3}{2} = \frac{y-1}{1} = \frac{z-2}{1}$. Let Q be the mirror image of the point $\brak{2,3,-1}$ with respect to L. Let a plane P be such that it passes through Q, and the line L is perpendicular to P. Then which of the following points is on the plane P? \hfill[July 2021]
 \begin{enumerate}
     \begin{multicols}{2}
     \item $\brak{-1,1,2}$
     \item $\brak{1,1,1}$
     \item $\brak{1,1,2}$
     \item $\brak{1,2,2}$
     \end{multicols}
 \end{enumerate}
 \item If the mean and variance of six observations $7, 10, 11, 15, a, b $ are 10 and $\frac{20}{3}$
 , respectively, then the value of $\abs{a -
 b}$ is equal to : \hfill[July 2021]
 \begin{enumerate}
 \begin{multicols}{2}  
     \item 9
     \item 11
     \item 7
     \item 1
 \end{multicols}
 \end{enumerate}
=======
\iffalse
	\title{2021}
	\author{AI24BTECH11003}
	\section{mcq-single}
\fi

%1
    \item If $\alpha+\beta+\gamma=2\pi$, then the system of equations\\
    $x+\brak{\cos\gamma}y+\brak{\cos\beta}z=0$\\
    $\brak{\cos\gamma}x+y+\brak{\cos\alpha}z=0$\\
    $\brak{\cos\beta}x+\brak{\cos\alpha}y+z=0$\\
    has:
    
    \hfill[Aug 2021]

        \begin{multicols}{2}
            \begin{enumerate}
                \item no solution
                \item infinitely many solutions
                \item exactly two solutions
                \item a unique solution
            \end{enumerate}
        \end{multicols}

%2
    \item Let $\overrightarrow{a},\overrightarrow{b},\overrightarrow{c}$ be three vectors mutually perpendicular to each other and have the same magnitude. If a vector $\overrightarrow{r}$ satisfies 
    $\overrightarrow{a}\times\left\{ \brak{\overrightarrow{r}-\overrightarrow{b}}\times\overrightarrow{a} \right\} + \overrightarrow{b}\times\left\{ \brak{\overrightarrow{r}-\overrightarrow{c}}\times\overrightarrow{b} \right\} + \overrightarrow{c}\times\left\{ \brak{\overrightarrow{r}-\overrightarrow{a}}\times\overrightarrow{c} \right\} = \overrightarrow{0}$, then $\overrightarrow{r}$ is equal to:
    
    \hfill[Aug 2021]

		\begin{multicols}{4}
			\begin{enumerate}
				\item $\frac{1}{3}\brak{\overrightarrow{a}+\overrightarrow{b}+\overrightarrow{c}}$
				\item $\frac{1}{3}\brak{2\overrightarrow{a}+\overrightarrow{b}+\overrightarrow{c}}$
				\item $\frac{1}{2}\brak{\overrightarrow{a}+\overrightarrow{b}+\overrightarrow{c}}$
				\item $\frac{1}{2}\brak{\overrightarrow{a}+\overrightarrow{b}+2\overrightarrow{c}}$
			\end{enumerate}
		\end{multicols}

%3
    \item The domain of the function $f\brak{x}=\arcsin\brak{\frac{3x^2+x-1}{\brak{x-1}^2}}+\arccos\brak{\frac{x-1}{x+1}}$ is:
    
    \hfill[Aug 2021]

        \begin{multicols}{4}
            \begin{enumerate}
                \item $\sbrak{0, \frac{1}{4}}$
                \item $\sbrak{-2,0}\cup\sbrak{\frac{1}{4}, \frac{1}{2}}$
                \item $\sbrak{\frac{1}{4}, \frac{1}{2}}\cup\{0\}$
                \item $\sbrak{0, \frac{1}{2}}$
            \end{enumerate}
        \end{multicols}

%4
    \item Let $S=\{1,2,3,4,5,6\}$. Then the probability that a randomly chosen onto function $g$ from $S$ to $S$ satisfies $g\brak{3}=2g\brak{1}$ is:
    
    \hfill[Aug 2021]

		\begin{multicols}{4}
			\begin{enumerate}
				\item $\frac{1}{10}$
				\item $\frac{1}{15}$
				\item $\frac{1}{5}$
				\item $\frac{1}{30}$
			\end{enumerate}
		\end{multicols}

%5
    \item Let $f:N\to N$ be a function such that $f\brak{m+n}=f\brak{m}+f\brak{n}$ for every $m,n\in N$. If $f\brak{6}=18$, then $f\brak{2}\cdot f\brak{3}$ is equal to:
    
    \hfill[Aug 2021]

		\begin{multicols}{4}
			\begin{enumerate}
				\item $6$
				\item $54$
				\item $18$
				\item $36$
			\end{enumerate}
		\end{multicols}
  
%6
    \item The distance of the point $\brak{-1, 2, -2}$ from the line of intersection of the planes $2x+3y+2z=0$ and $x-2y+z=0$ is:
    
    \hfill[Aug 2021]

        \begin{multicols}{4}
            \begin{enumerate}
                \item $\frac{1}{\sqrt{2}}$
                \item $\frac{5}{2}$
                \item $\frac{\sqrt{42}}{2}$
                \item $\frac{\sqrt{34}}{2}$
            \end{enumerate}
        \end{multicols}

%7
    \item Negation of the statement $\brak{p\lor q}\implies\brak{q\lor r}$ is:
    
    \hfill[Aug 2021]

        \begin{multicols}{4}
            \begin{enumerate}
                \item $p\land \sim q\land \sim r$
                \item $\sim p\land q\land \sim r$
                \item $\sim p\land q\land r$
                \item $p\land q\land r$
            \end{enumerate}
        \end{multicols}
		
%8
    \item If $\alpha=\underset{x\to\frac{\pi}{4}}{lim}\frac{\tan^3 x-\tan x}{\cos\brak{x+\frac{\pi}{4}}}$ and $\beta=\underset{x\to 0}{lim}\brak{\cos x}^{\cot x}$ are the roots of the equation $ax^2+bx-4=0$, then the ordered pair $\brak{a,b}$ is:
    
    \hfill[Aug 2021]

        \begin{multicols}{4}
            \begin{enumerate}
                \item $\brak{1,-3}$
                \item $\brak{-1,3}$
                \item $\brak{-1,-3}$
                \item $\brak{1,3}$
            \end{enumerate}
        \end{multicols}

%9
    \item The locus of midpoints of the line segments joining $\brak{-3,-5}$ and the points on the ellipse $\frac{x^2}{4}+\frac{y^2}{9}=1$ is:
    
    \hfill[Aug 2021]

        \begin{multicols}{2}
            \begin{enumerate}
                \item $9x^2+4y^2+18x+8y+145=0$
                \item $36x^2+16y^2+90x+56y+145=0$
                \item $36x^2+16y^2+108x+80y+145=0$
                \item $36x^2+16y^2+72x+32y+145=0$
            \end{enumerate}
        \end{multicols}

%10
    \item If $\frac{dy}{dx}=\frac{2^xy+2^y\cdot 2^x}{2^x+2^{x+y}\log_e2}, y\brak{0}=0$, then for $y=1$, the value of x lies in the interval:
    
    \hfill[Aug 2021]

        \begin{multicols}{4}
            \begin{enumerate}
                \item $\brak{1,2}$
                \item $\left( \frac{1}{2},1\right]$
                \item  $\brak{2,3}$
                \item $\left( 0,\frac{1}{2}\right]$
            \end{enumerate}
        \end{multicols}
        
%11
    \item An angle of intersection of the curves, $\frac{x^2}{a^2}+\frac{y^2}{b^2}=1$ and $x^2+y^2=ab,a>b$, is:
    
    \hfill[Aug 2021]

        \begin{multicols}{4}
            \begin{enumerate}
                \item $\arctan\brak{\frac{a+b}{\sqrt{ab}}}$
                \item $\arctan\brak{\frac{a-b}{2\sqrt{ab}}}$
                \item $\arctan\brak{\frac{a-b}{\sqrt{ab}}}$
                \item $\arctan\brak{2\sqrt{ab}}$
            \end{enumerate}
        \end{multicols}

%12
    \item If $y\frac{dy}{dx}=x\sbrak{\frac{y^2}{x^2}+\frac{\phi\brak{\frac{y^2}{x^2}}}{\phi'\brak{\frac{y^2}{x^2}}}},x>0,\phi>0$, and $y\brak{1}=-1$, then $\phi\frac{y^2}{x^4}$ is equal to:
    
    \hfill[Aug 2021]

        \begin{multicols}{4}
            \begin{enumerate}
                \item $4\phi\brak{2}$
                \item $4\phi\brak{1}$
                \item $2\phi\brak{1}$
                \item $\phi\brak{1}$
            \end{enumerate}
        \end{multicols}
        
%13
    \item The sum of the roots of the equation $x+1-2\log_2\brak{2+2^x}+2\log_4\brak{10-2^{-x}}=0$, is:
    
    \hfill[Aug 2021]

        \begin{multicols}{4}
            \begin{enumerate}
                \item $\log_2 14$
                \item $\log_2 11$
                \item $\log_2 12$
                \item $\log_2 13$
            \end{enumerate}
        \end{multicols}

%14
    \item If z is a complex number such that $\frac{z-i}{z-1}$ is purely imaginary, then the minimum value of $\abs{z-\brak{3+3i}}$ is:
    
    \hfill[Aug 2021]

        \begin{multicols}{4}
            \begin{enumerate}
                \item $2\sqrt{2}-1$
                \item $3\sqrt{2}$
                \item $6\sqrt{2}$
                \item $2\sqrt{2}$
            \end{enumerate}
        \end{multicols}

%15
    \item Let $a_1,a_2,a_3,\cdots$ be an A.P. If $\frac{a_1+a_2+\cdots+a_{10}}{a_1+a_2+\cdots+a_p}=\frac{100}{p^2},p\neq10$, then $\frac{a_{11}}{a_{10}}$ is equal to:
    
    \hfill[Aug 2021]
    
        \begin{multicols}{4}
            \begin{enumerate}
                \item $\frac{19}{21}$
                \item $\frac{100}{121}$
                \item $\frac{21}{19}$
                \item $\frac{121}{100}$
            \end{enumerate}
        \end{multicols}
