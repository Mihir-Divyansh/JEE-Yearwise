
\iffalse
  \title{2021}
  \author{EE24BTECH11032}
  \section{mcq-single}
\fi
%   \begin{enumerate}
    \item A spherical gas balloon of radius $16$ meter subtends an angle $60\degree$ at the eye of the observer A while the angle of elevation of its center from the eye of A is $75\degree$. Then the height (in meter) of the top most point of the balloon from the level of the observers eye is: \hfill{\sbrak{Jul 2021}}
		\begin{enumerate}
			\item $8\brak{2+2\sqrt{3}+\sqrt{2}}$
			\item $8\brak{\sqrt{6}+\sqrt{2}+2}$
			\item $8\brak{\sqrt{2}+2+\sqrt{3}}$
			\item $8\brak{\sqrt{6}-\sqrt{2}+2}$
		\end{enumerate}
	\item Let $f\brak{x}=3\sin^4{x}+10\sin^3{x}+6\sin^2{x}-3$, $x \in \sbrak{\frac{-\pi}{6}, \frac{\pi}{2}}$. Then f is: \hfill{\sbrak{Jul 2021}}
		\begin{enumerate}
			\item increasing in $\brak{\frac{-\pi}{6},\frac{\pi}{2}}$
			\item decreasing in $\brak{0,\frac{\pi}{2}}$
			\item increasing in $\brak{\frac{-\pi}{6},0}$
			\item decreasing in $\brak{\frac{-\pi}{6},0}$
		\end{enumerate}
    \item Let $S_n$ be the sum of first n terms of an arithmetic progression. If $S_{3n}=3S_{2n}$, then the value of $\frac{S_{4n}}{S_{2n}}$ is: \hfill{\sbrak{Jul 2021}}
    \begin{enumerate}
        \item $6$
        \item $4$
        \item $2$
        \item $8$
    \end{enumerate}
    \item The locus of the centroid of the triangle formed by any point P on the hyperbola $16x^2-9y^2+32x+36y-164=0$, and its foci is: \hfill{\sbrak{Jul 2021}}
    \begin{enumerate}
        \item $16x^2-9y^2+32x+36y-36=0$
        \item $9x^2-16y^2+36x+32y-144=0$
        \item $16x^2-9y^2+32x+36y-144=0$
        \item $9x^2-16y^2+36x+32y-36=0$
    \end{enumerate}
\item Let the vectors $\brak{2+a+b}\hat{i}+\brak{a+2b+c}\hat{j}-\brak{b+c}\hat{k},\brak{1+b}\hat{i}+2b\hat{j}-b\hat{k}$ and $\brak{2+b}\hat{i}+2b\hat{j}+\brak{1-b}\hat{k}$ a,b,c $\in \vec{R}$ be co-planar. Then which of the following is true? \hfill{\sbrak{Jul 2021}}
\begin{enumerate}
    \item $2b=a+c$
    \item $3c=a+b$
    \item $a=b+2c$
    \item $2a=b+c$
\end{enumerate}
\item Let f:$\vec{R}\rightarrow\vec{R}$ be defined as \\
$f\brak{x}=\left\{ \begin{array}{ll} \frac{\lambda\abs{x^2-5x+6}}{\mu\brak{5x-x^2-6}}, \quad x < 2 \\ e^{\frac{\tan\brak{x-2}}{x-\myceil{x}}}, \quad x>2 \\ \mu, \quad x=2 \end{array} \right. $\\
where $\myceil{x}$ is the greatest integer less than or equal to x. If f is continuous at $x=2$, Then $\lambda + \mu$ equal to: \hfill{\sbrak{Jul 2021}}
\begin{enumerate}
    \item e\brak{-e+1}
    \item e\brak{e-2}
    \item 1
    \item 2e-1
\end{enumerate}
\item The value of the definite integral\\
$\int_{\frac{\pi}{24}}^{\frac{5\pi}{24}} \frac{dx}{1+\sqrt[3]{\tan2x}}$ is: \hfill{\sbrak{Jul 2021}}
\begin{enumerate}
    \item $\frac{\pi}{3}$
    \item $\frac{\pi}{6}$
    \item $\frac{\pi}{12}$
    \item $\frac{\pi}{18}$
\end{enumerate}
\item If b is very small as compared to the value of a, so that the cube and other higher powers of $\frac{b}{a}$ can be neglected in the identity\\
$\frac{1}{a-b}+\frac{1}{a-2b}+\frac{1}{a-3b}....+\frac{1}{a-nb}=\alpha n+\beta n^2+\gamma n^3$, then the value of $\gamma$ is: \hfill{\sbrak{Jul 2021}}
\begin{enumerate}
    \item $\frac{a^2+b}{3a^3}$
    \item $\frac{a+b}{3a^2}$
    \item $\frac{b^2}{3a^3}$
    \item $\frac{a+b^2}{3a^3}$
\end{enumerate}
\item Let $y=y\brak{x}$ be the solution of the differential equation $\frac{dy}{dx}=1+xe^{y-x}, -\sqrt{2}<x<\sqrt{2},y\brak{0}=0$ then, the minimum value of $y\brak{x}, x \in \brak{-\sqrt{2}, \sqrt{2}}$ is equal to: \hfill{\sbrak{Jul 2021}}
\begin{enumerate}
    \item $\brak{2-\sqrt{3}}-log_e2$
    \item $\brak{2+\sqrt{3}}+log_e2$
    \item $\brak{1+\sqrt{3}}-log_e\brak{\sqrt{3}-1}$
    \item $\brak{1-\sqrt{3}}-log_e\brak{\sqrt{3}-1}$
\end{enumerate}
\item The Boolean expression\\
$\brak{p \implies q} \land \brak{q \implies \sim p} $ is equivalent to: \hfill{\sbrak{Jul 2021}}
\begin{enumerate}
    \item $\sim$q
    \item q
    \item p
    \item $\sim$p
\end{enumerate}
\item The area (in sq. units) of the region, given by the set $\cbrak{\brak{x,y} \in \vec{RxR}|x\geq0,2x^2\leq y\leq 4-2x}$ is \hfill{\sbrak{Jul 2021}}
\begin{enumerate}
    \item $\frac{8}{3}$
    \item $\frac{17}{3}$
    \item $\frac{13}{3}$
    \item $\frac{7}{3}$
\end{enumerate}
\item The sum of all values of x in $\sbrak{0,2\pi}$, for which $\sin x+\sin2x+\sin3x+\sin4x=0$, is equal to: \hfill{\sbrak{Jul 2021}}
\begin{enumerate}
    \item $8\pi$
    \item $11\pi$
    \item $12\pi$
    \item $9\pi$
\end{enumerate}
\item Let g:$\vec{N} \to \vec{N}$ be defined as \\
$g\brak{3n+1}=3n+2$\\
$g\brak{3n+2}=3n+3$\\
$g\brak{3n+3}=3n+1$, for all $n\geq0$.\\
Then whcih of the following statements is true? \hfill{\sbrak{Jul 2021}}
\begin{enumerate}
    \item There exist an onto function f:$\vec{N} \to \vec{N}$ such that fog=f
    \item There exist a one-one function f:$\vec{N}\to \vec{N}$ such that fog=f
    \item gogog=g
    \item There exists a function f:$\vec{N} \to \vec{N}$ such that gof=f
\end{enumerate}
\item Let f:$\lsbrak{0},\rbrak{\infty}\to \lsbrak{0},\rbrak{\infty}$ be defined as $f\brak{x}=\int_0^x \myceil{y}dy$ where $\myceil{x}$ is the greatest integer less then or equal to x. Which of the following is true? \hfill{\sbrak{Jul 2021}}
\begin{enumerate}
    \item f is continuous at every point in $\lsbrak{0},\rbrak{\infty}$ and differentiable except at the integer points.
    \item f is both continuous and differentiable except at integer points in $\lsbrak{0},\rbrak{\infty}$.
    \item f is continuous everywhere except at the integer points in $\lsbrak{0},\rbrak{\infty}$.
    \item f is differentiable at every point in $\lsbrak{0},\rbrak{\infty}$.
\end{enumerate}
\item The values a and b, for which the system of equations\\ 
$2x+3y+6z=8$\\
$x+2y+az=5$\\
$3x+5y+9z=b$\\
has no solution, are: \hfill{\sbrak{Jul 2021}}
\begin{enumerate}
    \item $a=3,b\neq13$
    \item $a\neq3,b\neq3$
    \item $a\neq3,b=3$
    \item $a=3,b=13$
\end{enumerate}
