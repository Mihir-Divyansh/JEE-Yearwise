\iffalse
\title{2021}
\author{EE24Btech11024}
\section{integer}
\fi


\item Let $n\in \mathbb{N}$ and $\sbrak{x}$ denote the greatest integer less than or equal to $x$. If the sum of $\brak{n+1}$ terms $\comb{n}{0}$, $3\cdot\comb{n}{1}$, $5\cdot\comb{n}{2}$, $7\cdot\comb{n}{3} \dots$ is equal to $2^{100}\cdot 101$, then $2\sbrak{\frac{n-1}{2}}$ is equal to \rule{1cm}{0.15mm}.

\hfill{\brak{\text{Jul 2021}}}

\item Consider the function $f\brak{x} = \begin{cases} \frac{P\brak{x}}{\sin\brak{x-2}} & x\neq 2, \\7 & x=2 .\end{cases}$ where $P\brak{x}$ is a polynomial such that ${P^\prime}^\prime\brak{x}$ is always a constant and $P\brak{3}=9$. If $f\brak{x}$ is continuous at $x=2$, then $P\brak{5}$ is equal to \rule{1cm}{0.15mm}.

\hfill{\brak{\text{Jul 2021}}}

\item The equation of a circle is $Re\brak{z^2}+2\brak{Im\brak{z}}^2+2Re\brak{z}=0$, where $z=x+iy$. A line which passes through the centre of the given circle and the vertex of parabola, $x^2-6x-y+13=0$, has y-intercept equal to \rule{1cm}{0.15mm}.

\hfill{\brak{\text{Jul 2021}}}

\item If a rectangle is inscribed in an equilateral triangle of side length $2\sqrt{2}$ as shown in the figure, then the square of the largest area of such a rectangle is \rule{1cm}{0.15mm}.
\\\begin{center}
   \scalebox{0.75}{\begin{tikzpicture}
    \draw (0,0) -- (5,0) -- (2.5,4.33) -- cycle;
    \draw (1,0) -- (1,1.67) -- (4,1.67) -- (4,0) -- cycle;
\end{tikzpicture}}
\end{center}

\hfill{\brak{\text{Jul 2021}}}

\item If $\brak{\vec{a}+3\vec{b}}$ is perpendicular to $\brak{7\vec{a}-5\vec{b}}$ and $\brak{\vec{a}-4\vec{b}}$ is perpendicular to $\brak{7\vec{a}-2\vec{b}}$, then the angle between $\vec{a}$ and $\vec{b}$ \brak{\text{in degrees}} is \rule{1cm}{0.15mm}.

\hfill{\brak{\text{Jul 2021}}}

\item Let a curve $y=f\brak{x}$ pass through the point $\brak{2,\brak{\log_e{2}}^2}$ and have slope $\frac{2y}{x\log_e{x}}$ for all positive real values of $x$. Then the value of $f\brak{e}$ is equal to \rule{1cm}{0.15mm}.

\hfill{\brak{\text{Jul 2021}}}

\item If $a+b+c=1$, $ab+bc+ca=2$ and $abc=3$, then the value of $a^4+b^4+c^4$ is equal to \rule{1cm}{0.15mm}.

\hfill{\brak{\text{Jul 2021}}}

\item A fair coin is tossed $n$-times such that the probability of getting at least one head is at least $0.9$. Then the minimum value of $n$ is \rule{1cm}{0.15mm}. 

\hfill{\brak{\text{Jul 2021}}}

\item If the co-efficient of $x^7$ and $x^8$ in the expansion of $\brak{2+\frac{x}{3}}^n$ are equal, then the value of n is equal to \rule{1cm}{0.15mm}.

\hfill{\brak{\text{Jul 2021}}}

\item If the lines $\frac{x-k}{1}=\frac{y-2}{2}=\frac{z-3}{3}$ and $\frac{x+1}{3}=\frac{y+2}{2}=\frac{z+3}{1}$ are co-planar then, the value of $k$ is \rule{1cm}{0.15mm}.

\hfill{\brak{\text{Jul 2021}}}

