\iffalse
\title{27-07-2021}
\author{AI24BTECH11032}
\section{mcq-single}
\fi

\item Let $f : \mathbf{R} \to \mathbf{R}$ be a function such that $f\brak{2}=4$ and $f'\brak{2}=1.$ Then the value of $\lim_{x \to 2} \frac{x^2 f\brak{2} - 4 f\brak{x}}{x - 2}$ is equal to :\hfill{July-2021}
\begin{multicols}{4}
    \begin{enumerate}
        \item $4$
        \item $8$
        \item $16$
        \item $12$
    \end{enumerate}
\end{multicols}
\item Let P and Q be two distinct points on a circle which has centre at $C\brak{2,3}$ and which passes through origin O .If OC is perpendicular to both the line segment CP and CQ then the set $\cbrak{P,Q}$ is equal to\hfill{july-2021}

    \begin{enumerate}
        \item $\cbrak{\brak{4,0},{\brak{0,6}}}$
        \item $\cbrak{\brak{2+2\sqrt{2},3-\sqrt{5}},{\brak{2-2\sqrt{2},3+\sqrt{5}}}}$
        \item $\cbrak{\brak{2+2\sqrt{2},3+\sqrt{5}},{\brak{2-2\sqrt{2},3-\sqrt{5}}}}$
        \item $\cbrak{\brak{-1,5},{\brak{5,1}}}$
    \end{enumerate}
\item let $\alpha$ , $\beta$ be two roots of the equation $x^{2} +\brak{20}^{\frac{1}{4}}x+\brak{5}^{\frac{1}{2}} = 0 .$ Then $\alpha^{8}+\beta^{8}$ is equal to\hfill{july-2021}
\begin{multicols}{4}
    \begin{enumerate}
        \item $10$
        \item $100$
        \item $50$
        \item $160$
    \end{enumerate}
\end{multicols}
\item The probability that a randomly selected $2-\text{digit}$ number belongs to the set $\cbrak{n \in \mathbf{N} : \brak{2^{n} - 2} \text{is a multiple of } 3}$ is equal to \hfill{July-2021}
\begin{multicols}{4}
    \begin{enumerate}
        \item $\frac{1}{6}$
        \item $\frac{2}{3}$
        \item $\frac{1}{2}$
        \item $\frac{1}{3}$
    \end{enumerate}
\end{multicols}
\item Let 
\begin{align*}
    A = \cbrak{\brak{x,y} \in \mathbf{R}\times\mathbf{R}\mid 2x^{2}+2y^{2}-2x-2y=1},\\
    B = \cbrak{\brak{x,y} \in \mathbf{R}\times\mathbf{R}\mid 4x^{2}+4y^{2}-16y+7=0} and\\
    C = \cbrak{\brak{x,y} \in \mathbf{R}\times\mathbf{R}\mid x^{2}+y^{2}-4x-2y+5\leq r^{2}},
\end{align*}
Then the minimum value of $\abs{r}$ such that $A \cup B \subseteq C$ is equal  to \hfill{July-2021}
\begin{multicols}{2}
    \begin{enumerate}
        \item $\frac{3+\sqrt{10}}{2}$
        \item $\frac{2+\sqrt{10}}{2}$
        \item $\frac{3+2\sqrt{10}}{2}$
        \item $1+\sqrt{5}$
    \end{enumerate}
\end{multicols}

