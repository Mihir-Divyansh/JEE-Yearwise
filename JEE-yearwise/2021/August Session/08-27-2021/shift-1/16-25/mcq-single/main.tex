\iffalse
  \title{Assignment}
  \author{ee24btech11030}
  \section{mcq-single}
\fi

%   \begin{enumerate}
\item If $\alpha$, $\beta$ are the distinct roots of $x^2+ bx + c = 0$ , then $\lim_{x \to \beta}\frac{e^{2(x^2 + bx + c)} - 1 - 2(x^2 + bx + c)}{{(x - \beta)}^2}$ is equal to:  \\ \hfill{[AUG 2021]}
    \begin{multicols}{4}
    \begin{enumerate}
        \item $b^2 + 4c$
        \item $2(b^2 + 4c)$
        \item $2(b^2 - 4c)$
        \item $b^2 - 4c$
    \end{enumerate}
    \end{multicols}
    \item When a certain biased die is rolled, a particular face occurs with probability $\frac{1}{6} - x$ and its opposite face occurs with probability $\frac{1}{6} + x$. All other faces occur with probability $\frac{1}{6}$ Note that opposite faces sum to 7 in any die. If 0 $<$ x $<$ $\frac{1}{6}$ and the probability of obtaining total sum = 7, when such a die is rolled twice, is $\frac{13}{96}$, then the value of x is:  \\\hfill{[AUG 2021]}
    \begin{multicols}{4}
    \begin{enumerate}
        \item $\frac{1}{16}$
        \item $\frac{1}{8}$
        \item $\frac{1}{9}$
        \item $\frac{1}{12}$
    \end{enumerate} 
    \end{multicols}
    \item If $x^2+ 9y^2 - 4x + 3 = 0$, x, y $\in$ R , then x and y respectively lie in the intervals:  \\\hfill{[AUG 2021]}
    \begin{multicols}{2}
    \begin{enumerate}
        \item $\left[\frac{-1}{3},\frac{1}{3}\right]$ and $\left[\frac{-1}{3},\frac{1}{3}\right]$\\
        \item $\left[\frac{-1}{3},\frac{1}{3}\right]$ and $\left[1,3\right]$
        \item $\left[1,3\right]$ and $\left[1,3\right]$\\
        \item $\left[1,3\right]$ and $\left[\frac{-1}{3},\frac{1}{3}\right]$
    \end{enumerate}
    \end{multicols}
    \item $\int_{6}^{16} \frac{\ln x^2}{\ln{x^2} + \ln{(x^2 - 44x + 484)}} \, dx$ is equal to: \\\hfill{[AUG 2021]}
    \begin{multicols}{4}
    \begin{enumerate}
        \item 6
        \item 8
        \item 5
        \item 10
    \end{enumerate} 
    \end{multicols}
    \item A wire of length 20 m is to be cut into two pieces.One of the pieces is to be made into a square and the other into a regular hexagon. Then the length of the side (in meters) of the hexagon, so that the combined area of the square and the hexagon is minimum, is:  \\\hfill{[AUG 2021]}
    \begin{multicols}{4}
    \begin{enumerate}
        \item $\frac{5}{2 + \sqrt{3}}$
        \item $\frac{10}{2 + 3\sqrt{3}}$
        \item $\frac{5}{3 + \sqrt{3}}$
        \item $\frac{10}{3 + 2\sqrt{3}}$
    \end{enumerate} 
    \end{multicols}
%   \end{enumerate}
