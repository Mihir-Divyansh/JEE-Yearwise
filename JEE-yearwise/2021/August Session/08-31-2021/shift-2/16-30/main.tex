
\iffalse
  \title{August:2021}
  \author{AI24BTECH11016}
  \section{mcq-single}
\fi
	\item
	Let $\vec{A}$ be the set of all points $\myvec{\alpha , \beta}$ such that the area of triangle formed by the points $\myvec{5 , 6}$ $\myvec{3 , 2}$ and $\myvec{\alpha , \beta}$ is 12 square units. Then the least possible length of a line segment joining the origin to a point in $\vec{A}$, is 
	\hfill [Aug 2021]
		\begin{enumerate}
			\item  $\frac{4}{\sqrt{5}}$
			\item  $\frac{16}{\sqrt{5}}$
			\item  $\frac{8}{\sqrt{5}}$
			\item  $\frac{12}{\sqrt{12}}$
		\end{enumerate}
	\item
	The number of solutions of the equation $32^{\tan^{2}{x}} + 32^{\sec^{2}{x}} = 81$, $0 \leq x \leq \frac{\pi}{4}$ is:
	\hfill [Aug 2021]
		\begin{enumerate}
			\item 3
			\item 1
			\item 0
			\item 2
		\end{enumerate}
	\item 
	Let $f$ be any continuous function on $\sbrak{0, 2}$ and twice differentiable on $\brak{0, 2}$. If $f\brak{0} = 0$, $f\brak{1} = 1$ and $f\brak{2} = 2$, then
	\hfill [Aug 2021]
		\begin{enumerate}
			\item  $f^{\prime\prime}\brak{x} = 0$ for all $x \in \brak{0, 2}$
			\item  $f^{\prime\prime}\brak{x} = 0$ for some $x \in \brak{0, 2}$
			\item  $f^{\prime}\brak{x} = 0$ for some $x \in \sbrak{0, 2}$
			\item  $f^{\prime\prime}\brak{x} > 0$ for all $x \in \brak{0, 2}$
		\end{enumerate}
	\item 
	If $\sbrak{x}$ is the greatest integer $\leq x$, then $\pi^{2}\int_{0}^{2} \sin{\frac{\pi x}{2}}(x-\sbrak{x})^{\sbrak{x}} dx$ is equal to:
	\hfill [Aug 2021]
		\begin{enumerate}
			\item $2\brak{\pi-1}$
			\item $4\brak{\pi-1}$
			\item $4\brak{\pi+1}$
			\item $2\brak{\pi+1}$
		\end{enumerate}
	\item 
	The mean and variance of 7 observations are 8 and 16 respectively. If two observations are 6 and 8, then the variance of the remaining 5 observations is:
	\hfill [Aug 2021]
		\begin{enumerate}
			\item $\frac{92}{5}$
			\item $\frac{134}{5}$
			\item $\frac{536}{25}$
			\item $\frac{112}{5}$
		\end{enumerate}

