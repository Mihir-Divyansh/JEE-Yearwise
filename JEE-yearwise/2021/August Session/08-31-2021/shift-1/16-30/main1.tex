\iffalse
\title{2021}
\author{ee24btech11009}
\section{mcq-single}
\fi
%\begin{enumerate}[start=16]
\item If $\frac{dy}{dx} = \frac{2^{x+y}-2^x}{2^y}$ with $y\brak{0}=1$, then $y\brak{1}$ is equal to:\hfill{[Aug 2021]}
\begin{enumerate}
\item $\log_2\brak{2 + e}$
\item $\log_2\brak{1 + e}$
\item $\log_2\brak{2e}$
\item $\log_2\brak{1 + e^2}$
\end{enumerate}
\item $\displaystyle \lim_{x \to 0} \frac{\sin^2\brak{\pi\cos 4x}}{x^4}$
is equal to:\hfill{[Aug 2021]}
\begin{enumerate}
\item $\pi^2$  
\item $2\pi^2$
\item $4\pi^2$
\item $4\pi$
\end{enumerate}
\item A vertical pole is divided in the ratio $3:7$ by a mark on it with lower part shorter than the upper part. If the lower part subtend equal angles at a point on the ground $18$ m away from the base of the pole, then the height of the pole $\brak{\text{in meters}}$?\hfill{[Aug 2021]}
\begin{enumerate}
\item $12\sqrt{15}$
\item$12\sqrt{10}$
\item $8\sqrt{10}$
\item $6\sqrt{10}$
\end{enumerate}
\item If $a_r = \cos\brak{\frac{2r\pi}{9}} + i\sin\brak{\frac{2r\pi}{9}}, r= 1,2,3,\cdots,i$ then the determinant 
$\mydet{
a_1 & a_2 & a_3 \\
a_4 & a_5 & a_6 \\
a_7 & a_8 & a_9}
$
 is equal to:\hfill{[Aug 2021]}
\begin{enumerate}
\item $a_2a_6 - a_4a_8$
\item $a_9$
\item $a_1a_9 - a_3a_7$
\item $a_5$
\end{enumerate}
\item The line $12x\cos\theta+5y\sin\theta = 60$ is tangent to which of the following curves?\hfill{[Aug 2021]}
\begin{enumerate}
\item $x^2 + y^2 = 169$
\item $144x^2 + 25y^2 = 3600$
\item $25x^2 + 12y^2 = 3600$
\item $x^2 + y^2 = 60$
\end{enumerate}
%\end{enumerate}
