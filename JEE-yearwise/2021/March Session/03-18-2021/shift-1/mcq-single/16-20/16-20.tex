\iffalse
\title{2021}
\author{EE24BTECH11005}
\section{mcq-single}
\fi
\item If the functions are defined as $f\brak{x}=\sqrt{x}$ and $g\brak{x}=\sqrt{1-x}$, then what is the common domain of the following functions:\\
	$f+g,f-g,\frac{f}{g},\frac{g}{f},g-f$ where $\brak{f\pm g} \brak{x}=f\brak{x} \pm g\brak{x}, \brak{\frac{f}{g}}=\frac{f\brak{x}}{g\brak{x}}$
		\hfill \brak{\text{Mar 2021}}\\ \begin{enumerate}
				\begin{multicols}{2}
					\item $0\le x \le 1$
				\columnbreak
					\item $0 \le x < 1$
				\end{multicols}
				\begin{multicols}{2}
					\item $0 < x < 1$
				\columnbreak
					\item $0<x\le 1$
				\end{multicols}
		\end{enumerate}
	\item If \begin{align*}
			f\brak{x}=
		\begin{cases}
			\frac{1}{\abs{x}} &; \abs{x}\ge 1\\
			ax^2+b &;\abs{x}<1
		\end{cases}
	\end{align*}
		is differentiable at every point of the domain, then the values of $a$ and $b$ are respectively
		\hfill \brak{\text{Mar 2021}}\\ \begin{enumerate}
				\begin{multicols}{2}
				\item $\frac{1}{2},\frac{1}{2}$
				\columnbreak
			\item $\frac{1}{2},-\frac{3}{2}$
				\end{multicols}
				\begin{multicols}{2}
				\item $\frac{5}{2},-\frac{3}{2}$
				\columnbreak
			\item $-\frac{1}{2},\frac{3}{2}$
				\end{multicols}
		\end{enumerate}
	\item The sum pf all the $4-$digit distinct numbers that can be formed with the digits $1,2,2,3$ is,
	\hfill \brak{\text{Mar 2021}}\\ \begin{enumerate}
				\begin{multicols}{2}
				\item $26664$
				\columnbreak
			\item $122664$
				\end{multicols}
				\begin{multicols}{2}
				\item $122234$
				\columnbreak
			\item $22264$
				\end{multicols}
		\end{enumerate}
	\item Let ,
		\begin{align*}
			&A+2B=\myvec{1&2&0\\6&-3&3\\-5&3&1}\\
			&2A-B=\myvec{2&-1&5\\2&-1&6\\0&1&2}
		\end{align*}
		If $tr\brak{A}$ denotes the sum of all diagonal entries of the matrix $A$, then $tr\brak{A}-tr\brak{B}$ is,
		\hfill \brak{\text{Mar 2021}}\\ \begin{enumerate}
				\begin{multicols}{2}
					\item $0$
				\columnbreak
					\item $1$
				\end{multicols}
				\begin{multicols}{2}
					\item $2$
				\columnbreak
					\item $3$
				\end{multicols}
		\end{enumerate}
	\item The value of 
		\begin{align*}
			3+\frac{1}{4+\frac{1}{3+\frac{1}{4+\frac{1}{3+\dots \infty}}}}
		\end{align*}
		is equal to,
		\hfill \brak{\text{Mar 2021}}\\ \begin{enumerate}
				\begin{multicols}{2}
				\item $1.5+\sqrt{3}$
				\columnbreak
			\item $2+\sqrt{3}$
				\end{multicols}
				\begin{multicols}{2}
				\item $3+2\sqrt{3}$
				\columnbreak
			\item $4+\sqrt{3}$
				\end{multicols}
			\end{enumerate}

