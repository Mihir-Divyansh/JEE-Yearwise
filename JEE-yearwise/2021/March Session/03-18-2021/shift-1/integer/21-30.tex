\iffalse
\title{2021}
\author{EE24BTECH11005}
\section{integer}
\fi
		\item The number of times digit $3$ will be written when listing the integers from $1$ to $1000$ is \rule{2cm}{0.1pt}	\hfill \brak{2021-March}
	\item The equation of the planes parallel to the plane $x-2y+2z-3=0$ which are at unit distace from the point $\brak{1,2,3}$ is $ax+by+cz+d=0$. If $\brak{b-d}=k\brak{c-a}$, then the positive value of $k$ is\rule{2cm}{0.1pt}	\hfill \brak{2021-March} 
	\item Let $f\brak{x},g\brak{x}$ be two functions satisfying $f\brak{x^2}+g\brak{4-x}=4x^3$ and $g\brak{4-x}+g\brak{x}=0$, then the value of $\int_{-4}^4 f\brak{x^2} dx$ is,\rule{2cm}{0.1pt}	\hfill \brak{2021-March} 
	\item The mean age of $25$ teachers in a school is $40$ years. A teacher retires at the age of $60$ years and a new teacher is appointed in his place. If the mean age of the teachers in this school now is $39$ years, then the age of the newly appointed teacher is\rule{2cm}{0.1pt}	\hfill \brak{2021-March} 
	\item A square $ABCD$ has all its vertices on the curve $x^2y^2 = 1$. The midpoints of its sides also lie on the same curve. Then, the square of the area of $ABCD$ is\rule{2cm}{0.1pt}	\hfill \brak{2021-March}  
	\item The missing value in the following figure is,\rule{2cm}{0.1pt}	\hfill \brak{2021-March} 
\begin{figure}[H]
\centering
\begin{circuitikz}
\tikzstyle{every node}=[font=\large]
\draw  (7.5,14.75) circle (1cm);
\draw [short] (8.25,15.5) -- (8.75,16);
\draw [short] (6.75,14) -- (6.25,13.5);
\draw  (7.5,14.75) circle (1.75cm);
\draw [short] (8.75,13.5) -- (8.25,14);
\draw [short] (6.25,16) -- (6.75,15.5);
\draw [short] (5.75,14.75) -- (9.25,14.75);
\draw [short] (7.5,13) -- (7.5,16.5);
\node [font=\large] at (6.25,15.25) {1};
\node [font=\large] at (7,16) {2};
\node [font=\large] at (8,16) {3};
\node [font=\large] at (8.75,15.25) {5};
\node [font=\large] at (8.75,14.25) {4};
\node [font=\large] at (8,13.5) {7};
\node [font=\large] at (7,13.5) {8};
\node [font=\large] at (6.25,14.25) {12};
\node [font=\large] at (7,15) {1};
\node [font=\large] at (8,15) {?};
\node [font=\large] at (8,14.25) {$3^3$};
\node [font=\large] at (7,14.25) {$4^4$};
\end{circuitikz}
\end{figure}
	\item The number of solutions of the equation $\abs{\cot x}=\cot x +\brak{\frac{1}{\sin x}}$ in the interval $\sbrak{0, 2\pi}$ is\rule{2cm}{0.1pt}	\hfill \brak{2021-March} 
	\item Let $z_1,z_2$ be the roots of the equations $z_2+a_z+12=0$ and $z_1,z_2$ form an equilateral triangle with origin. Then, the value of $\abs{a}$ is \rule{2cm}{0.1pt}	\hfill \brak{2021-March} 
	\item Let the plane $ax+by+cz+d=0$ bisect the line joining the points $\myvec{4\\-3\\1},\myvec{2\\3\\-5}$ at right angles. If $a,b,c,d$ are integers, then the minimum value of $\brak{a^2+b^2+c^2+d^2}$ is,\rule{2cm}{0.1pt}	\hfill \brak{2021-March} 
	\item If $f\brak{x}=\int\frac{\sbrak{5x^8+x^6}}{\sbrak{x^2+1+2x^7}^2}dx, \brak{x\ge0}$, $f\brak{0}=0$ and $f\brak{1}=\frac{1}{k}$, then the value of $k$ is, \rule{2cm}{0.1pt}	\hfill \brak{2021-March} 
