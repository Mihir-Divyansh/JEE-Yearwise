\iffalse
\title{2021}
\author{AI24BTECH11009}
\section{mcq-single}
\fi
\item Two dice are rolled. If both dices have six faces numbered 1, 2, 3, 5, 7 and 11, then the probability that the sum of the numbers on the top faces is less than or equal to 8 is: \hfill[Mar 2021]
\begin{enumerate}
    \item $\frac{17}{36}$
    \item $\frac{4}{9}$
    \item $\frac{5}{12}$
    \item $\frac{1}{2}$\\
\end{enumerate}
\item The inverse of $y = 5^{\log{x}}$ is: \hfill[Mar 2021]
  \begin{enumerate}
      \item $x = 5^{\log{y}}$
      \item $x = y^{\log{5}}$
      \item $x = y^{\frac{1}{\log{5}}}$
      \item $x = 5^{\frac{1}{\log{y}}}$\\
  \end{enumerate}
\item In a school, there are three types of games to be played. Some of the students play two types of games, but none play all three games. Which Venn diagrams can justify the above statements. \hfill[Mar 2021]
\begin{figure}[!ht]
\centering
\resizebox{0.7\textwidth}{!}{%
\begin{circuitikz}
\tikzstyle{every node}=[font=\large]
\draw  (4.25,14.75) circle (2cm);
\draw  (11.5,14.75) circle (2cm);
\draw  (19.75,15) circle (2cm);
\draw  (4.25,14.25) circle (0.75cm);
\draw  (5,14.75) circle (0.75cm);
\draw  (10.25,14.75) circle (1.5cm);
\draw  (12.5,14.75) circle (1.5cm);
\draw  (18.75,15) circle (1.5cm);
\draw  (20.75,15) circle (1cm);
\node [font=\large] at (4.5,12.25) {P};
\node [font=\large] at (11.75,12.25) {Q};
\node [font=\large] at (19.75,12.25) {R};
\end{circuitikz}
}%
\end{figure}
\begin{enumerate}
         \item P and R
         \item P and Q
         \item None of these
         \item Q and R\\
     \end{enumerate}
 \item The area of the triangle with vertices $A (z)$, $B (iz)$ and $C (z + iz)$ is: \hfill[Mar 2021]
 \begin{enumerate}
     \item $\frac{1}{2}\abs{z+iz}^2$
     \item 1
     \item $\frac{1}{2}$
     \item $\frac{1}{2}\abs{z}^2$\\
 \end{enumerate}
\item The 
   value of $\lim_{x\rightarrow 0+} \frac{\sbrak{\cos^{-1}\brak{x - \sbrak{x}^2}\cdot\sin^{-1}\brak{x - \sbrak{x}^2}}}{\sbrak{x - x^3}}$, where \sbrak{x} denote the greatest integer $\leq x$ is: \hfill[Mar 2021]
\begin{enumerate}
    \item 0
    \item $\frac{\pi}{4}$
    \item $\frac{\pi}{2}$
    \item $\pi$\\
\end{enumerate}
