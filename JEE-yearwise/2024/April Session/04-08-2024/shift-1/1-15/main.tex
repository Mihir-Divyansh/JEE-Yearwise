\iffalse
  \title{Assignment}
  \author{EE24BTECH11044}
  \section{mcq-single}
\fi

	\item The set of all $\alpha$, for which the vectors $\vec{a}=\alpha t\hat{i}+6\hat{j}-3\hat{k}$ and $\vec{b}=t\hat{i}-2\hat{j}-2\alpha t\hat{k}$ are inclined at an obtuse angle for all $t\in\mathbb{R}$, is \hfill{[Apr 2024]}
		\begin{enumerate}
			\item $\brak{-\frac{4}{3},1}$\\
			\item $\lbrak{-\frac{4}{3}},\rsbrak{0}$\\
			\item $\lsbrak{0},\rbrak{1}$\\
			\item $\lbrak{-2},\rsbrak{0}$\\
		\end{enumerate}
	\item Let $\vec{P}\brak{x,y,z}$ be a point in the first octant, whose projection in the $xy$-plane is the point $\vec{Q}$. Let $OP=\gamma$; the angle between $OQ$ and the positive $x$-axis be $\theta$; and the angle between $OP$ and the positive $z$-axis be $\phi$, where $\vec{O}$ is the origin. Then the distance of $\vec{P}$ from $x$-axis is  \hfill{[Apr 2024]}

		\begin{enumerate}
			\item $\gamma\sqrt{1+\cos^2{\phi}\sin^2{\theta}}$\\
			\item $\gamma\sqrt{1-\sin^2{\phi}\cos^2{\theta}}$\\
			\item $\gamma\sqrt{1+\cos^2{\theta}\sin^2{\phi}}$\\
			\item $\gamma\sqrt{1-\sin^2{\theta}\cos^2{\phi}}$\\
		\end{enumerate}
	\item Let the circles $C_1:{\brak{x-\alpha}}^2+{\brak{y-\beta}}^2={r_1}^2$ and $C_2:{\brak{x-8}}^2+{\brak{y-\frac{15}{2}}}^2={r_2}^2$ touch each other externally at the point $\brak{6,6}$. If the point $\brak{6,6}$ divides the line segment joining the centers of the circles $C_1$ and $C_2$ internally in the ratio $2:1$, then $\brak{\alpha+\beta}+4\brak{{r_1}^2+{r_2}^2}$ equals  \hfill{[Apr 2024]}
		\begin{enumerate}
			\item $110$\\
			\item $125$\\
			\item $145$\\
			\item $130$\\
		\end{enumerate}
	\item Let $I\brak{x}=\int\frac{6}{\sin^2{x}{\brak{1-\cot{x}}}^2}dx$. If $I\brak{0}=3$, then $I\brak{\frac{\pi}{12}}$ is equal to  \hfill{[Apr 2024]}
		\begin{enumerate}
			\item $2\sqrt{3}$\\
			\item $6\sqrt{3}$\\
			\item $\sqrt{3}$\\
			\item $3\sqrt{3}$\\
		\end{enumerate}
	\item If $\sin{x}=-\frac{3}{5}$, where $\pi<x<\frac{3\pi}{2}$, then $80\brak{\tan^2{x}-\cos{x}}$ is equal to  \hfill{[Apr 2024]}
		\begin{enumerate}
			\item $19$\\
			\item $18$\\
			\item $109$\\
			\item $108$\\
		\end{enumerate}
	\item Let $z$ be a complex number such that $\abs{z+2}=1$ and $\operatorname{Im}\brak{\frac{z+1}{z+2}}=\frac{1}{5}$. Then the value of $\abs{\operatorname{Re}\brak{\overline{z+2}}}$ is \hfill{[Apr 2024]}
		\begin{enumerate}
			\item $\frac{2\sqrt{6}}{5}$\\
			\item $\frac{24}{5}$\\
			\item $\frac{\sqrt{6}}{5}$\\
			\item $\frac{1+\sqrt{6}}{5}$\\
		\end{enumerate}
	\item Let $f\brak{x}=4\cos^3{x}+3\sqrt{3}\cos^2{x}-10$. The number of points of local maxima of $f$ in interval $\brak{0,2\pi}$ is \hfill{[Apr 2024]}
		\begin{enumerate}
			\item $4$\\
			\item $1$\\
			\item $2$\\
			\item $3$\\
		\end{enumerate}
	\item For the function $f\brak{x}=\brak{\cos{x}}-x+1$, $x\in\mathbb{R}$, between the following two statements\\
		$\brak{S_1}$ $f\brak{x}=0$ for only one value of $x$ in $\sbrak{0,\pi}.$\\
		$\brak{S_2}$ $f\brak{x}$ is decreasing in $\sbrak{0,\frac{\pi}{2}}$ and increasing in \sbrak{\frac{\pi}{2},\pi}.\\ \hfill{[Apr 2024]}
		\begin{enumerate}
			\item Both $\brak{S_1}$ and $\brak{S_2}$ are incorrect.\\
			\item Only $\brak{S_1}$ is correct.\\
			\item Both $\brak{S_1}$ and $\brak{S_2}$ are correct.\\
			\item Only $\brak{S_2}$ is correct.\\
		\end{enumerate}
	\item Let $A=\begin{bmatrix}
			2 & a & 0 \\
			1 & 3 & 1 \\
			0 & 5 & b 
			\end{bmatrix}$. If $A^3=4A^2-A-21I$, where $I$ is the identity matrix of order $3\times3$, then $2a+3b$ is equal to \hfill{[Apr 2024]}
			\begin{enumerate}
				\item $-10$\\
				\item $-9$\\
				\item $-12$\\
				\item $-13$\\
			\end{enumerate}
	\item Let $\sbrak{t}$ be the greatest integer less than or equal to $t$. Let $A$ be the set of all prime factors of $2310$ and $f:A\to\mathbb{Z}$ be the function $f\brak{x}=\sbrak{\log_2\brak{x^2+\sbrak{\frac{x^3}{5}}}}$. The number of one-to-one functions from $A$ to the range of $f$ is  \hfill{[Apr 2024]}
		\begin{enumerate}
			\item $25$\\
			\item $120$\\
			\item $20$\\
			\item $24$\\
		\end{enumerate}
	\item Let $y=y\brak{x}$ be the solution of the differential equation $\brak{1+y^2}e^{\tan{x}}dx+\cos^2{x}\brak{1+e^{2\tan{x}}}dy=0,y\brak{0}=1$. Then $y\brak{\frac{\pi}{4}}$ is equal to \hfill{[Apr 2024]}
		\begin{enumerate}
			\item $\frac{1}{e^2}$\\
			\item $\frac{1}{e}$\\
			\item $\frac{2}{e}$\\
			\item $\frac{2}{e^2}$\\
		\end{enumerate}
	\item The equation of two sides $AB$ and $AC$ of a triangle $ABC$ are $4x+y=14$ and $3x-2y=5$, respectively. The point $\brak{2,-\frac{4}{3}}$ divides the third side $BC$ internally in the ratio $2:1$. The equation of the side $BC$ is \hfill{[Apr 2024]}
		\begin{enumerate}
			\item $x-3y-6=0$\\
			\item $x-6y-10=0$\\
			\item $x+3y+2=0$\\
			\item $x+6y+6=0$\\
		\end{enumerate}
	\item If the shortest distance between the lines
		$$L_1:\vec{r}=\brak{2+\lambda}\hat{i}+\brak{1-3\lambda}\hat{j}+\brak{3+4\lambda}\hat{k},    \lambda\in\mathbb{R}$$
		$$L_2:\vec{r}=2\brak{1+\mu}\hat{i}+3\brak{1+\mu}\hat{j}+\brak{5+\mu}\hat{k},\in\mathbb{R}$$
		is $\frac{m}{\sqrt{n}}$, where $\gcd\brak{m,n}=1$, then the value of $m+n$ equals  \hfill{[Apr 2024]}
		\begin{enumerate}
			\item $377$\\
			\item $390$\\
			\item $387$\\
			\item $384$\\
		\end{enumerate}
	\item If the set $R=\cbrak{\brak{a,b}:a+5b=42,a,b\in\mathbb{R}}$ has $m$ elements and $\sum_{n=1}^{m}\brak{1-i^{n!}}=x+iy$, where $i=\sqrt{-1}$, then the value of $m+x+y$ is \hfill{[Apr 2024]}
		\begin{enumerate}
			\item $8$\\
			\item $5$\\
			\item $4$\\
			\item $12$\\
		\end{enumerate}
	\item The value of $k\in\mathbb{N}$ for which the integral $I_n=\int_{0}^{1}{\brak{1-x^k}}^ndx$, $n\in\mathbb{N}$, satisfies $147I_{20}=148I_{21}$ is \hfill{[Apr 2024]}
		\begin{enumerate}
			\item $8$\\
			\item $10$\\
			\item $7$\\
			\item $14$\\
		\end{enumerate}

