\iffalse
	\title{2024}
	\author{AI24BTECH11003}
	\section{mcq-single}
\fi

%1
    \item Let $
    f\brak{x}=
    \begin{cases}
    -2 & -2\leq x \leq 0\\
    x-2 & 0 < x \leq2
    \end{cases}$ and $h\brak{x}=f\brak{\abs{x}}+\abs{f\brak{x}}$. Then $\underset{-2}{\overset{2}{\int}}h\brak{x} dx$ is equal to
    
    \hfill[Apr 2024]

        \begin{multicols}{4}
            \begin{enumerate}
                \item $1$
                \item $2$
                \item $6$
                \item $4$
            \end{enumerate}
        \end{multicols}

%2
    \item Let $f:R\to R$ be a function given by 
    $f\brak{x}=
    \begin{cases}
        \frac{1-\cos2x}{x^2} & x<0\\
        \alpha & x=0\\
        \frac{\beta\sqrt{1-\cos x}}{x} & x>0
    \end{cases}$ where $\alpha,\beta\in R$. If $f$ is continuous at $x=0$, then $\alpha^2+\beta^2$ is equal to
    
    \hfill[Apr 2024]

		\begin{multicols}{4}
			\begin{enumerate}
				\item $6$
				\item $48$
				\item $12$
				\item $3$
			\end{enumerate}
		\end{multicols}

%3
    \item A square is inscribed in the circle $x^2+y^2-10x-6y+30=0$. One side of this square is parallel to $y=x+3$. If $\brak{x_i,y_i}$ are the vertices of the square, then $\sum\brak{x_i^2+y_i^2}$ is equal to:
    
    \hfill[Apr 2024]

        \begin{multicols}{4}
            \begin{enumerate}
                \item $148$
                \item $156$
                \item $160$
                \item $152$
            \end{enumerate}
        \end{multicols}

%4
    \item The vertices of a triangle are $A\brak{-1,3},B\brak{-2,2}$ and $C\brak{3,-1}$. A new triangle is formed by shifting the sides of the triangle by one unit inwards. Then the equation of the side of the new triangle nearest to origin is:
    
    \hfill[Apr 2024]

		\begin{multicols}{2}
			\begin{enumerate}
				\item $x+y-\brak{2-\sqrt{2}}=0$
				\item $x-y-\brak{2+\sqrt{2}}=0$
				\item $x+y+\brak{2+\sqrt{2}}=0$
				\item $-x+y-\brak{2-\sqrt{2}}=0$
			\end{enumerate}
		\end{multicols}

%5
    \item The sum of all rational terms in the expansion of $\brak{2^\frac{1}{5}+5^\frac{1}{3}}^{15}$ is equal to:
    
    \hfill[Apr 2024]

		\begin{multicols}{4}
			\begin{enumerate}
				\item $3133$
				\item $633$
				\item $931$
				\item $6131$
			\end{enumerate}
		\end{multicols}
  
%6
    \item Let the sum of the maximum and the minimum values of the function $f\brak{x}=\frac{2x^2-3x+8}{2x^2+3x+8}$ be $\frac{m}{n}$, where $\gcd\brak{m,n}=1$. Then $m+n$ is equal to:
    
    \hfill[Apr 2024]

        \begin{multicols}{4}
            \begin{enumerate}
                \item $217$
                \item $201$
                \item $182$
                \item $195$
            \end{enumerate}
        \end{multicols}

%7
    \item Let the point on the line passing through the points $P\brak{1,-2,3}$ and $Q\brak{5,-4,7}$, farther from the origin and at a distance of 9 units from the point P, be $\brak{\alpha,\beta,\gamma}$. Then $\alpha^2+\beta^2+\gamma^2$ is equal to
    
    \hfill[Apr 2024]

        \begin{multicols}{4}
            \begin{enumerate}
                \item $165$
                \item $150$
                \item $160$
                \item $155$
            \end{enumerate}
        \end{multicols}
		
%8
    \item There are 5 points $P_1,P_2,P_3,P_4,P_5$ on the side $AB$, excluding points $A$ and $B$, of a triangle $ABC$. Similarly, there are 6 points $P_6,P_7,\cdots,P_{11}$ on side $BC$ and 7 points $P_{12},P_{13},\cdots,P_{18}$ on the side $CA$ of the triangle. The number of triangles, that can be formed using the points $P_1,P_2,\cdots,P_{18}$ as vertices, is:
    
    \hfill[Apr 2024]

        \begin{multicols}{4}
            \begin{enumerate}
                \item $771$
                \item $776$
                \item $751$
                \item $796$
            \end{enumerate}
        \end{multicols}

%9
    \item If the domain of the function $\arcsin\brak{\frac{3x-22}{2x-19}}+\log_e\brak{\frac{2x^2-8x+5}{x^2-3x-10}}$ is $\left(\alpha,\beta\right]$, then $3\alpha+10\beta$ is equal to
    
    \hfill[Apr 2024]

        \begin{multicols}{4}
            \begin{enumerate}
                \item $100$
                \item $95$
                \item $98$
                \item $97$
            \end{enumerate}
        \end{multicols}

%10
    \item Let $\alpha\in\brak{0,\infty}$ and $A=\myvec{1 & 2 & \alpha \\ 1 & 0 & 1 \\ 0 & 1 & 2}$. If $\det\brak{adj\brak{2A-A^\top}}\cdot adj\brak{A-2A^\top}=2^8$, then $\brak{\det\brak{A}}^2$ is equal to
    
    \hfill[Apr 2024]

        \begin{multicols}{4}
            \begin{enumerate}
                \item $16$
                \item $1$
                \item $49$
                \item $36$
            \end{enumerate}
        \end{multicols}
        
%11
    \item Let $\alpha,\beta\in R$. Let the mean and variance of 6 observations $-3,4,7,-6\alpha,\beta$ be 2 and 23, respectively. The mean deviation about the mean of these 6 observations is:
    
    \hfill[Apr 2024]

        \begin{multicols}{4}
            \begin{enumerate}
                \item $\frac{14}{3}$
                \item $\frac{11}{3}$
                \item $\frac{13}{3}$
                \item $\frac{16}{3}$
            \end{enumerate}
        \end{multicols}

%12
    \item Let a unit vector which makes an angle of $60\degree$ with $2\hat{i}+2\hat{j}-\hat{k}$ and an angle of $45\degree$ with $\hat{i}-\hat{k}$ be $\overrightarrow{C}$. Then $\overrightarrow{C}+\brak{-\frac{1}{2}\hat{i}+\frac{1}{3\sqrt{2}}\hat{j}-\frac{\sqrt{2}}{3}\hat{k}}$ is:
    
    \hfill[Apr 2024]

        \begin{multicols}{2}
            \begin{enumerate}
                \item $-\frac{\sqrt{2}}{3}\hat{i}+\frac{\sqrt{2}}{3}\hat{j}+\brak{\frac{1}{2}+\frac{2\sqrt{2}}{3}}\hat{k}$
                \item $\frac{\sqrt{2}}{3}\hat{i}-\frac{1}{2}\hat{k}$
                \item $\frac{\sqrt{2}}{3}\hat{i}+\frac{1}{3\sqrt{2}}\hat{j}-\frac{1}{2}\hat{k}$
                \item $\brak{\frac{1}{\sqrt{3}}+\frac{1}{2}}\hat{i}+\brak{\frac{1}{\sqrt{3}}-\frac{1}{3\sqrt{2}}}\hat{j}+\brak{\frac{1}{\sqrt{3}}+\frac{\sqrt{2}}{3}}\hat{k}$ 
            \end{enumerate}
        \end{multicols}
        
%13
    \item If 2 and 6 are the roots of the equation $ax^2+bx+1=0$, then the quadratic equation, whose roots are $\frac{1}{2a+b}$ and $\frac{1}{6a+b}$, is:
    
    \hfill[Apr 2024]

        \begin{multicols}{4}
            \begin{enumerate}
                \item $x^2+8x+12=0$
                \item $x^2+10x+16=0$
                \item $2x^2+11x+12=0$
                \item $4x^2+14x+12=0$
            \end{enumerate}
        \end{multicols}

%14
    \item If the system of equations $x+\brak{\sqrt{2}\sin\alpha}y+\brak{\sqrt{2}\cos\alpha}z=0, x+\brak{\cos\alpha}y+\brak{\sin\alpha}z=0, x+\brak{\sin\alpha}y-\brak{\cos\alpha}z=0$ has a non-trivial solution, then $\alpha\in\brak{0,\frac{\pi}{2}}$ is equal to:
    
    \hfill[Apr 2024]

        \begin{multicols}{4}
            \begin{enumerate}
                \item $\frac{11\pi}{24}$
                \item $\frac{7\pi}{24}$
                \item $\frac{3\pi}{4}$
                \item $\frac{5\pi}{24}$
            \end{enumerate}
        \end{multicols}

%15
    \item Three urns A, B, and C contain 7 red, 5 black; 5 red, 7 black and 6 red, 6 black balls, respectively. One of the urns is selected at random and a ball is drawn from it. If the ball drawn is black, then the probability that it is drawn from urn A is:
    
    \hfill[Apr 2024]

        \begin{multicols}{4}
            \begin{enumerate}
                \item $\frac{5}{18}$
                \item $\frac{7}{18}$
                \item $\frac{5}{16}$
                \item $\frac{4}{17}$
            \end{enumerate}
        \end{multicols}

