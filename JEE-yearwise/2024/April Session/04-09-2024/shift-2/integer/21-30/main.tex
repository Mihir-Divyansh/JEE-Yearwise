\iffalse
\title{Assignment}
\author{EE24BTECH11035}
\section{integer}
\fi

\item For a differentiable function $f : \mathbb{R} \to \mathbb{R}$, suppose $f'(x) = 3f(x) + a$, where $a \in \mathbb{R}$,$f(0) = 1$ and $\lim_{x \to \infty} f(x) = 7$. Then $9f(-\log_e 3)$ is equal to \dots.\hfill{(April 2024)}

\item Consider the circle $C: x^2 + y^2 = 4$ and the parabola $P: y^2 = 8x$. If the set of all values of $\alpha$, for which three chords of the circle $C$ on three distinct lines passing through the point $(\alpha, 0)$ are bisected by the parabola $P$ is the interval $(p, q)$, then $(2q - p)^2$ is equal to \dots.
	\hfill{(April 2024)}
\item If
\begin{equation*}
\left( \frac{1}{\alpha + 1} + \frac{1}{\alpha + 2} + \ldots + \frac{1}{\alpha + 1012} \right) 
- \left( \frac{1}{2 \cdot 1} + \frac{1}{4 \cdot 3} + \frac{1}{6 \cdot 5} + \ldots + \frac{1}{2024 \cdot 2023} \right) 
= \frac{1}{2024}
\end{equation*}
then $\alpha$ is equal to \dots.
\hfill{(April 2024)}
\item The number of integers, between 100 and 1000 having the sum of their digits equal to 14, is \dots.
\hfill{(April 2024)}
\item Consider the matrices 
\begin{equation*}
A = \myvec{ 2 & -5 \\ 3 & 20 },B=\myvec{20 \\ m}\quad and\quad X = \myvec{ x \\ y }.
\end{equation*}
Let the set of all $n$, for which the system of equations $AX = B$ has a negative solution (i.e., $x < 0$ and $y < 0$), be the interval $(a, b)$. Then 
\begin{equation*}
8\int_a^b  |A|\, dm
\end{equation*}
is equal to \dots.\hfill{(April 2024)}
\item Let $A = \{(x, y): 2x + 3y = 23, x, y \in \mathbb{N}\}$ and $B = \{x, y \in \mathbb{A}\}$. Then the number of one-one functions from $A$ to $B$ is equal to \dots.\hfill{(April 2024)}

\item Let the inverse trigonometric functions take principal values. The number of real solutions of the equation 
\begin{equation*}
2\sin^{-1} x + 3\cos^{-1} x = \frac{2\pi}{5}
\end{equation*}
is \dots.
\hfill{(April 2024)}
\item Let the set of all values of $p$, for which 
\begin{equation*}
f(x) = (p^2 - 6p + 8)(\sin^2 2x - \cos^2 2x) + 2(2 - p)x + 7
\end{equation*}
does not have any critical point, be the interval $(a, b)$. Then $16ab$ is equal to \dots.
\hfill{(April 2024)}
\item Let $A, B, C$ be three points on the parabola $y^2 = 6x$ and let the line segment $AB$ meet the line $L$ through $C$ parallel to the $x$-axis at the point $D$. Let $M$ and $N$ respectively be the feet of the perpendiculars from $A$ and $B$ on $L$. Then 
\begin{equation*}
\left(\frac{AM - BN}{CD}\right)^2
\end{equation*}
is equal to \dots.
\hfill{(April 2024)}
\item The square of the distance of the image of the point $(6, 1, 5)$ in the line 
\begin{equation*}
\frac{x - 1}{3} = \frac{y}{2} = \frac{z - 2}{4}
\end{equation*}
from the origin is \dots.
\hfill{(April 2024)}
%\end{enumerate}

%\end{document}

