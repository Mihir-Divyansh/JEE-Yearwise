\iffalse
\title{04-06-2024-shift-1-16-30}
\author{AI24BTECH11011 }
\section{mains}
\fi
        \item For $n \in \mathbb{N}$, if $\cot^{-1}3+\cot^{-1}4+\cot^{-1}5+\cot^{1}n=\frac{\pi}{4}$. then n is equal to $\makebox[3cm][l]{\underline{\hspace{1cm}}}$.
\hfill{(April 2024)} 
\item Let $\alpha\beta\gamma=45;\alpha,\beta,\gamma \in \mathbb{R}$. If $x\brak{\alpha,1,2}+y\brak{1,\beta,2}+z\brak{2,3,\gamma}=\brak{0,0,0}$ for some $x,y,z \in \mathbb{R},xyz \neq 0$, then $6\alpha+4\beta+\gamma$ is equal to $\makebox[3cm][l]{\underline{\hspace{1cm}}}$.
\hfill{(April 2024)} 
\item Let the first term of a series be $T_1=6$ and its $r^{th}$ term $T_r=3T_{r-1}+6^r,r=2,3,\cdots,n$.If sum of the first n terms of the series is $\frac{1}{5}\brak{n^2-12n+39}\brak{4\cdot6^n-5\cdot 3^n +1}$, then n is equal to $\makebox[3cm][l]{\underline{\hspace{1cm}}}$.
\hfill{(April 2024)} 
\item Let $\vec{P}$ be the point $\brak{10,-2,-1}$ and $\vec{Q}$ be the foot of perpendicular drawn from the point $\vec{R}\brak{1,7,6}$ on the line passing through the points $\brak{2,-5,11}$ and $\brak{-6,7,-5}$.Then the length of the line segment $PQ$ is equal to $\makebox[3cm][l]{\underline{\hspace{1cm}}}$.
\hfill{(April 2024)} 
\item If the second, third and fourth terms in the expansion of $\brak{x+y}^n$ are 135,30 and $\frac{10}{3}$ respectively. Then $6\brak{n^3+x^2+y}$ is equal to $\makebox[3cm][l]{\underline{\hspace{1cm}}}$.
\hfill{(April 2024)} 
\item Let $L_1,L_2$ are the lines passing through the point $\vec{P}\brak{0,1}$ and touching the parabola $9x^2+12x+18y-14=0$. Let $\vec{Q}$ and $\vec{R}$ be the points on the lines $L_1,L_2$ such that $\Delta PQR$ is an isosceles triangle with base $QR$.If the slopes of the lines $QR$ are $m_1,m_2$, then $16\brak{m_1^2+m_2^2}$ is equal to $\makebox[3cm][l]{\underline{\hspace{1cm}}}$.
\hfill{(April 2024)} 
\item Given the vectors:$\vec{a} = 2\hat{i} - 3\hat{j} + 4\hat{k}, \quad \vec{b} = 3\hat{i} + 4\hat{j} - 5\hat{k}$ and a vector $\vec{c}$ such that:$\vec{a} \times (\vec{b} + \vec{c}) + \vec{b} \times \vec{c} = \hat{i} + 8\hat{j} + 13\hat{k}$ with the condition: $\vec{a} \cdot \vec{c} = 13$ Then,  $\brak{( 24 - \vec{b} \cdot \vec{c} }$ is equal to $\makebox[3cm][l]{\underline{\hspace{1cm}}}$.
\hfill{(April 2024)} 
\item Let  conic $C$ pass through the point $\brak{4,-2}$ and $\vec{P}\brak{x,y},x \geq 3,$ be any point on $C$. Let the slope of the line touching the conic $C$ only at a single point $\vec{P}$ be half the slope of the line joining the points $\vec{P}$ and $\brak{3,-5}$. If the focal distance of the point $\brak{7,1}$ on $C$ is d, then 12d equals  $\makebox[3cm][l]{\underline{\hspace{1cm}}}$.
\hfill{(April 2024)} 
\item Let $r_k=\frac{\int_0^1\brak{1-x^7}^kdx}{\int_0^1\brak{1-x^7}^{k+1}dx},k \in \mathbb{N}$. Then the value of $\sum_{k=1}^{10}\frac{1}{7\brak{r_k-1}}$ is equal to $\makebox[3cm][l]{\underline{\hspace{1cm}}}$.
\hfill{(April 2024)} 
\item Let $x_1,x_2,x_3,x_4$ be the solution of the equation $4x^4+8x^3-17x^2-12x+9=0$ and $\brak{4+x_1^2}\brak{4+x_2^2}\brak{4+x_3^2}\brak{4+x_4^2}=\frac{125}{16}m$. Then the value of m is $\makebox[3cm][l]{\underline{\hspace{1cm}}}$.
\hfill{(April 2024)} 

