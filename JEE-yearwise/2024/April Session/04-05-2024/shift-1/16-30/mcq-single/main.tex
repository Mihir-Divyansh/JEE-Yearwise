\iffalse
\title{Assignment-3}
\author{EE24BTECH11049-Patnam Shariq Faraz Muhammed}
\section{mcq-single}
\fi

%\begin{document}
%\end{doucment}
    %1st Question
    \item 
    The integral 
    \begin{align*}
        \int_0^{\frac{\pi}{4}}\frac{136\sin{x}}{3\sin{x} + 5\cos{x}}\, dx 
    \end{align*}
    is equal to:

    \hfill{\sbrak{\text{Apr 2024}}}

    \begin{enumerate}
    \begin{multicols}{2}
        \item $3\pi - 10\log_e\brak{2\sqrt{2}} + 10\log_e5$
        \item $3\pi - 50\log_e2 + 20\log_e5$
        \item $3\pi - 30\log_e2 + 20\log_e5$
        \item $3\pi - 25\log_e2 + 10\log_e5$
    \end{multicols}   
    \end{enumerate}

    %2nd Question 
    \item 
    If $y = y\brak{x}$ is the solution of the differential equation 
    \begin{align*}
        \frac{dy}{dx} + 2y = \sin{\brak{2x}},
    \end{align*}
    $y\brak{0} = \frac{3}{4}$, then $y\brak{\frac{\pi}{8}}$ is equal to:

    \hfill{\sbrak{\text{Apr 2024}}}

    \begin{enumerate}
    \begin{multicols}{4}
        \item $e^{\frac{\pi}{8}}$
        \item $e^{-\frac{\pi}{8}}$
        \item $e^{\frac{\pi}{4}}$
        \item $e^{-\frac{\pi}{4}}$
    \end{multicols}   
    \end{enumerate}

    %3rd Question 
    \item 
    Let two straight line drawn from the origin $\vec{O}$ intersect the line $3x + 4y = 12$ at the points $\vec{P}$ and $\vec{Q}$ such that $\Delta OPQ$ is an isosceles triangle and $\angle POQ = 90^{\degree}$. If $l = \vec{OP}^2 + \vec{PQ}^2 + \vec{QO}^2$, then the greatest integer less than or equal to $l$ is:

    \hfill{\sbrak{\text{Apr 2024}}}

    \begin{enumerate}
    \begin{multicols}{4}
        \item $44$
        \item $48$
        \item $42$
        \item $46$
    \end{multicols}   
    \end{enumerate}

    %4th Question 
    \item 
    If the function 
    \begin{align*}
        f\brak{x} = \frac{\sin{3x} + \alpha\sin{x} - \beta\cos{3x}}{x^3}, x \in \mathbf{R}
    \end{align*}
    is continuous at $x = 0$, then $f\brak{0}$ is equal to:

    \hfill{\sbrak{\text{Apr 2024}}}

    \begin{enumerate}
    \begin{multicols}{4}
        \item $-4$
        \item $4$
        \item $2$
        \item $-2$
    \end{multicols}   
    \end{enumerate}

    %5th Question 
    \item 
    consider the following statements: 
    \begin{align*}
        \textbf{Statement }\mathrm{I:} & \text{For any two complex numbers } z_1, z_2, \\
        & \brak{\abs{z_1} + \abs{z_2}}\abs{\frac{z_1}{\abs{z_1}} + \frac{z_2}{\abs{z_2}}} \leq 2\brak{\abs{z_1} + \abs{z_2}}, \text{ and }\\
        \textbf{Statement }\mathrm{II:} & \text{If } x, y, z \text{ and three distinct complex numbers and } \\
        & a, b, c \text{ are three positive real numbers such that } \\
        & \frac{a}{\abs{y - z}} = \frac{b}{\abs{z - x}} = \frac{c}{\abs{x - y}}, \text{ then} \\
        & \frac{a^2}{y - z} + \frac{b^2}{z - x} + \frac{c^2}{x - y} = 1.
    \end{align*}
    Between the above two statements: 

    \hfill{\sbrak{\text{Apr 2024}}}
    
    \begin{enumerate}
        \item statement $\mathrm{I}$ is correct but statement $\mathrm{II}$ is incorrect. 
        \item both statement $\mathrm{I}$ and statement $\mathrm{II}$ are correct.
        \item statement $\mathrm{I}$ is incorrect but statement $\mathrm{II}$ is correct. 
        \item both statement $\mathrm{I}$ and statement $\mathrm{II}$ are incorrect.  
    \end{enumerate}
%\end{enumerate}
%\end{document}
