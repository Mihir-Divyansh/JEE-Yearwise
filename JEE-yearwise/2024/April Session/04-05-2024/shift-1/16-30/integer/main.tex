\iffalse
\title{Assignment-4}
\author{EE24BTECH11049-Patnam Shariq Faraz Muhammed}
\section{integer}
\fi

%\begin{document}
%\begin{enumerate}
    %1st Question
    \item
    Let $a_1, a_2, a_3, \dots $ be in arithmetic progression progression of positive terms. Let
    \begin{align*}
        A_k = a_1^2 - a_2^2 + a_3^2 - a_4^2 + \dots + a_{2k-1}^2 -a_{2k}^2.
    \end{align*} 
    If $A_3 = -153$, $A_5 = -435$ and $a_1^2 + a_2^2 + a_3^2 = 66$, then $a_{17} - A_7$ is equal to \rule{1cm}{0.1pt}

    \hfill{\sbrak{\text{Apr 2024}}}

    %2nd Question 
    \item 
    Suppose $\vec{AB}$ is focal chord of the parabola $y^2 = 12x$ of length $l$ and slope $m < \sqrt{3}$, If the distance of the chord $\vec{AB}$ from the origin is $d$, then $ld^2$ is equal to \rule{1cm}{0.1pt}

    \hfill{\sbrak{\text{Apr 2024}}}

    %3rd Question 
    \item 
    The number real roots of the equation $\abs{x}\abs{x + 2} -  5\abs{x + 1} - 1 = 0$ is \rule{1cm}{0.1pt}

    \hfill{\sbrak{\text{Apr 2024}}}

    %4th Question 
    \item 
    If 
    \begin{align*}
        S = \cbrak{a \in \mathbf{R} : \abs{2a - 1} = 3\sbrak{a} + 2\cbrak{a}}, A = 72\sum_{a \in S}a, 
    \end{align*}
    where $\sbrak{t}$ denotes the greatest integer less than or equal to $t$ and $\cbrak{t}$ represents the fractional part of $t$, then $A$ is equal to \rule{1cm}{0.1pt}

    \hfill{\sbrak{\text{Apr 2024}}}

    %5th Question 
    \item 
    Let $\vec{\Bar{a}} = \hat{i} - \vec{3}\hat{j} + \vec{7}\hat{k}$, $\vec{\Bar{b}} = \vec{2}\hat{i} - \hat{j} + \hat{k}$ and $\vec{\Bar{c}}$ be a vector such that $\brak{\vec{\Bar{a}} + 2\vec{\Bar{b}}} \times \vec{\Bar{c}} = 3\brak{\vec{\Bar{c}} \times \vec{\Bar{a}}}$. If $\vec{\Bar{a}} . \vec{\Bar{c}} = 130$, then $\vec{\Bar{b}} . \vec{\Bar{c}}$ is equal to \rule{1cm}{0.1pt}

    \hfill{\sbrak{\text{Apr 2024}}}
    
    %6th Question 
    \item 
    The area of the  region enclosed by the parabolas $y = x^2 - 5x$ and $y = 7x - x^2$ is \rule{1cm}{0.1pt}

    \hfill{\sbrak{\text{Apr 2024}}}

    %7th Question 
    \item 
    let $f$ be a differentiable function in the interval $\brak{0, \infty}$ such that $f\brak{1} = 1$ and 
    \begin{align*}
        \lim_{t \to x} \frac{t^2f\brak{x} - x^2f\brak{t}}{t - x} = 1, \forall x > 0.
    \end{align*}
    Then $2f\brak{2} + 3f\brak{3}$ is equal to \rule{1cm}{0.1pt}

    \hfill{\sbrak{\text{Apr 2024}}}

    %8th Question 
    \item 
    If the constant term in the expansion of 
    \begin{align*}
        \brak{1 + 2x - 3x^3}\brak{\frac{3}{2}x^2 - \frac{1}{3x}}^9 
    \end{align*}
    is $p$, then $108p$ is equal to \rule{1cm}{0.1pt}

    \hfill{\sbrak{\text{Apr 2024}}}

    %9th Question 
    \item 
    From a lot of $10$ items, which include $3$ defective items, a sample of $5$ items is drawn at random. Let the random variable $X$ denote the numbers of defective items in the sample. If the variance of $X$ is $\sigma$, then $96\sigma^2$ is equal to \rule{1cm}{0.1pt}

    \hfill{\sbrak{\text{Apr 2024}}}

    %10th Question 
    \item 
    The number of ways of getting sum $16$ on throwing a dice four times is \rule{1cm}{0.1pt}

    \hfill{\sbrak{\text{Apr 2024}}}

%\end{enumerate}
%\end{document}
