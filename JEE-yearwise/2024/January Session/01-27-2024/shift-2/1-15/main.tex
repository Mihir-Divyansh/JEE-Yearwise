\iffalse
\title{Assignment}
\author{ee24btech11059}
\section{mcq-single}
\fi
   	\item{
    		
    		For $\lambda>0$, let $\theta$ be the angle between the vectors $\vec{a}=\hat{i}+\lambda \hat{j}-3 \hat{k}$ and $\vec{b}=3 \hat{i}-\hat{j}+2 \hat{k}$. If the vectors $\vec{a}+\vec{b}$ and $\vec{a}-\vec{b}$ are mutually perpendicular, then the value of $(14 \cos\theta)^{2}$ is equal to\text{ }
    		\hfill
    		{[Jan 2024]}
    		\begin{multicols}{4}
    			\begin{enumerate}
    				\item 20
    				\item 25
    				\item 40
    				\item 50
    			\end{enumerate}
    		\end{multicols}
    	}
    \item{
          	Let $A=\myvec{1 & 2 \\ 0 & 1}$ and $B=I+\operatorname{adj}(A)+(\operatorname{adj} A)^{2}+\ldots+(\operatorname{adj} A)^{10}$
          	Then, the sum of all the elements of the matrix B is: \text{  } \hfill
                {[Jan 2024]}
                \begin{multicols}{4}
					\begin{enumerate}
						\item -88
						\item -124
						\item 22
						\item -110
					\end{enumerate}
				\end{multicols}
            }
    %code by ysiddhanth 

\item{
        	
        	
        	Let $y=y(x)$ be the solution of the differential equation 
        	$\brak{x^{2}+4}^{2} d y+\brak{2 x^{3} y+8 x y-2} d x=0$. If $y\brak{0}=0$, then $y\brak{2}$ is equal to
        	\hfill
        	{[Jan 2024]}
        	\begin{multicols}{4}
        		\begin{enumerate}
        			\item  $\frac{\pi}{32}$
        			\item  $2\pi$
        			\item  $\frac{\pi}{16}$
        			\item  $\frac{\pi}{8}$
        		\end{enumerate}
        	\end{multicols}
        	
        }
    \item{
     		
     		Let C be a circle with radius $\sqrt{10}$ units and centre at the origin. Let the 
     		line $x+y=2$ intersects the circle C at the points P and Q. Let MN be a chord of C 
     		of length 2 unit and slope -1. Then, a distance (in units) between the chord PQ and the chord MN is \text{ }
            \hfill
            {[Jan 2024]}
            \begin{multicols}{4}
                \begin{enumerate}
                	\item $\sqrt{2}+1$
                	\item $3-\sqrt{2}$
                	\item $2-\sqrt{3}$
                	\item $\sqrt{2}-1$
                \end{enumerate}
            \end{multicols}
        
        }
    \item{
            
            Consider a hyperbola H having centre at the origin and foci on the x -axis. Let $\mathrm{C}_{1}$ be the circle touching the hyperbola H and having the centre at the origin. Let $\mathrm{C}_{2}$ be the circle touching the hyperbola H at its vertex and having the centre at one of its foci. If areas (in sq units) of $\mathrm{C}_{1}$ and $\mathrm{C}_{2}$ are $36 \pi$ and $4 \pi$, respectively, then the length (in units) of latus rectum of H is \text{ }
           	\hfill
                {[Jan 2024]}
            
           \begin{multicols}{4}
            	\begin{enumerate}
            		\item $\frac{14}{3}$
            		\item $\frac{28}{3}$
            		\item $\frac{11}{3}$
            		\item $\frac{10}{3}$
            	\end{enumerate}
            \end{multicols}
        
        }
        \item{
        	
        	Let $f\brak{x}=3 \sqrt{x-2}+\sqrt{4-x}$ be a real valued function. If $\alpha$ and $\beta$ are respectively the minimum and the maximum values of $f$, then $\alpha^{2}+2 \beta^{2}$ is equal to\\
        	\text{ }
        	\hfill
        	{[Jan 2024]}
        	
        	\begin{multicols}{4}
        		\begin{enumerate}
        			\item 24
        			\item 44
        			\item 38
        			\item 42
        		\end{enumerate}
        	\end{multicols}
        	
        }
 	\item{
        	 If the mean of the following probability distribution of a radam variable X :
        	 \begin{center}
	        	\begin{tabular}{|c|c|c|c|c|c|}
	        		\hline X & 0 & 2 & 4 & 6 & 8 \\
	        		\hline $\mathrm{P}(\mathrm{X})$ & $a$ & $2 a$ & $a+b$ & $2 b$ & $3 b$ \\
	        		\hline
	        	\end{tabular}
        	\end{center}
        	is $\frac{46}{9}$, then the variance of the distribution is
        	\hfill
        	{[Jan 2024]}
        	
        	\begin{multicols}{4}
        		\begin{enumerate}
        			\item $\frac{173}{27}$
        			\item $\frac{151}{27}$
        			\item $\frac{581}{81}$
        			\item $\frac{566}{81}$
        		\end{enumerate}
        	\end{multicols}
        	
        }
 	
 	\item{
			
			Let P be the point of intersection of the lines $\frac{x-2}{1}=\frac{y-4}{5}=\frac{z-2}{1}$ and
			$\frac{x-3}{2}=\frac{y-2}{3}=\frac{z-3}{2}$. Then, the shortest distance of P from the line $4 x=2 y=z$ is\hfill
			{[Jan 2024]}
			
			\begin{multicols}{4}
				\begin{enumerate}
					\item $\frac{\sqrt{14}}{7}$
					\item $\frac{3 \sqrt{14}}{7}$
					\item $\frac{6 \sqrt{14}}{7}$
					\item $\frac{5 \sqrt{14}}{7}$
				\end{enumerate}
			\end{multicols}
			
		}
    \item{
          	
          	If the value of the integral $\int_{-1}^{1} \frac{\cos \alpha x}{1+3^{x}} d x$ is $\frac{2}{\pi}$.Then, a value of $\alpha$ is
             \text{ }
             \hfill
                {[Jan 2024]}
            \begin{multicols}{4}
                \begin{enumerate}
                	\item $\frac{\pi}{6}$
                	\item $\frac{\pi}{2}$
                	\item $\frac{\pi}{3}$
                	\item $\frac{\pi}{4}$
                \end{enumerate}
            \end{multicols}

        %code by ysiddhanth 
        
        }
    \item{
	        
	       	Let a relation R on $\mathrm{N} \times \mathrm{N}$ be defined as:
	        $\left(x_{1}, y_{1}\right) R\left(x_{2}, y_{2}\right)$ if and only if $x_{1} \leq x_{2}$ or $y_{1} \leq y_{2}$.
	        Consider the two statements:\\
	        (I) R is reflexive but not symmetric.\\
	        (II) $R$ is transitive
	        Then which one of the following is true?
             \hfill
                {[Jan 2024]}
            \begin{multicols}{2}
                \begin{enumerate}
                	\item Both (I) and (II) are correct.
                	\item Only (I) is correct.
                	\item Only (II) is correct.
                	\item Neither (I) nor (II) is correct.
                \end{enumerate}
            \end{multicols}
        
        }
    \item{
    		
    		Let PQ be a chord of the parabola $y^{2}=12 x$ and the midpoint of PQ be at $(4,1)$. Then, which of the following point lies on the line passing through the points $P$ and Q?
             \hfill
                {[Jan 2024]}
			\begin{multicols}{4}
				\begin{enumerate}
					\item $\brak{\frac{3}{2},-16}$
					\item $\brak{3,-3}$
					\item $\brak{2,-9}$
					\item $\brak{\frac{1}{2},-20}$
				\end{enumerate}
			\end{multicols}
        
        }
    \item{
        	
        	The area (in sq. units) of the region $S=\cbrak{z \in \mathbb{C}:|z-1| \leq 2 ;(z+\bar{z})+i(z-\bar{z}) \leq 2, \operatorname{Im}(z) \geq 0}$ is\hfill
                {[Jan 2024]}
				\begin{multicols}{4}
	                \begin{enumerate}
	                	\item $\frac{7 \pi}{4}$
	                	\item $\frac{7 \pi}{3}$
	                	\item $\frac{17 \pi}{8}$
	                	\item $\frac{3 \pi}{2}$
	                \end{enumerate}
				\end{multicols}
        
        }
 \item{
    	
	    	Let $\vec{a}=\hat{i}+\hat{j}+\hat{k}, \vec{b}=2 \hat{i}+4 \hat{j}-5 \hat{k}$ and $\vec{c}=x \hat{i}+2 \hat{j}+3 \hat{k}, x \in \mathbb{R}$.
	    	
	    	If $\vec{d}$ is the unit vector in the direction of $\vec{b}+\vec{c}$ such that $\vec{a} \cdot \vec{d}=1$, then $(\vec{a} \times \vec{b}) \cdot \vec{c}$ is equal to
	    	\text{   }\hfill
	    	{[Jan 2024]}
	    	\begin{multicols}{4}
	    		\begin{enumerate}
	    			\item 11
	    			\item 6
	    			\item 9
	    			\item 3
	    		\end{enumerate}
	    	\end{multicols}
	    	
	    }
    \item{
			
			Given that the inverse trigonometric function assumes principal values only. Let $x$, $y$ be any two real numbers in $\sbrak{-1,1}$ such that $\cos ^{-1} x-\sin ^{-1} y=\alpha, \frac{-\pi}{2} \leq \alpha \leq \pi$.
			
			Then, the minimum value of $x^{2}+y^{2}+2 x y \sin \alpha$ is
			\text{   }\hfill
			{[Jan 2024]}
			\begin{multicols}{4}
				\begin{enumerate}
						\item $\frac{1}{2}$
						\item 0
						\item -1
						\item $\frac{-1}{2}$
				\end{enumerate}
			\end{multicols}
			
		}

    \item{
        
           
           	Let $f(x)=\int_{0}^{x}\left(t+\sin \left(1-e^{t}\right)\right) d t, x \in \mathbb{R}$. Then, $\lim _{x \rightarrow 0} \frac{f(x)}{x^{3}}$ is equal to\\ \text{ }
             \hfill
              {[Jan 2024]}
			\begin{multicols}{2}              
	              		\begin{enumerate}
	              			\item $-\frac{2}{3}$
	              			\item $\frac{1}{6}$
	              			\item $-\frac{1}{6}$
	              			\item $\frac{2}{3}$
	              	\end{enumerate}
  			\end{multicols}      
        }



