\iffalse
\title{JEE 2024}
\author{ee24btech11018}
\section{integer}
\fi
%\begin{enumerate}
	\item A group of $40$ students appeared in an examination of $3$ subjects - Mathematics, Physics \& Chemistry. It was found that all students passed in at least one of the subjects, $20$ students passed in Mathematics, $25$ students passed in Physics, $16$ students passed in Chemistry, at most $11$ students passed in both Mathematics and Physics, at most $15$ students passed in both Physics and Chemistry, at most $15$ students passed in both Mathematics and Chemistry. The maximum number of students passed in all the three subjects is \hfill \sbrak{Jan 2024}
	\item If $d_1$ is the shortest distance between the lines $x + 1 = 2y = -12z$, $x = y + 2 = 6z - 6$ and $d_2$ is the shortest distance between the lines $\frac{x-1}{2}=\frac{y+8}{-7}=\frac{z-4}{5}$, $\frac{x-1}{2}=\frac{y-2}{1}=\frac{z-6}{-3}$, then the value of $\frac{32\sqrt{3d_1}}{d_2}$ is: \hfill \sbrak{Jan 2024}
	\item Let the latus rectum of the hyperbola $\frac{x^2}{9}-\frac{y^2}{b^2}=1$ subtend and angle of $\frac{\pi}{3}$ at the centre of the hyperbola. If $b^2$ is equal to $\frac{l}{m}\brak{1+\sqrt{n}}$, where $l$ and $m$ are the co-prime numbers, then $l^2+m^2+n^2$ is equal to \hfill \sbrak{Jan 2024}
	\item Let $A = \cbrak{1, 2, 3, \dots 7}$ and let $P\brak{A}$ denote the power set of $A$. If the number of functions $f:A\to P\brak{A}$ such that $a\in f\brak{a}$,$\forall a\in A$ is $m^n$, $m$ and $n\in\mathbb{N}$ and $m$ is least, then $m+n$ is equal to \hfill \sbrak{Jan 2024}
	\item The value $9\int_0^9\sbrak{\sqrt{\frac{10x}{x+1}}}dx$, where $\sbrak{t}$ denotes the greatest integer less than or equal to $t$, is  \hfill \sbrak{Jan 2024}
	\item Number of integral terms in the expansion of $\cbrak{7^{\brak{\frac{1}{2}}}+11^{\brak{\frac{1}{6}}}}^{824}$ is equal to \hfill \sbrak{Jan 2024}
	\item Let $y = y\brak{x}$ be the solution of the differential equation $\brak{1-x^2}dy=\sbrak{xy+\brak{x^3+2}\sqrt{3\brak{1-x^2}}}dx$, $-1<x<1$, $y\brak{0}=0$. If $y\brak{\frac{1}{2}}=\frac{m}{n}$, $m$ and $n$ are co-prime numbers, then $m+n$ is equal to \hfill \sbrak{Jan 2024} 
	\item Let $\alpha$, $\beta \in \mathbb{N}$ be the roots of the equation $x^2-70x+\lambda=0$, where $\frac{\lambda}{2},\frac{\lambda}{3}\notin \mathbb{N}$. if $\lambda$ assumes the minimum possible value, then $\frac{\brak{\sqrt{\alpha-1}+\sqrt{\beta-1}}\brak{\lambda+35}}{\abs{\alpha-\beta}}$ is equal to: \hfill \sbrak{Jan 2024}
	\item If the function $f\brak{x} = \begin{cases} \frac{1}{\abs{x}},& \abs{x}\geq2\\ ax^2+2b,& \abs{x}<2\end{cases}$ is differentiable on $\mathbb{R}$, then $48\brak{a+b}$ is equal to  \hfill \sbrak{Jan 2024}
	\item Let $\alpha=1^2+4^2+8^2+13^2+19^2+26^2+\cdots$ upto $10$ terms and $\beta=\sum\limits_{n=1}^{10}n^4$. If $4\alpha-\beta=55k+40$, then $k$ is equal to  \hfill \sbrak{Jan 2024}
%\end{enumerate}
%\end{document}
