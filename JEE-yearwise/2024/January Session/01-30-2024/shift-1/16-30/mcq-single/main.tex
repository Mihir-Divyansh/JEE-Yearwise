\iffalse
\title{JEE 2024}
\author{ee24btech11018}
\section{mcq-single}
\fi
%\begin{enumerate}
\item Let $M$ denote the median of the following frequency distributions.\\
	\begin{table}[h!]
		\centering
	\begin{tabular}{|c|c|c|c|c|c|}
\hline
Class & 0-4 & 4-8 & 8-12 & 12-16 & 16-20 \\ 
\hline
Frequency & 3 & 9 & 10 & 8 & 6 \\
\hline
\end{tabular}	
	\end{table}\\
		Then $20 M$ is equal to: \hfill \sbrak{Jan 2024} 
		\begin{enumerate}
	\item $416$
	\item $104$
	\item $52$
	\item $208$
		\end{enumerate}
	\item If $f\brak{x} = \mydet{2\cos^4x & 2\sin^4x & 3+\sin^2x \\ 3+2\cos^4x & 2\sin^4x & \sin^22x \\ 2\cos^4x & 3+\sin^4x & \sin^22x}$ then $\frac{1}{5}f^{\prime}\brak{0}$ is equal to  \hfill \sbrak{Jan 2024} 
		\begin{enumerate}
			\item $0$
			\item $2$
			\item $2$
			\item $6$
		\end{enumerate}
	\item Let $A\brak{2, 3, 5}$ and $C\brak{-3, 4, -2}$ be opposite vertices of a parallelogram $ABCD$. If the diagonal $\overrightarrow{BD} = \hat{i}+2\hat{j}+3\hat{k}$ then the area of the parallelogram is equal to \hfill \sbrak{Jan 2024} 
		\begin{enumerate}
			\item $\frac{1}{2}\sqrt{410}$
			\item $\frac{1}{2}\sqrt{474}$
			\item $\frac{1}{2}\sqrt{586}$
			\item $\frac{1}{2}\sqrt{306}$
		\end{enumerate}
	\item If $2\sin^3x + \sin 2x \cos x + 4\sin x - 4 = 0$ has exactly $3$ solutions in the interval $\sbrak{0,\frac{n\pi}{2}}, n \in \mathbb{N}$, then the roots of the equation $x^2+nx+\brak{n-3}=0$ belong to: \hfill \sbrak{Jan 2024} 
		\begin{enumerate}
			\item $\brak{0, \infty}$
			\item $\brak{-\infty, 0}$
			\item $\brak{-\frac{\sqrt{17}}{2}, \frac{\sqrt{17}}{2}}$
			\item $\mathbb{Z}$
		\end{enumerate}
	\item Let $f:\sbrak{-\frac{\pi}{2}, \frac{\pi}{2}} \to \mathbb{R}$ be a differentiable function such that $f\brak{0}=\frac{1}{2}$. If the $\lim\limits_{x\to0} \frac{x\int_o^xf\brak{t}dt}{e^{x^2}-1}=\alpha$, then $8\alpha^2$ is equal to: \hfill \sbrak{Jan 2024} 
		\begin{enumerate}
			\item $16$
			\item $2$
			\item $1$
			\item $4$
		\end{enumerate}
%\end{enumerate}
%\end{document}
