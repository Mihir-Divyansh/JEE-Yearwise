 \iffalse
  \title{2024}
  \author{EE24BTECH11010}
  \section{mcq-single}
\fi
 \item If $z$
 is a complex number, then the number of common roots of the equations $z^{1985} + z^{100} + 1 = 0 $ and $z^3 + 2z^2 + 2z + 1 = 0 $ 
, is equal to \hfill [Jan 2024]
\begin{enumerate}
    \begin{multicols}{2}
        \item 0
        \item 2
        \item 1
        \item 3
    \end{multicols}
\end{enumerate}

  \item Suppose $2-p, p, 2 - \alpha, \alpha$
 are the coefficients of four consecutive terms in the expansion of $\brak{1 + x}^n$
. Then the value of $p^2 - \alpha^2 + 6\alpha + 2p$
 equals \hfill[Jan 2024]
 \begin{enumerate}
     \begin{multicols}{2}
        \item 8
        \item 4
        \item 6
        \item 10
     \end{multicols}
 \end{enumerate}
 \item If the domain of the function $f\brak{x} = \log_e\brak{\frac{2x +3}{4x^2 + x -3}} + \cos^{-1}\brak{\frac{2x-1}{x+2}} $ is $ (\alpha, \beta ] $  , then the value of $5\beta - 4\alpha$
 is equal to \hfill[Jan 2024]
 \begin{enumerate}
     \begin{multicols}{2}
         \item 9
         \item 12
         \item 11
         \item 10
     \end{multicols}
 \end{enumerate}
 \item Let $f : \mathbb{R} \to \mathbb{R}$ be a function defined by $f\brak{x} = \frac{x}{\brak{1 + x^4} ^{\frac{1}{4}}}$ and $g\brak{x} = f\brak{f\brak{f\brak{f\brak{x}}}}$. Then, $18 \int_0^{\sqrt{2\sqrt{5}}} x^2 g\brak{x}dx$ is equal to \hfill[Jan 2024]
 \begin{enumerate}
     \begin{multicols}{2}
         \item 36
         \item 33
         \item 39
         \item 42
     \end{multicols}
 \end{enumerate}
 \item Let $R = \myvec{x & 0 & 0 \\ 0 & y & 0 \\ 0 & 0 & z}$ be a non-zero $3 \times 3$ matrix, where $x\sin{\theta} = y \sin{\brak{\theta + \frac{2\pi}{3}}} = z \sin{\brak{\theta + \frac{4\pi}{3}}} \ne 0, \theta \in \brak{0 , 2\pi}$. For a square matrix $M$, let trace $\brak{M}$ denote the sum of all diagonal entries of $M$. Then among the statements: \\ \\
 $\brak{\text{I}} \operatorname{Trace} \brak{R} = 0$\\
 $\brak{\text{II}}$ If $\operatorname{Trace}$ $\brak{\operatorname{adj}\brak{\operatorname{adj\brak{R}}}} = 0$, then $R$ has exactly one non-zero entry.
 \hfill[Jan 2024]
 \begin{enumerate}  
 \begin{multicols}{2}
         \item Only (I) is true
         \item Only (II) is true
         \item Both (I) and (II) are true
         \item Neither (I) nor (II) is true
 \end{multicols}
 \end{enumerate}
