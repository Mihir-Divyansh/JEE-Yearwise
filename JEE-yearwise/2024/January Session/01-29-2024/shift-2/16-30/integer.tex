\iffalse
\title{January 2024, shift 2}
\author{EE24BTECH11062}	
\section{integer}
\fi
%\begin{enumerate}
\item Let the slope of the line $45x+5y+3=0$ be $27r_1+\frac{9r_2}{2}$ for some $r_1,r_2\in R$ then $\lim_{x \to 3}\brak{ \int_{3}^{x} \frac{8t^2}{\frac{3r_2x}{2}-r_2x^2-r_1x^3-3x} \, dt}$ is equal to \hfill{[Jan 2024]}

\item Let the area of the region $\cbrak{\brak{x,y}:0\leq x \leq 3, 0\leq y\leq min\cbrak{x^2+2,2x+2}}$ be $A$. Then $12A$ is equal to \hfill{[Jan 2024]}

\item Let $f\brak{x}=\sqrt{\lim_{r \to x}\cbrak{\frac{2r^2\sbrak{\brak{f\brak{r}}^2-f\brak{x}f\brak{r}}}{r^2-x^2}-r^3e^{\frac{f\brak{r}}{r}}}}$ be differentiable in $\brak{-\infty,0}\cup\brak{0,\infty}$ and $f\brak{1}=1$. Then the value of $ea$, such that $f\brak{a}=0$, is equal to \hfill{[Jan 2024]}

\item Let for any three distinct consecutive terms $a,b,c$ of an A.P, the lines $ax+by+c=0$ be concurrent at the point $\vec{P}$ and $\vec{Q}\brak{\alpha,\beta}$ be a point such that the system of equations \\
$x+y+z=6\\
2x+5y+\alpha z=\beta\\
x+2y+3z=4$, has infinitely many solutions. Then $\brak{PQ}^2$ is equal to \hfill{[Jan 2024]}

\item Let $\vec{P}\brak{\alpha,\beta}$ be a point on the parabola $y^2=4x$. If $\vec{P}$ lies on the chord of the parabola $x^2=8y$ whose midpoint id $\brak{1,\frac{5}{4}}$, then $\brak{\alpha-28}\brak{\beta-8}$ is equal to \hfill{[Jan 2024]}

\item Let the set $C-\cbrak{\brak{x,y}\mid x^2-2^y=2023,x,y\in N}$. Then $\sum_{\brak{x,y}\in C}\brak{x+y}$ is equal to \hfill{[Jan 2024]}

\item If $\int_{\frac{\pi}{6}}^{\frac{\pi}{3}} \sqrt{1-\sin 2x} \, dx=\alpha+\beta\sqrt{2}+\gamma\sqrt{3}$, where $\alpha,\beta$ and $\gamma$ are rational numbers, then $3\alpha+4\beta-gamma$ is equal to \hfill{[Jan 2024]}

\item Let $\vec{O}$ be the origin, and $\vec{M}$ and $\vec{N}$ be the points on the lines $\frac{x-5}{4}=\frac{y-4}{1}=\frac{z-5}{3}$ and $\frac{x+8}{12}=\frac{y+2}{5}=\frac{z+11}{9}$ respectively such that $MN$ is the shortest distance between the given lines. Then $\overrightarrow{OM}\cdot\overrightarrow{ON}$ is equal to \hfill{[Jan 2024]}

\item Remainder when $64^{{32}^{32}}$ is divided by 9 is equal to \hfill{[Jan 2024]}

\item Let $\alpha,\beta$ be the roots of the equation $x^2-\sqrt{6}x+3=0$ such that $Im\brak{\alpha}>Im\brak{\beta}$. Let $a,b$ be integers not divisible by 3 and $n$ be a natural number such that $\frac{\alpha^{99}}{\beta}+\alpha^{99}=3^n\brak{a+ib},i=\sqrt{-1}$. Then $n+a+b$ is equal to \hfill{[Jan 2024]}
%\end{enumerate}
%\end{document}


