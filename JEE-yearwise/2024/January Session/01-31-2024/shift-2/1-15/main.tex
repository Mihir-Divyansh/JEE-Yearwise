\iffalse
  \title{Assignment}
  \author{ee24btech11030}
  \section{mcq-single}
\fi

%   \begin{enumerate}
\item  The number of ways in which 21 identical apples can be distributed among three children such that each child gets at least 2 apples, is \hfill{[JAN 2024]}
    \begin{multicols}{4}
    \begin{enumerate}
        \item 406
        \item 130
        \item 142
        \item 136
    \end{enumerate}
    \end{multicols}
    \bigskip
    \item Let A (a, b), B(3, 4) and (-6, -8) respectively denote the centroid, circumcentre and orthocentre of a triangle. Then, the distance of the point P(2a + 3, 7b + 5) from the line 2x + 3y - 4 = 0 measured parallel to the line x - 2y - 1 = 0 is  \hfill{[JAN 2024]}
    \begin{multicols}{4}
    \begin{enumerate}
        \item $\frac{15\sqrt{5}}{7}$
        \item $\frac{17\sqrt{5}}{6}$
        \item $\frac{17\sqrt{5}}{7}$
        \item $\frac{\sqrt{5}}{17}$
    \end{enumerate} 
    \end{multicols}
    \bigskip
    \item Let $z_1$ and $z_2$ be two complex number such that $z_1 + z_2 = 5$ and $z_1^3 + z_2^3 = 20 + 15i$ . Then  $|z_1^{4} + z_2^{4}|$ equal-\hfill{[JAN 2024]}
    \begin{multicols}{4}
    \begin{enumerate}
        \item $30\sqrt{3}$
        \item $75$
        \item $15\sqrt{15}$
        \item $25\sqrt{3}$
    \end{enumerate} 
    \end{multicols}
    \bigskip
    
    \item Let a variable line passing through the centre of the circle $x^2  + y^2 - 16x - 4y = 0$, meet the positive co-ordinate axes at the point A and B. Then the minimum value of OA + OB, where O is the origin, is equal to\hfill{[JAN 2024]}
    \begin{multicols}{4}
    \begin{enumerate}
        \item 12
        \item 18
        \item 20
        \item 24
    \end{enumerate} 
    \end{multicols}
    \bigskip
    \item Let f,g :(0,$\infty$) be two functions defined by $f(x) = \int_{-x}^{x}(|t| - t^2)e^{-r^{2}}dt$ and g(x) = $\int_{0}^{x^2}t^{\frac{1}{2}}e^{-t}dt$. Then the value of (f($\sqrt{\ln9}) + g(\sqrt{\ln9}))$ is equal to\hfill{[JAN 2024]}
    \begin{multicols}{4}
    \begin{enumerate}
        \item 6
        \item 9
        \item 8
        \item 10
    \end{enumerate} 
    \end{multicols}
    \bigskip
    \item Let $(\alpha,\beta,\gamma)$  be mirror image of the point (2, 3, 5) in the line $\frac{x - 1}{2} = \frac{y - 2}{3} = \frac{z - 3}{4}$. Then $2\alpha + 3\beta + 4\gamma$ is equal to\hfill{[JAN 2024]}
    \begin{multicols}{4}
    \begin{enumerate}
        \item 32
        \item 33
        \item 31
        \item 34
    \end{enumerate} 
    \end{multicols}
    \bigskip
    \item Let P be a parabola with vertex (2, 3) and directrix 2x + y = 6. Let an ellipse E : $\frac{x^2}{a^2} + \frac{y^2}{b^2} = 1$,a $>$ b of eccentricity $\frac{1}{\sqrt{2}}$ pass through the focus of the parabola P. Then the square of the length of the latus rectum of E, is  \hfill{[JAN 2024]}
    \begin{multicols}{4}
    \begin{enumerate}
        \item $\frac{385}{8}$
        \item $\frac{347}{8}$
        \item $\frac{512}{25}$
        \item $\frac{656}{25}$
    \end{enumerate} 
    \end{multicols}
    \bigskip
    \item The temperature T(t) of a body at time t = 0 is $160\,^{\circ}\mathrm{F}$
    and it decreases continuously as per the differential equation $\frac{dT}{dt} = -K(T - 80)$ , where K is positive constant. If T(15) = $120\,^{\circ}\mathrm{F}$, then T(45) is \hfill{[JAN 2024]}
    \begin{enumerate}
        \item $85\,^{\circ}\mathrm{F}$
        \item $95\,^{\circ}\mathrm{F}$
        \item $90\,^{\circ}\mathrm{F}$
        \item $80\,^{\circ}\mathrm{F}$
    \end{enumerate}
    \bigskip
    \item If $2^{\text{nd}}$, $8^{\text{th}}$, $44^{\text{th}}$ terms of A.P. are $1^{\text{st}},2^{\text{nd}}$ and $3^{\text{rd}}$ terms respectively of G.P. and first term of A.P. is 1 then the sum of first 20 terms of A.P. is  \hfill{[JAN 2024]}
    \begin{multicols}{4}
    \begin{enumerate}
        \item 970
        \item 916
        \item 980
        \item 990
    \end{enumerate} 
    \end{multicols}
    \bigskip
    \item Let f : $\rightarrow$ R $\rightarrow$ (0,$\infty$)  be strictly increasing function such that $\lim_{x \to \infty} \frac{f(7x)}{f(x)}$. Then, the value of $\lim_{x \to \infty} \left[\frac{f(5x)}{f(x)} - 1\right]$ is equal to \hfill{[JAN 2024]}
    \begin{multicols}{4}
    \begin{enumerate}
        \item 4
        \item 0
        \item $\frac{7}{5}$
        \item 1
    \end{enumerate} 
    \end{multicols}
    \bigskip
    \item The area of the region enclosed by the parabolas $y = 4 - x^2$ and $3y = (x - 4)^2$ is in (sq. unit)?\hfill{[JAN 2024]}
    \begin{multicols}{4}
    \begin{enumerate}
        \item $\frac{14}{3}$
        \item 4
        \item $\frac{32}{3}$
        \item 6
    \end{enumerate} 
    \end{multicols}
    \bigskip
    \item Let the mean and the variance of 6 observation a, b, 68, 44, 48, 60 be 55 and 194, respectively if a $>$ b, then a + 3b is\hfill{[JAN 2024]}
    \begin{multicols}{4}
    \begin{enumerate}
        \item 200
        \item 190
        \item 180
        \item 210
    \end{enumerate} 
    \end{multicols}
    \bigskip
    \item If the function f : ($-\infty , -1] \rightarrow (a , b]$ defined by f(x) = $e^{x^3 -3x + 1}$ is one-one and onto, then the distance of the point $\vec{P}\myvec{2b + 4, a + 2}$ from the line x + $ e^{-3} $y = 4 is :  \hfill{[JAN 2024]}
    \begin{multicols}{4}
    \begin{enumerate}
        \item 18
        \item 20
        \item 24
        \item 32
    \end{enumerate} 
    \end{multicols}
    \bigskip
    \item Consider the function f : $(0 , \infty) \rightarrow R$ defined by $f (x) = e^{-|\ln{x}|}$. If m and n be respectively the number of points at which f is not continuous and f is not differentiable, then m + n is \hfill{[JAN 2024]} 
    \begin{multicols}{4}
    \begin{enumerate}
        \item 0
        \item 3
        \item 1
        \item 2
    \end{enumerate} 
    \end{multicols}
    \bigskip
    \item The number of solutions, of the equation $e^{\sin
    {x}} - 2e^{-\sin{x}} = 2$ is\hfill{[JAN 2024]}
    \begin{multicols}{4}
    \begin{enumerate}
        \item 2
        \item more than 2
        \item 1
        \item 0
    \end{enumerate} 
    \end{multicols}
    \bigskip


%   \end{enumerate}
