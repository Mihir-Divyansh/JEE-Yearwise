\iffalse
    \title{2020}
    \author{ee24btech11009}
    \section{mcq-single}
\fi
%\begin{enumerate}
\item If $A=\myvec{\cos\theta&i\sin\theta\\i\sin\theta&\cos\theta},\left(\theta=\frac{\pi}{24}\right)$ and $A^5=\myvec{a&b\\c&d}$,where $i=\sqrt{-1}$,then which one of the following is not true?\hfill{[Sep 2020]}
\begin{enumerate}
    \item $0\leq a^2+b^2\leq1$
    \item $a^2-d^2=0$
    \item $a^2-b^2=\frac{1}{2}$
    \item $a^2-c^2=1$
\end{enumerate}
\item Let $\lfloor t\rfloor$ denote the greatest integer $\leq t$. Then the equation in $x$,$\lfloor x\rfloor^2+2\lfloor x+2\rfloor-7=0$ has:\hfill{[Sep 2020]}
\begin{enumerate}
    \item no integral solution
    \item exactly four integral solutions
    \item exactly two solutions
    \item infinitely many solutions
    \end{enumerate}
\item Let $\alpha$ and $\beta$ be the roots of $x^2-3x+p=0$ and $\gamma$ and $\delta$ be the roots of $x^2-6x+q=0$. If $\alpha,\beta,\gamma,\delta$ form a geometric progression. Then ratio $\brak{2q+p}:\brak{2q-p}$ is:\hfill{[Sep 2020]}
\begin{enumerate}
    \item 3 :1
    \item 33 :31
    \item 5 :3
    \item 9 :7
\end{enumerate}
\item Let $\frac{x^2}{a^2}+\frac{y^2}{b^2}=1$ $\brak{a>b}$ be a given ellipse, the length of whose latus rectum is 10. If its eccentricity is the maximum value of the function, $\phi\brak{t}=\frac{5}{12}+t-t^2$, then $a^2 + b^2$ is equal to\hfill{[Sep 2020]}
\begin{enumerate}
    \item 126
    \item 135
    \item 145
    \item 116
\end{enumerate}
\item A triangle $ABC$ lying in the first quadrant has two vertices as $ \vec{A}\brak{1,2}$ and $\vec{B}\brak{3,1}$. If $\angle BAC=90^\circ$, and ar$\brak{\triangle ABC}$ is $5\sqrt{5}$ s units, then the abscissa of the vertex $\vec{C}$ is:\hfill{[Sep 2020]}
\begin{enumerate}
    \item $1+\sqrt{5}$
    \item $2+\sqrt{5}$
    \item $1+2\sqrt{5}$
    \item $2\sqrt{5}-1$
\end{enumerate}
\item Let $f\brak{x}=\abs{x-2}$ and $g\brak{x}=f\brak{f\brak{x}}$,$x\in \sbrak{0, 4}$. Then $\int_0^3 \brak{g\brak{x}-f\brak{x}} dx$ is equal to:\hfill{[Sep 2020]}
\begin{enumerate}
    \item $\frac{3}{2}$
    \item 0
    \item $\frac{1}{2}$
    \item 1
\end{enumerate}
\item Given the following two statements : \\
$(S_{1}) : \brak{q \vee p} \rightarrow \brak{p \leftrightarrow \sim q} \text{ is a tautology.} \\
(S_{2}) : \sim q \wedge \brak{\sim p \leftrightarrow q} \text{ is a fallacy.} \\
\text{Then :}$ \\\hfill{[Sep 2020]}
\begin{enumerate}
\item $\text{ only } \brak{S_{1}} \text{ is correct.}$
\item $\text{ both } \brak{S_{1}} \text{ and } \brak{S_{2}} \text{ are correct.}$
\item $\text{ both } \brak{S_{1}} \text{ and } \brak{S_{2}} \text{ are not correct.}$
\item $\text{ only } \brak{S_{2}} \text{ is correct.}$
\end{enumerate}
\item Let $\vec{P}\brak{3,3}$ be a point on the hyperbola, $\frac{x^2}{a^2}-\frac{y^2}{b^2}=1$. If the normal to it at $\vec{P}$ intersects the x-axis at $\brak{9,0}$ and $e$ is its eccentricity, then the ordered pair $\brak{a^2, e^2}$ is equal to:\hfill{[Sep 2020]}
\begin{enumerate}
    \item $\brak{\frac{9}{2},3}$
    \item $\brak{\frac{9}{2},2}$
    \item $\brak{\frac{3}{2},2}$
    \item $\brak{9,3}$
\end{enumerate}
\item Let $f\brak{x}=\int\frac{\sqrt{x}}{\brak{1+x}^2} dx$. Then $f\brak{3}-f\brak{1}$ is equal to:\hfill{[Sep 2020]}
\begin{enumerate}
    \item $-\frac{\pi}{6}+\frac{1}{2}+\frac{\sqrt{3}}{4}$
    \item $\frac{\pi}{6}+\frac{1}{2}-\frac{\sqrt{3}}{4}$
    \item $-\frac{\pi}{12}+\frac{1}{2}+\frac{\sqrt{3}}{4}$
    \item $\frac{\pi}{12}+\frac{1}{2}-\frac{\sqrt{3}}{4}$
\end{enumerate}
\item A survey shows that $63\%$ of the people in a city read newspaper $ A$ whereas $76\%$ read newspaper $B$. If $x\%$ of the people read both the newspapers, then a possible value of $x$ can be:\hfill{[Sep 2020]}
\begin{enumerate}
\item 65 
\item 37
\item 29
\item 55
\end{enumerate}
\item Let $u=\frac{2z+i}{z-ki},z=x+iy$ and $k>0$. If the curve represented by $Re\brak{u}+Im\brak{u}=1$ intersects the y-axis at the points $\vec{P}$ and $\vec{Q}$ where $PQ=5$, then the value of $k$ is:\hfill{[Sep 2020]}
\begin{enumerate}
    \item 4
    \item $\frac{1}{2}$
    \item 2
    \item $\frac{3}{2}$
\end{enumerate}
\item Let $x_0$ be the point of local maxima of $f\brak{x}=\overrightarrow{a}\cdot\brak{\overrightarrow{b}\times \overrightarrow{c}}$, where $\overrightarrow{a}=x\hat{i}+2\hat{j}+3\hat{k}$,$\overrightarrow{b}=-2\hat{i}+x\hat{j}-\hat{k}$, and $\overrightarrow{c}=7\hat{i}-2\hat{j}+x\hat{k}$. Then the value of $ \overrightarrow{a}\cdot \overrightarrow{b}+\overrightarrow{b}\cdot \overrightarrow{c}+\overrightarrow{c}\cdot \overrightarrow{a}$ at $x=x_0$ is:\hfill{[Sep 2020]}
\begin{enumerate}
    \item -30
    \item 14
    \item -4
    \item -22
\end{enumerate}
\item Two vertical poles $ AB = 15 $ m and $CD = 10$ m
are standing apart on a horizontal ground with
points $\vec{A}$ and $\vec{C}$ on the ground. If $\vec{P}$ is the point
of intersection of $BC$ and $AD$, then the height
of $\vec{P}$ (in m) above the line $AC$ is :\hfill{[Sep 2020]}
\begin{enumerate}
\item  $\frac{20}{3}$
\item  $ 5 $
\item  $ \frac{10}{3}$
\item  $ 6 $
\end{enumerate}
\item The mean and variance of 8 observations are $10$ and $13.5$, respectively. If 6 of these observations are $5, 7, 10, 12, 14, 15, $then the absolute difference of the remaining two observations is :\hfill{[Sep 2020]}
\begin{enumerate}
\item $7$
\item $3$
\item $5$ 
\item $9$
\end{enumerate}
\item The integral $\int\brak{\frac{x}{x\sin x+\cos x}}^2 dx$ is equal to (where $C$ is a constant of integration):\hfill{[Sep 2020]}
\begin{enumerate}
    \item $\sec x+\frac{x\tan x}{x\sin x+\cos x}+C$
    \item $\sec x-\frac{x\tan x}{x\sin x+\cos x}+C$
    \item $\tan x+\frac{x\sec x}{x \sin x + \cos x}+C$
    \item $\tan x-\frac{x\sec x}{x\sin x+\cos x}+C$
\end{enumerate}

%\end{enumerate}

