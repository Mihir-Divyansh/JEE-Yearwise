\iffalse
\title{2020}
\author{EE24BTECH11063}
\section{mcq-single}
\fi
\item If for some $\alpha\in\;R$, the lines $L_1 : \frac{\brak{x+1}}{2}=\frac{\brak{y-2}}{-1}=\frac{\brak{z-1}}{2}$ and $L_2 : \frac{\brak{x+2}}{\alpha}=\frac{\brak{y+1}}{5-\alpha}=\frac{\brak{z+1}}{1}$ are coplanar, the the line $L_2$ passes through the point:  \hfill{[September 2020]}
    \begin{enumerate}
    
        
    
        \item \brak{2, -10. -2}
        \item \brak{10, -2, -2}
        \item \brak{10, 2, 2}
        \item \brak{-2, 10, 2}
        
        \end{enumerate}
        \item The value of $\left[\frac{\brak{-1+i\sqrt{3}}}{\brak{1-i}}\right]^{30}$
        \begin{enumerate}
        \begin{multicols}{4}
            \item $2^{15}i$
            \item $-2^{15}$
            \item $-2^{15}i$
            \item $6^5$
            \end{multicols}
        \end{enumerate}
\item Let y=y\brak{x} be the solution of the differential equation $\cos{x}\brak{\frac{dy}{dx}}+2y \sin{x}\;=\;\sin{2x},\;x\in\brak{0,\frac{\pi}{2}}$. If $y\brak{\frac{\pi}{3}}=0$, then $y\brak{\frac{\pi}{4}}$ is equal to:  \hfill{[September 2020]}
        \begin{enumerate}
        \item $2+\sqrt{2}$
        \item $\sqrt{2}-2$
        \item $\brak{\frac{1}{\sqrt{2}}}-1$
        \item $2-\sqrt{2}$
        \end{enumerate}
    \item If the system of linear equations \hfill{[September 2020]}
    \begin{align*}
        x+y+3z=0
    \end{align*}
    \begin{align*}
        x+3y+k^2z=0
    \end{align*}
    \begin{align*}
        3x+y+3z=0
    \end{align*}
    has a non-zero solution \brak{x,y,z} for some $k\in R$, then x+\brak{\frac{y}{z}} is equal to : \hfill{[September 2020]}
    \begin{enumerate}
    
        \item -9
        \item 9
        \item -3
        \item 3
       \end{enumerate}

 
 \item Which of the following points lies on the tangent to the curve $4x^3e^y+x^4e^y+2\sqrt{y+1}=3$ at the point \brak{1,0} ? \hfill{[September 2020]}
 \begin{enumerate}
     \item \brak{2,6}
     \item \brak{2,2}
     \item \brak{-2,6}
     \item \brak{-2,4}
 \end{enumerate}
