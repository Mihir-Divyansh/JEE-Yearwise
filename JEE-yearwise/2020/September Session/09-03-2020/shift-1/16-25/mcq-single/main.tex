\iffalse
\title{2020}
\author{EE24BTECH11021}
\section{mcq-single}
\fi
  
  \item If the number of integral terms in the expansion of $\brak{3^{\frac{1}{2}}+5^{\frac{1}{8}}}^n$ is exactly $33$, then the least value of n is $\colon$
    \hfill{[Sep-2020]}
        \begin{enumerate}
            \item $264$
            \item $256$
            \item $128$
            \item $248$
        \end{enumerate}
    \item If $\alpha$ and $\beta$ are the roots of the equation $x^2+px+2=0$ and $\frac{1}{\alpha}$ and $\frac{1}{\beta}$ are the roots of the equation $2x^2+2qx+1=0$, then $\brak{\alpha-\frac{1}{\alpha}}\brak{\beta-\frac{1}{\beta}}\brak{\alpha + \frac{1}{\beta}}\brak{\beta +\frac{1}{\alpha}}$ is equal to$\colon$
    \hfill{[Sep-2020]}
        \begin{enumerate}
            \item $\frac{9}{4}\brak{9+p^2}$
            \item $\frac{9}{4}\brak{9+q^2}$
            \item $\frac{9}{4}\brak{9-p^2}$
            \item $\frac{9}{4}\brak{9-q^2}$
        \end{enumerate}
    \item Let $\sbrak{t}$ denote the greatest integer $\leq t$. If for some $\lambda\in R-\cbrak{0,1}$,\\
    $\lim_{x \to 0} \abs{\frac{1-x+\abs{x}}{\lambda-x+\sbrak{x}}}=L$, then $L$ is equal to$\colon$
   \hfill{[Sep-2020]}
        \begin{enumerate}
            \item $0$
            \item $2$
            \item $\frac{1}{2}$
            \item $1$
        \end{enumerate}
    \item $2\pi -\brak{\sin^{-1}{\frac{4}{5}}+\sin^{-1}{\frac{5}{13}}+\sin^{-1}{\frac{16}{65}}}$ is equal to $\colon$
    \hfill{[Sep-2020]}
        \begin{enumerate}
            \item $\frac{7\pi}{4}$
            \item $\frac{5\pi}{4}$
            \item $\frac{3\pi}{2}$
            \item $\frac{\pi}{2}$
        \end{enumerate}
    \item The proposition $p \rightarrow \sim \brak{p\wedge \sim q}$ is equivalent to $\colon$
    \hfill{[Sep-2020]}
        \begin{enumerate}
            \item $\brak{\sim p}\vee q$
            \item $q$
            \item $\brak{\sim p}\wedge q$
            \item $\brak{\sim p}\vee \brak{\sim q}$
        \end{enumerate}
