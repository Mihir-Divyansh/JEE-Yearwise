\iffalse
\title{Assignment}
\author{K.AKSHAY TEJA}
\section{mcq-single}
\fi
% Question 1
\item Consider the following statements:\\
P : I have fever\\
Q : I will not take medicine\\
R : I will take rest\\
The statement "If I have fever, then I will take
medicine and I will take rest" is equivalent to:  \hfill \brak{Jan 2023}

\begin{multicols}{2}
\begin{enumerate}
    \item $\brak{\brak{- P} \lor - Q} \land \brak{\brak{- P} \lor R}$
    \item $\brak{\brak{- P} \lor - Q} \land \brak{\brak{- P} \lor - R}$
    \item $\brak{P \lor Q} \land \brak{\brak{- P} \lor R}$
    \item $\brak{P \lor - Q} \land \brak{P \lor - R}$
\end{enumerate}
\end{multicols}

% Question 2
\item Let A be a point on the x-axis. Common tangents are drawn from A to the curves $x^2 + y^2 = 8$ and $y^2 = 16x$. If one of these tangents touches the two curves at Q and R, then $\brak{QR}^2$ is equal to   \hfill \brak{Jan 2023}

\begin{multicols}{4}    
\begin{enumerate}
    \item $64$
    \item $76$
    \item $81$
    \item $72$
\end{enumerate}
\end{multicols}


% Question 3
\item Let $q$ be the maximum integral value of $p$ in $\sbrak{0, 10}$ for which the roots of the equation $x^2 - px + \frac{5}{4}p = 0$ are rational. Then the area of the region $\brak{x, y} : 0 \leq y \leq \brak{x - q}^2, 0 \leq x \leq q$ is   \hfill \brak{Jan 2023}

\begin{multicols}{4}    
\begin{enumerate}
    \item $243$
    \item $25$
    \item $\frac{125}{3}$
    \item $164$
\end{enumerate}
\end{multicols}


% Question 4
\item If the functions $f\brak{x} = \frac{x^3}{3} + 2bx + a\frac{x^2}{2} $ and $g\brak{x} = \frac{x^3}{3} + ax + bx^2$, $a \neq 2b$, have a common extreme point, then $a + 2b + 7$ is equal to   \hfill \brak{Jan 2023}

\begin{multicols}{4}
\begin{enumerate}
    \item $4$
    \item $\frac{3}{2}$
    \item $3$
    \item $6$
\end{enumerate}
\end{multicols}

% Question 5
\item The range of the function $f\brak{x} = \sqrt{3 - x} + \sqrt{2 + x}$ is  \hfill \brak{Jan 2023}

\begin{multicols}{4}
\begin{enumerate}
    \item $\sbrak{\sqrt{5}, \sqrt{10}}$
    \item $\sbrak{2\sqrt{2}, \sqrt{11}}$
    \item $\sbrak{\sqrt{5}, \sqrt{13}}$
    \item $\sbrak{\sqrt{2}, \sqrt{7}}$
\end{enumerate}
\end{multicols}

% Question 6
\item The solution of the differential equation $ \frac{dy}{dx} = -\brak{\frac{x^2 + 3y^2}{3x^2 + y^2}}, \, y\brak{1} = 0$  is   \hfill \brak{Jan 2023}

\begin{multicols}{2}
\begin{enumerate}
    \item $\log_e \abs{ x + y } - \frac{xy}{\brak{ x + y }^2} = 0$
    \item $\log_e \abs{ x + y } + \frac{xy}{\brak{ x + y }^2} = 0$
    \item $\log_e \abs{ x + y } + \frac{2xy}{\brak{ x + y }^2} = 0$
    \item $\log_e \abs{ x + y } - \frac{2xy}{\brak{ x + y }^2} = 0$
\end{enumerate}
\end{multicols}

% Question 7
\item Let $x = \brak{ 8\sqrt{3} + 13}^{13}$ and $y = \brak{7\sqrt{2} + 9}^9$. If $\sbrak{t}$ denotes the greatest integer $\leq t$, then   \hfill \brak{Jan 2023}

\begin{multicols}{2}
\begin{enumerate}
    \item $\sbrak{x} + \sbrak{y}$ is even
    \item $\sbrak{x}$ is odd but $\sbrak{y}$ is even
    \item $\sbrak{x}$ is even but $\sbrak{y}$ is odd
    \item $\sbrak{x}$ and $\sbrak{y}$ are both odd
\end{enumerate}
\end{multicols}


% Question 8
\item A vector v in the first octant is inclined to the x-axis at $60^\circ$, to the y-axis at $45^\circ$ and to the z-axis at an acute angle. If a plane passing through the points $\brak{\sqrt{2}, -1, 1}$ and $\brak{a, b, c}$ is normal to $\overrightarrow{v}$, then   \hfill \brak{Jan 2023}

\begin{multicols}{2}
\begin{enumerate}
    \item $\sqrt{2}a + b + c = 1$
    \item $a + b + \sqrt{2}c = 1$
    \item $a + \sqrt{2}b + c = 1$
    \item $\sqrt{2}a - b + c = 1$
\end{enumerate}
\end{multicols}

% Question 9
\item Let $f, g$ and $h$ be the real valued functions defined on $\mathbb{R}$ as $f \brak{x} = \brak{ 
\begin{array}{ll}
\frac{x}{\abs{x}}, & x \neq 0 \\
1, & x = 0 
\end{array}
}$ \\
$g \brak{x} = \brak{ 
\begin{array}{ll}
\frac{\sin \brak{x + 1}}{\brak{x + 1}}, & x \neq -1 \\
1, & x = -1 
\end{array}
}$ and $h\brak{x} = 2\sbrak{x} - f\brak{x}$, where $\sbrak{x}$ is the greatest integer $\leq x$. The value of $\lim_{x \to -1} g \brak{h \brak{x - 1}} $ is   \hfill \brak{Jan 2023}

\begin{multicols}{4}
\begin{enumerate}
    \item 1 
    \item $\sin\brak{1}$ 
    \item -1 
    \item 0
\end{enumerate}
\end{multicols}

% Question 10
\item The number of ways of selecting two numbers $a$ and $b$, where $a \in \{2, 4, 6, \ldots, 100\} $ and $b \in \{1, 3, 5, \ldots, 99\} $ such that $2$ is the remainder when $a + b$ is divided by $23$ is   \hfill \brak{Jan 2023}

\begin{multicols}{4}
\begin{enumerate}
    \item 186 
    \item 54 
    \item 108 
    \item 268 
\end{enumerate}
\end{multicols}


% Question 11
\item If $P$ is a $3 \times 3$ real matrix such that $P^T = aP + \brak{a - 1}I,$ where $a > 1$, then    \hfill \brak{Jan 2023}

\begin{multicols}{2}
\begin{enumerate}
    \item $P$ is a singular matrix
    \item $\abs{\text{Adj } P} > 1$
    \item $\abs{\text{Adj } P} = 1$
    \item $\abs{\text{Adj } P} < 1$
\end{enumerate}
\end{multicols}

% Question 12
\item Let $\lambda \in \mathbb{R}, \overrightarrow{a} = \lambda \hat{i} + 2 \hat{j} - 3 \hat{k}, \overrightarrow{b} = \hat{i} - \lambda \hat{j} + 2 \hat{k}.$ If $\brak{\overrightarrow{a} + \overrightarrow{b}} \times \brak{\overrightarrow{a} \times \overrightarrow{b}} = 8 \hat{i} - 40 \hat{j} - 24 \hat{k},
$ then $\abs{\lambda \brak{\overrightarrow{a} + \overrightarrow{b}} \times \brak{\overrightarrow{a} - \overrightarrow{b}}}^2$ is equal to    \hfill \brak{Jan 2023}

\begin{multicols}{4}
\begin{enumerate}
    \item 140 
    \item 132 
    \item 144 
    \item 136 
\end{enumerate}
\end{multicols}


% Question 13
\item Let $\overrightarrow{a}$ and $\overrightarrow{b}$ be two vectors. Let $\abs{\overrightarrow{a}} = 1, \abs{\overrightarrow{b}} = 4$ and $\overrightarrow{a} \cdot \overrightarrow{b} = 2.$ If $\overrightarrow{c} = 2\overrightarrow{a} - 3\overrightarrow{b},$ then the value of $\overrightarrow{b} \cdot \overrightarrow{c}$ is   \hfill \brak{Jan 2023}

\begin{multicols}{4}
\begin{enumerate}
    \item $-24$ 
    \item $-48$ 
    \item $-84$ 
    \item $-60$ 
\end{enumerate}
\end{multicols}


% Question 14
\item Let $a_1 = 1,a_2,a_3,a_4,\ldots$ be consecutive natural numbers. Then $\tan ^{-1}\brak{\frac{1}{1 + a_1a_2}} + \tan ^{-1} \frac{1}{1 + a_2a_3} + \ldots + \tan ^{-1} \brak{\frac{1}{1 + a_{2021}a_{2022}}}$ is equal to  \hfill \brak{Jan 2023}\begin {multicols}{4}
\begin{enumerate}
    \item $\frac{\pi}{4} - \cot ^{-1}\brak{2022}$ 
    \item $\cot ^{-1}\brak{2022} - \frac{\pi}{4}$ 
    \item $\tan ^{-1} \brak{2022} - \frac{\pi}{4}$ 
    \item $\frac{\pi}{4} - \tan ^{-1}\brak{2022}$ 
\end{enumerate}
\end{multicols}

% Question 15
\item The parabolas: $a x^2 + 2 b x + c y = 0 $ and $d x^2 + 2 e x + f y = 0$ intersect on the line $y = 1$. If $a, b, c, d, e, f$ are positive real numbers and $a, b, c$ are in G.P., then   \hfill \brak{Jan 2023}

\begin{multicols}{4}
\begin{enumerate}
    \item $d, e, f$ are in A.P.
    \item $\frac{d}{a}, \frac{e}{b}, \frac{f}{c}$ are in G.P.
    \item $\frac{d}{a}, \frac{e}{b}, \frac{f}{c}$ are in A.P.
    \item $d, e, f$ are in G.P.
\end{enumerate}
\end{multicols} 
%\end{enumerate}
%\end{document}

