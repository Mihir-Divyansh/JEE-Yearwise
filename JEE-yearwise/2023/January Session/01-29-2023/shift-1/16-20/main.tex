\iffalse
\title{2023}   
\author{AI24Btech11024}
\section{mcq-single}              
\fi
\item Fifteen football players of club-team are given 15 T-shirts with their names written on the backside. If the players pick up the T-shirts randomly, then the probability that at least 3 players pick the correct T-shirt is \hfill{\brak{\text{Feb 2023}}}
\begin{enumerate} 
    \item $\frac{5}{24}$
    \item $\frac{2}{15}$
    \item $\frac{1}{6}$
    \item $\frac{5}{36}$
\end{enumerate}
\item Let \\$f\brak{\theta} = 3 \brak{ \sin^{4}\brak{ \frac{3\pi}{2} - \theta } + \sin^{4}\brak{ 3\pi + \theta } } - 2 \brak{ 1 - \sin^{2}\brak{2\theta} }$
\\and $S= \brak{ \theta \in \brak{0, \pi} : f'\brak{\theta} = -\frac{\sqrt{3}}{2} }$. If $ 4\beta = \sum_{\theta \in S} \theta $, then $ f\brak{\beta} $ is equal to \hfill{\brak{\text{Feb 2023}}}
\begin{enumerate}
    \item $\frac{11}{8}$
    \item $\frac{5}{4}$
    \item $\frac{9}{8}$
    \item $\frac{3}{2}$
\end{enumerate}
\item If p, q and r are the three propositions, then which of the following combinations of the truth values of p, q and r makes the logical expression 
$\cbrak{\brak{p \lor q}\land \brak{\brak{\sim p}\lor r}} \to \brak{\brak{\sim q} \lor r}$ false ? \hfill{\brak{\text{Feb 2023}}}
\begin{enumerate}
    \item $p=T,q=F,r=T$
    \item $p=T,q=T,r=F$
    \item $p=F,q=T,r=F$
    \item $p=T,q=F,r=F$
\end{enumerate}
\item There rotten apples are mixed accidently withseven good apples and four apples are drawn oneby one without replacement. Let the random
variable X denote the number of rotten apples. If $\mu$and $\sigma^{2}$ represent mean and variance of X,respectively, $10\brak{\mu^{2}+\sigma^{2}}$ then  is equal to \hfill{\brak{\text{Feb 2023}}}
\begin{enumerate}
    \item 20
    \item 250
    \item 25
    \item 30
\end{enumerate}
\item Let $y=f\brak{x}$ be the solution of the differential equation $y\brak{x+1}dx-x^{2}dy=0,y\brak{1}=e$. The $\lim_{x\to0^{+}}f\brak{x}$ is equal to  \hfill{\brak{\text{Feb 2023}}}

\begin{enumerate}
    \item 0
    \item $\frac{1}{e}$
    \item $e^{2}$
    \item $\frac{1}{e^{2}}$
\end{enumerate}

