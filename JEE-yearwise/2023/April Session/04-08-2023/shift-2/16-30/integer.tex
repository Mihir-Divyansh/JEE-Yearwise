\iffalse
\title{April 2023, Shift - 2}
\author{EE24BTECH11062}	
\section{integer}
\fi
%\begin{enumerate}
\item Let $R=\{a,b,c,d,e\}$ and $S=\{1,2,3,4\}$. Total number of onto functions $f: R\mapsto S$ such that $f\brak{a}\neq 1$, is equal to \hfill{[April 2023]}

\item Let $m$ and $n$ be the number of real roots of the quadratic equations $x^2-12x+\sbrak{x}+31=0$ and $x^2-5\abs{x+2}-4=0$ respectively, where \sbrak{x} denotes the greatest integer $\leq x$. Then $m^2+mn+n^2$ is equal to \hfill{[April 2023]}

\item Let $P_1$ be the plane $3x-y-7z=11$ and $P_2$ be the plane passing through points \brak{2,-1,0},\brak{2,0,-1}, and \brak{5,1,1}. If the foot of the perpendicular drawn from the point \brak{7,4,-1} on the line of intersection of the planes $P_1$ and $P_2$ is \brak{\alpha,\beta,\gamma}, then $\alpha+\beta+\gamma$ is equal to \hfill{[April 2023]}

\item If the domain of the function $\ln \brak{\frac{6x^2+5x+1}{2x-1}}+\cos^{-1}\brak{\frac{2x^2-3x+4}{3x-5}}$ is $\brak{\alpha,\beta}\cup(\gamma,\delta]$, then , $18\brak{\alpha^2+\beta^2+\gamma^2+\delta^2}$ is equal to \hfill{[April 2023]}

\item Let the area enclosed by the lines $x+y=2, y=0, x=0$ and the curve $f\brak{x}=min\{x^2+\frac{3}{4},1+\sbrak{x}\}$ where \sbrak{x} denotes the greatest integer $\leq x$, be $A$, then the value of $12A$ is \hfill{[April 2023]}

\item Let $0<z<y<x$ be three real numbers such that $\frac{1}{x},\frac{1}{y}, \frac{1}{z}$ are in an arithmetic progression and $x,\sqrt{2}y, z$ are in a geometric progression. If $xy+yz+zx=\frac{3}{\sqrt{2}}xyz$, then $3\brak{x+y+z}^2$ is equal to \hfill{[April 2023]}

\item Let the solution curve $x=x\brak{y}, 0<y<\frac{\pi}{2}$, of the differential equation $\brak{\ln\brak{\cos y}}^2\cos y dx-\brak{1+3x\ln \brak{\cos y}}\sin y dy =0$ satisfy $x\brak{\frac{\pi}{3}}=\frac{1}{2\ln 2}$. If $x\brak{\frac{\pi}{6}}=\frac{1}{\ln m-\ln n}$, where $m$ and $n$ are co-prime, then $mn$ is equal to  \hfill{[April 2023]}

\item Let $\sbrak{t}$ denote the greatest integer function. If $\int_{0}^{2.4} \sbrak{x^2} \, dx
=\alpha+\beta\sqrt{2}+\gamma\sqrt{3}+\delta\sqrt{5},$ then $\alpha+\beta+\gamma+\delta$ is equal to \hfill{[April 2023]}

\item The ordinates of the points $\vec{P}$ and $\vec{Q}$ on the parabola with focus \brak{3,0} and directrix $x=-3$ are in the ratio $3:1$. If $\vec{R}\brak{\alpha,\beta}$ is the point of intersection of the tangents to the parabola at $\vec{P}$ and $\vec{Q}$, then $\frac{\beta^2}{\alpha}$ is equal to \hfill{[April 2023]}

\item Let $k$ and $m$ be positive real numbers such that the function \\$f\brak{x}=$
$\begin{cases}
    3x^2+k\sqrt{x+1}, 0<x<1\\
    mx^2+k^2, x\geq 1 
\end{cases} $ \\is differentiable for all $x>0$. Then $\frac{8f^{\prime}\brak{8}}{f^{\prime}\brak{\frac{1}{8}}}$ is equal to \hfill{[April 2023]}
%\end{enumerate}
%\end{document}

