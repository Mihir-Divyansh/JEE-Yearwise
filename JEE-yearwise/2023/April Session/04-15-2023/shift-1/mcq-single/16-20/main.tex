\iffalse
\title{Assignment}
\author{EE24BTECH11038}
\section{mcq-single}
\fi
\item The number of real roots of the equation x$\abs{x}-5\abs{x+2}+6=0$, is \hfill{Apirl 2023}
\begin{enumerate}
    \item 5
    \item 6
    \item 4
    \item 3
\end{enumerate}
\bigskip
\item  Let the system of linear equations 
\begin{align*}
-x+2y-9z=7\\
-x+3y+7z=9\\
-2x+y+5z=8\\
-3x+y+13z=\lambda
\end{align*}
has a unique solution $x=\alpha,y=\beta,z=\gamma$. then the distance between the point $\brak{\alpha,\beta,\gamma}$ from the plane 2x-2y+z=$\lambda$ is \hfill{Apirl 2023}
\begin{enumerate}
    \item 7
    \item 9
    \item 13
    \item 11
\end{enumerate}
\bigskip
\item Let $A_1$ and $A_2$ be two  arithmetic means and $G_1,G_2,G_3$ be three geometric means of the two distinct positive numbers . Then $G_1^{4}+G_2^{4}+G_3^{4}+G_1^{2}G_3^{2}$ is equal to \hfill{Apirl 2023}
\begin{enumerate}
    \item $2\brak{A_1+A_2}G_1G_3$
    \item $\brak{A_1+A_2}^2G_1G_3$
    \item $2\brak{A_1+A_2}G_1^{2}G_3^{2}$
     \item $\brak{A_1+A_2}G_1^{2}G_3^{2}$
\end{enumerate}
\bigskip
\item The negation of $ \brak{p \lor q} \land \brak{\sim p \lor q} $ is: \hfill{Apirl 2023}
\begin{enumerate}
    \item $\brak{\brak{\sim p \land q}}\land q$
    \item $\sim\brak{p\lor q}$
    \item $p\lor q$
    \item $\brak{\brak{\sim p \land q}}\lor p$
\end{enumerate}
\bigskip
\item The total number of three-digit numbers, divisible by 3, which can be formed using the digits 1, 3, 5, 8, if
repetition of digits is allowed, is \hfill{Apirl 2023}
\begin{enumerate}
    \item 21
    \item 18
    \item 20
    \item 22
\end{enumerate}
