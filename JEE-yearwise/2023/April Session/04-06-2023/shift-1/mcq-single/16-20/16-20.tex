\iffalse
    \title{2023}
    \author{EE24BTECH11005}
    \section{mcq-single}
\fi 
\item If $2x^y+3y^x=20$ then $\frac{dy}{dx}$ at $\myvec{2\\2}$ is equal to,
		\hfill{\brak{\text{Apr 2023}}} \\ \begin{enumerate}
				\begin{multicols}{2}
				\item $-\brak{\frac{3+\log_e8}{2+\log_e4}}$
				\columnbreak
			\item $-\brak{\frac{2+\log_e8}{3+\log_e4}}$
				\end{multicols}
				\begin{multicols}{2}
				\item $-\brak{\frac{3+\log_e4}{2+\log_e8}}$
				\columnbreak
			\item $-\brak{\frac{3+\log_e16}{4+\log_e8}}$
				\end{multicols}
		\end{enumerate}
	\item If the system of equations
		\begin{align*}
			&x+y+az=b\\
			&2x+5y+2z=6\\
			&x+2y+3z=3
		\end{align*}
		has infinitely many solutions, then $2a+3b$ is equal to, 
		\hfill{\brak{\text{Apr 2023}}} \\ \begin{enumerate}
				\begin{multicols}{2}
					\item $28$
				\columnbreak
					\item $20$
				\end{multicols}
				\begin{multicols}{2}
					\item $25$
				\columnbreak
					\item $23$
				\end{multicols}
		\end{enumerate}
	\item Let $\brak{1+x+2x^2}^{20}=a_0+a_1x+a_2x^2+\dots+a_{40}x^{40}$. Then, $a_1+a_3+a_5+\dots+a_{37}$is equal to,
		\hfill{\brak{\text{Apr 2023}}} \\ \begin{enumerate}
				\begin{multicols}{2}
				\item $2^{20}\brak{2^{20}+21}$
				\columnbreak
			\item $2^{19}\brak{2^{20}+21}$
				\end{multicols}
				\begin{multicols}{2}
				\item $2^{20}\brak{2^{20}-21}$
				\columnbreak
			\item $2^{19}\brak{2^{20}-21}$
				\end{multicols}
		\end{enumerate}
	\item Let $5f\brak{x}+4f\brak{\frac{1}{x}}=\frac{1}{x}+3, x>0$, then $\int_1^2f\brak{x}dx$ is equal to,
			\hfill{\brak{\text{Apr 2023}}} \\ \begin{enumerate}
				\begin{multicols}{2}
					\item $10\log_e2-6$
				\columnbreak
					\item $10\log_e2+6$
				\end{multicols}
				\begin{multicols}{2}
					\item $5\log_e-3$
				\columnbreak
					\item $5\log_e2+3$
				\end{multicols}
		\end{enumerate}
	\item The mean and variance of a set of $15$ numbers are $12$ and $14$ respectively. The mean and variance of another set of $15$ numbers are $14$ and $\sigma^2$ respectively. If the variance of all the $30$ numbers in the two sets is $13$, then $\sigma^2$ is equal to,
		\hfill{\brak{\text{Apr 2023}}} \\ \begin{enumerate}
				\begin{multicols}{2}
				\item $12$
				\columnbreak
			\item $10$
				\end{multicols}
				\begin{multicols}{2}
				\item $11$
				\columnbreak
			\item $9$
				\end{multicols}
			\end{enumerate}
