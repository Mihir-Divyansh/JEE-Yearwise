\iffalse
  \title{Assignment}
  \author{ee24btech11030}
  \section{integer}
\fi

%   \begin{enumerate}
 \item Let A=\{-4,-3,-2,0,1,3,4\} and R = \{(a , b) $\in$ A$\times$A : b = $|a|$ or $b^2$ = a + 1\} be a relation on A. Then the minimum number of elements, that must be added to the relation R so that it becomes reflexive and symmetric, is  \underline{\hspace{1cm}}.\hfill{[APR 2023]}
    \bigskip
    
    \item Let $f_n = \int_{0}^{\frac{\pi}{2}}\left(\sum_{k=1}^{n}\sin^{(k - 1)}{x}\right)\left(\sum_{k=1}^{n}(2k - 1)\sin^{(k - 1)}{x}\right)\cos{x}dx , n \in N$. Then $f_{21} - f_{20}$ is equal to \underline{\hspace{1cm}}.\hfill{[APR 2023]}
    \bigskip
    
    \item If y = y(x) is the solution of the differential equation 
    $\frac{dy}{dx} + \frac{4x}{x^2 - 1}y = \frac{x + 2}{(x^2 - 1)^{\frac{5}{2}}} , x > 1$ such that y(2) = $\frac{2}{9}\ln{2 + \sqrt{3}}$ and y($\sqrt{2}$) = $\alpha\ln{\sqrt{\alpha} + \beta} + \beta - \sqrt{\gamma} , \alpha,\beta,\gamma \in N$ then $\alpha\beta\gamma$ is equal to \underline{\hspace{1cm}}.\hfill{[APR 2023]}
    \bigskip
    
    \item Total numbers of 3-digit numbers that are divisible by 6 and can be formed by using the digits 1, 2, 3, 4, 5 with repetition, is \underline{\hspace{1cm}}\hfill{[APR 2023]}
    \bigskip
    
    \item The remainder, when $7^{103}$ is divided by 17, is \underline{\hspace{1cm}}.\hfill{[APR 2023]}
    \bigskip
    
    \item Let f(x) = $\sum_{k=1}^{10}kx^k , x \in R $ If 2f(2) - f'(2) = 119($2^n$) + 1 then n is equal to \underline{\hspace{1cm}}. \hfill{[APR 2023]}
    \bigskip
    
    \item For x $\in$ (-1 , 1], the number of solutions of the equation $sin^{-1}{x} = 2 tan^{-1}{x}$ is equal to \underline{\hspace{1cm}}. \hfill{[APR 2023]}
    \bigskip
    
    \item The mean and standard deviation of the marks of 10 students were found to be 50 and 12 respectively, Later, it was observed that two marks 20 and 25 were wrongly read as 45 and 50 respectively. Then the correct variance is  \underline{\hspace{1cm}}. \hfill{[APR 2023]}
    \bigskip
    
    \item The foci of a hyperbola are ($\pm2 , 0$) and its eccentricity is $\frac{3}{2}$. A tangent, perpendicular to the line 2x + 3y = 6,is drawn at a point in the first quadrant on the hyperbola. If the intercepts made by the tangent on the x and y - axes are a and b respectively, then $|6a| + |5b|$ is equal to \underline{\hspace{1cm}}. \hfill{[APR 2023]}
    \bigskip
    
    \item Let [$\alpha$] denote the greatest integer $\leq \alpha$. Then  [$\sqrt1$] + [$\sqrt2$] + [$\sqrt3$] + $\cdots$ + [$\sqrt{120}$]  is equal to\underline{\hspace{1cm}}.\hfill{[APR 2023]}
%   \end{enumerate}
